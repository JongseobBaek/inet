\chapter{Other Network Protocols}
\label{cha:other-network-protocols}

\section{Overview}
\label{sec:networkprotocols:overview}

Network layer protocols in INET are not restricted to IPv4 and IPv6. INET nodes such as
\nedtype{Router} and \nedtype{StandardHost} can be configured to use an alternative
network layer protocols instead of, or in addition to, IPv4 and IPv6.

Node models contain three optional network layers that can be individually
turned on or off:

\begin{ned}
ipv4: <ipv4NetworkLayerType> like INetworkLayer if hasIpv4;
ipv6: <ipv6NetworkLayerType> like INetworkLayer if hasIpv6;
generic: <networkLayerType> like INetworkLayer if hasGn;
\end{ned}

In the default configuration, only IPv4 is turned on. If you want to use an
alternative network layer protocol instead of IPv4/IPv6, your configuration will
look something like this:

\begin{inifile}
**.hasIpv4 = false
**.hasIpv6 = false
**.hasGn = true
**.networkLayerType = "WiseRouteNetworkLayer"
\end{inifile}

The list of alternative network layers includes:

\begin{itemize}
  \item \nedtype{SimpleNetworkLayer} is a generic network layer where the
    concrete protocol type is a parameter
  \item \nedtype{NextHopNetworkLayer} is a network layer specialized
    for the ``Next-Hop Forwarding Protocol'', an abstract implementation of the
    next-hop routing concept
  \item \nedtype{WiseRouteNetworkLayer} is specialized for the Wise Route protocol
\end{itemize}

The list of network layer protocols that can be plugged into
\nedtype{SimpleNetworkLayer} includes:

\begin{itemize}
  \item \nedtype{Flooding} implements controlled flooding
  \item \nedtype{WiseRoute} implements Wise Route, a convergecasting protocol for wireless sensor networks
  \item \nedtype{ProbabilisticBroadcast} implements a multi-hop ad-hoc data dissemination protocol
  \item \nedtype{AdaptiveProbabilisticBroadcast} is a variant of the previous one
\end{itemize}

In addition to the network layer protocol, \nedtype{SimpleNetworkLayer}
includes an instance of \ttt{GlobalArp} for address resolution,
and an instance of \nedtype{EchoProtocol}, a module type that
implements a simple \textit{ping}-like protocol.

All the above network protocols can work with IPv4 addresses, IPv6 addresses,
use MAC address as network address (this is sometimes useful in WSNs),
or use sythetic addresses that only make sense within the simulation.
\footnote{This is possible because the implementation of these modules
simply use the \ttt{L3Address} C++ class, which is a variant type capable of
holding several types of L3 addresses.}

In apps, you need to specify which network layer protocol you want to use.
For example:

\begin{inifile}
**.app[*].networkProtocol = "flood"
\end{inifile}

TODO all apps support that?

\section{Protocols}
\label{sec:networkprotocols:protocols}

\subsection{Flooding}
\label{sec:networkprotocols:flooding}

\nedtype{Flooding} is a simple flooding protocol for network-level broadcast.
It remembers already broadcast messages, and does not rebroadcast
them if it gets another copy of that message.

\subsection{ProbabilisticBroadcast}
\label{sec:networkprotocols:probabilisticbroadcast}

\nedtype{ProbabilisticBroadcast} is a multi-hop ad-hoc data dissemination
protocol based on probabilistic broadcast.

This method reduces the number of packets sent on the channel (reducing the
broadcast storm problem) at the risk of some nodes not receiving the data.
It is particularly interesting for mobile networks.

The transmission probability for each attempt, the time between two transmission
attempts, the maximum number of broadcast transmissions of a packet, and
some other settings are parameters.

\subsection{AdaptiveProbabilisticBroadcast}
\label{sec:networkprotocols:adaptiveprobabilisticbroadcast}

\nedtype{AdaptiveProbabilisticBroadcast} is a variant of
\nedtype{ProbabilisticBroadcast} that automatically adjusts transmission
probabilities depending on the estimated number of neighbours.

\subsection{WiseRoute}
\label{sec:networkprotocols:wiseroute}

\nedtype{WiseRoute} implements Wise Route, a simple loop-free routing algorithm
that builds a routing tree from a central network point, designed for sensor
networks and convergecast traffic (WIreless SEnsor routing).

The sink (the device at the center of the network) broadcasts a route building
message. Each network node that receives it selects the sink as parent in the
routing tree, and rebroadcasts the route building message. This procedure
maximizes the probability that all network nodes can join the network, and
avoids loops.

The \fpar{sinkAddress} parameter specifies the sink network address,
\fpar{rssiThreshold} is a threshold to avoid using bad links (with too low RSSI
values) for routing, and \fpar{routeFloodsInterval} should be set to zero for
all nodes except the sink. Each \fpar{routeFloodsInterval}, the sink restarts
the tree building procedure. Set it to a large value if you do not want the tree
to be rebuilt.

\subsection{NextHopForwarding}
\label{sec:networkprotocols:nexthopforwarding}

The \nedtype{NextHopForwarding} module is an implementation of the next-hop
forwarding concept. (It can be thought of as an abstracted version of IP.)

The protocol needs an additional module, a \nedtype{NextHopRoutingTable} for its
operation. The routing table module is included in the
\nedtype{NextHopNetworkLayer} compound module.


\section{Address Types}
\label{sec:networkprotocols:address-types}

The following address types are available:

\begin{itemize}
  \item IPv4 address
  \item IPv6 address
  \item MAC address
  \item module ID
  \item module path
\end{itemize}

Protocols described in this chapter work with ``generic'' L3 addresses,
they can use any address type.

When choosing IPv4 addresses, an \nedtype{IPv4NetworkConfigurator} global
instance can be used to assign addresses to network interfaces. (Note that
\nedtype{IPv4NetworkConfigurator} also needs a per-node instance
of \nedtype{Ipv4NodeConfigurator} for it to work.)

\section{Address Resolution}
\label{sec:networkprotocols:address-resolution}

Address resolution is done by \nedtype{GlobalArp}.
If the address type is IPv4, \nedtype{Arp} can be used instead of
\nedtype{GlobalArp}.


\section{/////////////////////////////////////////////////////////}


\section{InternetCloud}

This module is an IPv4 router that can delay or drop packets (while retaining
their order) based on which interface card the packet arrived on and 
on which interface It is leaving the cloud. The delayer module is replacable.

By default the delayer module is ~MatrixCloudDelayer which lets you configure
the delay, drop and datarate parameters in an XML file. Packet flows, as defined
by incoming and outgoing interface pairs, are independent of each other.

The ~InternetCloud module can be used only to model the delay between two hops, but
it is possible to build more complex networks using several ~InternetCloud modules.

\section{PIM}

Protocol Independent Multicast -- not a network protocol 

models: \nedtype{PimSm}, \nedtype{PimDm}; \nedtype{Pim} is a compound module



%%% Local Variables:
%%% mode: latex
%%% TeX-master: "usman"
%%% End:

