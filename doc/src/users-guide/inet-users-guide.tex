\documentclass{book}
\usepackage{a4wide}

%% possible fonts -- in order of preference
%%\usepackage{palatino}
\usepackage{bookman}
%%\usepackage{charter}
%%\usepackage{newcent}
%%\usepackage{times}
%%\usepackage{avant}
%%\usepackage{helvet}
%%\usepackage{sans}
%%\usepackage{chancery}

\usepackage[svgnames]{xcolor}	% for color text support
\usepackage[T1]{fontenc}
\usepackage[11pt]{moresize}
\usepackage{setspace}
\usepackage{ifpdf}
\usepackage{verbatim}   % for the comment environment
\usepackage{makeidx}
\usepackage{longtable}  %% page wrapping table environment
\usepackage{colortbl}   %% colors for tables
\usepackage{fancyvrb}   %% the "Verbatim" environment
\usepackage{fancyhdr}   %% custom headers and footers
\usepackage{multicol}
%% \usepackage{enumitem}   %% compact bullet lists with \begin{itemize}[noitemsep]
\usepackage{csquotes}   %% for the "displayquote" environment
\usepackage{listings}   %% source code listings with syntax highlight (lstxxx commands)
\usepackage[tight]{shorttoc}   %% for generating a second table of contents, only containing chapter titles
\usepackage{bytefield}  %% for drawing protocol frames
\usepackage{paralist}   %% for compact lists
\usepackage[nottoc]{tocbibind}  %% makes Bibliography and Index show up in TOC
\settocbibname{References}

\setlength{\textwidth}{160mm}
%\setlength{\oddsidemargin}{12.5mm}
%\setlength{\evensidemargin}{12.5mm}
%\setlength{\topmargin}{0mm}
\setlength{\textheight}{220mm}
%\setlength{\parskip}{1ex}
%\setlength{\parindent}{5ex}

\renewcommand{\bottomfraction}{0.9}
\renewcommand{\topfraction}{0.9}
\renewcommand{\floatpagefraction}{0.9}

\newenvironment{htmlonly}{\expandafter\comment}{\expandafter\endcomment}
\newcommand{\pdfonly}{}

%% try to cure overfull hboxes
%% \tolerance=500

%% for navigation in dvi files, only needed by old teTeX versions
%%\usepackage{srcltx}

%% try this for spell checking: cat ess2002.tex | ispell -l -t -a | sort | uniq | more

%%
%% The following snippet changes the horizontal spacing between the number and
%% the title in the table of contents.
%%
%% http://tex.stackexchange.com/questions/33841/how-to-modify-the-space-between-the-numbers-and-text-of-sectioning-titles-in-the
%%
\makeatletter
 \renewcommand*\l@section{\@dottedtocline{1}{2em}{3em}}
 \renewcommand*\l@subsection{\@dottedtocline{2}{5em}{4em}}
\renewcommand*\l@chapter[2]{%
  \ifnum \c@tocdepth >\m@ne
    \addpenalty{-\@highpenalty}%
    \vskip 1.0em \@plus\p@
    \setlength\@tempdima{2em}%
    \begingroup
      \parindent \z@ \rightskip \@pnumwidth
      \parfillskip -\@pnumwidth
      \leavevmode \bfseries
      \advance\leftskip\@tempdima
      \hskip -\leftskip
      #1\nobreak\hfil \nobreak\hb@xt@\@pnumwidth{\hss #2}\par
      \penalty\@highpenalty
    \endgroup
  \fi}
\makeatother

%%
%% OMNeT++ logo, use as {\opp}
%%
\makeatletter
%%\DeclareRobustCommand{\omnetpp}{OM\-NeT\kern-.18em++\@}
\DeclareRobustCommand{\omnetpp}{OMNeT++\@}
\makeatother

\newcommand{\opp}{\omnetpp}

%%
%% PDF Header
%%
% note: \ifpdf now comes from the ifpdf package
%\newif\ifpdf
%\ifx\pdfoutput\undefined
%  \pdffalse
%\else
%  \pdfoutput=1
%  \pdftrue
%\fi
%% PDF-Info
\ifpdf
  \usepackage[pdftex]{graphicx}
  \usepackage[plainpages=false,linktocpage,bookmarksnumbered=true,pdftex]{hyperref}   %% automatic hyperlinking
  \pdfcompresslevel=9
  \pdfinfo{/Author (Andras Varga and others)
    /Title (INET Framework User's Guide)
    /Subject ()
    /Keywords (INET, INETMANET, OMNeT++, manual)}
\else
  \usepackage{graphicx}
  \usepackage[plainpages=false]{hyperref}   %% automatic hyperlinking
\fi

%%
%% Draft conditional to include unfinished parts
%%
\newif\ifdraft
%\draftfalse %% uncomment for final version
\drafttrue %% uncomment for draft version

%%
%% Generate Index
%%
\makeindex


%%
%% Link colors (hyperref package)
%%
\definecolor{MyDarkBlue}{rgb}{0.16,0.16,0.5}
%% XXX the next line apparently screws up all links except in TOC! they'll be colored nicely, but won't work.
%\hypersetup{
%    colorlinks=true,
%    linkcolor=MyDarkBlue,
%    anchorcolor=MyDarkBlue,
%    citecolor=MyDarkBlue,
%    filecolor=MyDarkBlue,
%    menucolor=MyDarkBlue,
%    runcolor=MyDarkBlue,
%    urlcolor=blue,
%}

%%
%% Heading and Footer
%%
\pagestyle{fancy}
\fancyhf{}
\renewcommand{\footrulewidth}{0.5pt}
\renewcommand{\chaptermark}[1]{\markboth{#1}{}}
\lhead{INET Framework User's Guide -- \leftmark}
\rfoot{\thepage}

%% this is used for chapter start pages
\fancypagestyle{plain}{
    \rfoot{\thepage}
}

%%
%% Use \begin{graybox}...\end{graybox} for notes
%%
\definecolor{MyGray}{rgb}{0.85,0.85,1.0}
\makeatletter\newenvironment{graybox}%
   {\begin{flushright}\begin{lrbox}{\@tempboxa}\begin{minipage}[r]{0.95\textwidth}}%
   {\end{minipage}\end{lrbox}\colorbox{MyGray}{\usebox{\@tempboxa}}\end{flushright}}%
\makeatother


\newenvironment{note}{\begin{graybox}\textbf{NOTE: }}{\end{graybox}}
\newenvironment{hint}{\begin{graybox}\textbf{HINT: }}{\end{graybox}}
\newenvironment{warning}{\begin{graybox}\textbf{WARNING: }}{\end{graybox}}
\newenvironment{caution}{\begin{graybox}\textbf{CAUTION: }}{\end{graybox}}
\newenvironment{rationale}{\begin{graybox}\textbf{Rationale: }}{\end{graybox}}
\newenvironment{important}{\begin{graybox}\textbf{IMPORTANT: }}{\end{graybox}}

%%
%% Set up listings package
%%
\lstloadlanguages{C++,make,perl,tcl,XML,R,Matlab}

%% See listings.pdf,pp20
\lstdefinelanguage{NED} {
    morekeywords={allowunconnected,bool,channel,channelinterface,connections,const,
                  default,double,extends,false,for,gates,if,import,index,inout,input,
                  int,like,module,moduleinterface,network,output,package,parameters,
                  property,simple,sizeof,string,submodules,this,true,types,volatile,
                  xml,xmldoc},
    sensitive=true,
    morecomment=[l]{//},
    morestring=[b]",
}
\lstdefinelanguage{MSG} {
    morekeywords={abstract,bool,char,class,cplusplus,double,enum,extends,false,
                  fields,int,long,message,namespace,noncobject,packet,properties,
                  readonly,short,string,struct,true,unsigned},
    sensitive=true,
    morecomment=[l]{//},
    morestring=[b]",
}
\lstdefinelanguage{inifile} {
    morekeywords={},
    sensitive=true,
    morecomment=[l]{\#},
    morestring=[b]",
}
\lstdefinelanguage{pseudocode} {
    morekeywords={if,then,else,otherwise,whenever,while},
    sensitive=true,
    morecomment=[l]{//},
    morestring=[b]",
    mathescape=true,
}

%% thick ruler on the left; also, designate backtick as LaTeX escape character
%% (e.g. \opp needs to be written as `\opp` inside listing blocks)
\lstset{
    escapechar=`,
    basicstyle=\ttfamily,
    identifierstyle=\color{Black},
    stringstyle=\color{DarkBlue},
    commentstyle=\color{SeaGreen},
    keywordstyle=\bfseries\color{Purple},
    showstringspaces=false,
    frame=leftline,
    framesep=10pt,
    framerule=3pt,
    xleftmargin=15pt
}

\definecolor{NEDRulerColor}{rgb}{0.5,1.0,0.5}  % pale green
\definecolor{MSGRulerColor}{rgb}{0.5,1.0,0.5}  % pale green
\definecolor{CPPRulerColor}{rgb}{0.8,0.5,0.2}  % pale orange
\definecolor{IniRulerColor}{rgb}{0.9,0.9,0.3}  % pale yellow
\definecolor{FileListingRulerColor}{rgb}{0.85,0.85,0.85}  % grey
%\definecolor{CommandLineRulerColor}{rgb}{0.9,0.9,0.2}
\definecolor{PseudoCodeRulerColor}{rgb}{0.0,1.0,1.0}  % cyan
\definecolor{XMLRulerColor}{rgb}{0.8,0.8,1.0}  % pale blue

%% See listings.pdf,pp39
\lstnewenvironment{ned}
    {\lstset{language=NED,rulecolor=\color{NEDRulerColor}}}
    {}
\lstnewenvironment{msg}
    {\lstset{language=MSG,rulecolor=\color{MSGRulerColor}}}
    {}
\lstnewenvironment{cpp}
    {\lstset{language=C++,rulecolor=\color{CPPRulerColor}}}
    {}
\lstnewenvironment{inifile}
    {\lstset{language=inifile,rulecolor=\color{IniRulerColor}}}
    {}
\lstnewenvironment{filelisting}
    {\lstset{language={},rulecolor=\color{FileListingRulerColor}}}
    {}
\lstnewenvironment{commandline}
    {\lstset{language={},framesep=11pt,framerule=1pt,xleftmargin=16pt}}
    {}
\lstnewenvironment{pseudocode}
    {\lstset{language=pseudocode,rulecolor=\color{PseudoCodeRulerColor}}}
    {}
\lstnewenvironment{XML}
    {\lstset{language=XML,rulecolor=\color{XMLRulerColor}}}
    {}

% add caption={#2} to display caption
\newcommand{\xmlsnippet}[2]{%
    \lstinputlisting[language=XML,rulecolor=\color{XMLRulerColor},linerange=<!\-\-#1\-\->-<!\-\-End\-\->,includerangemarker=false,firstnumber=0]{Snippets.xml}}
\newcommand{\cppsnippet}[2]{%
    \lstinputlisting[language=C++,rulecolor=\color{CPPRulerColor},linerange=//!#1-//!End,includerangemarker=false,firstnumber=0]{Snippets.cc}}
\newcommand{\msgsnippet}[2]{%
    \lstinputlisting[language=msg,rulecolor=\color{MSGRulerColor},linerange=//!#1-//!End,includerangemarker=false,firstnumber=0]{Snippets.msg}}
\newcommand{\nedsnippet}[2]{%
    \lstinputlisting[language=ned,rulecolor=\color{NEDRulerColor},linerange=//!#1-//!End,includerangemarker=false,firstnumber=0]{Snippets.ned}}
\newcommand{\inisnippet}[2]{%
    \lstinputlisting[language=inifile,rulecolor=\color{IniRulerColor},linerange=\#!#1-\#!End,includerangemarker=false,firstnumber=0]{Snippets.ini}}

%%
%% some customization
%%
\setlength{\parindent}{0pt}
\setlength{\parskip}{1ex}

%%
%% Shortcuts
%%
\newcommand{\appendixchapter}{\chapter} %% html converter needs to know which chapters are appendices

\newcommand{\tbf}{\textbf} %% bold faced text
\newcommand{\ttt}{\texttt} %% type writer font text

\newcommand{\tab}{\hspace*{5mm}} %% tabulator settings

\newcommand{\new}{$^{New!}$}
\newcommand{\changed}{$^{Changed!}$}

\newcommand{\program}{\textbf}

\newcommand{\includepng}{\includegraphics}
\newcommand{\includesvg}{\includegraphics}

%% Colordefinition for table header rows (requires package colortbl)
\newcommand{\tabheadcol}{\rowcolor[gray]{0.8}}

%%
%% Function/Class/Macro/Variable/Program/Parameter/Define names
%%
%% Write the names in type writer font and do an index entry
%% Allows word wrap by automatic hyphenation
%%
%% Usage: \ffunc{take()}
%%    or: \ffunc[take()]{take(obj)}
%% the second form uses the bracketed word for the index entry
%%

\newcommand{\protocol}[1]{%
    {#1}}

%% NED type names
\newcommand{\nedtype}[2][\DefaultOpt]{\def\DefaultOpt{#2}%
  \index{#1}%
  \texttt{\hyphenchar\font=`\-\relax#2}}

%% MSG type names
\newcommand{\msgtype}[2][\DefaultOpt]{\def\DefaultOpt{#2}%
  \index{#1}%
  \texttt{\hyphenchar\font=`\-\relax#2}}

%% Function names
\newcommand{\ffunc}[2][\DefaultOpt]{\def\DefaultOpt{#2}%
  \index{#1}%
  \texttt{\hyphenchar\font=`\-\relax#2}}

%% Class names
\newcommand{\cppclass}[2][\DefaultOpt]{\def\DefaultOpt{#2}%
  \index{#1}%
  \texttt{\hyphenchar\font=`\-\relax#2}}

%% Macro names
\newcommand{\fmac}[2][\DefaultOpt]{\def\DefaultOpt{#2}%
  \index{#1}%
  \texttt{\hyphenchar\font=`\-\relax#2}}

%% Variable names
\newcommand{\fvar}[2][\DefaultOpt]{\def\DefaultOpt{#2}%
  \index{#1}%
  \texttt{\hyphenchar\font=`\-\relax#2}}

%% Program names
\newcommand{\fprog}[2][\DefaultOpt]{\def\DefaultOpt{#2}%
  \index{#1}%
  \texttt{\hyphenchar\font=`\-\relax#2}}

%% Parameter names
\newcommand{\fpar}[2][\DefaultOpt]{\def\DefaultOpt{#2}%
  \index{#1}%
  \texttt{\hyphenchar\font=`\-\relax#2}}

%% Defines
\newcommand{\fdef}[2][\DefaultOpt]{\def\DefaultOpt{#2}%
  \index{#1}%
  \texttt{\hyphenchar\font=`\-\relax#2}}

%% NED/MSG properties
\newcommand{\fprop}[2][\DefaultOpt]{\def\DefaultOpt{#2}%
  \index{#1}%
  \texttt{\hyphenchar\font=`\-\relax#2}}

%% Keywords (NED, MSG)
\newcommand{\fkeyword}[2][\DefaultOpt]{\def\DefaultOpt{#2}%
  \index{#1}%
  \textbf{\texttt{\hyphenchar\font=`\-\relax#2}}}

%% Configuration options
\newcommand{\fconfig}[2][\DefaultOpt]{\def\DefaultOpt{#2}%
  \index{#1}%
  \textbf{\texttt{\hyphenchar\font=`\-\relax#2}}}

%% File names
\newcommand{\ffilename}[2][\DefaultOpt]{\def\DefaultOpt{#2}%
  \index{#1}%
  \texttt{\hyphenchar\font=`\-\relax#2}}

%% Signals
\newcommand{\fsignal}[2][\DefaultOpt]{\def\DefaultOpt{#2}%
  \index{#1}%
  \texttt{\hyphenchar\font=`\-\relax#2}}

\newcommand{\fgate}[1]{\texttt{\hyphenchar\font=`\-\relax#1}}

%% do not number subsubsections
%\setcounter{secnumdepth}{4}

% limit the depth of TOC
\setcounter{tocdepth}{2}

%%
%% Start of document
%%
\begin{document}

%% set the image type preference
\DeclareGraphicsExtensions{.pdf,.png}

\pagestyle{empty}
\pagenumbering{roman}
\include{title}
\cleardoublepage

%%\setcounter{page}{1}
%\newpage
%%\pagenumbering{roman}

%% \shorttableofcontents{Chapters}{0}
%% \cleardoublepage

\tableofcontents
\cleardoublepage

\pagestyle{fancy}
\pagenumbering{arabic}

\include{ch-introduction}
\cleardoublepage

\chapter{Getting Started}
\label{cha:gettingstarted}

\section{Introduction}
\label{cha:gettingstarted:introduction}

where to put the source files: you can copy and modify the INET framework (fork it)
in the hope that you'll contribute back the changes; or you can develop in
a separate project (create new project in the IDE; mark INET as referenced project)

\section{Contributing to INET}
\label{cha:gettingstarted:contributing-to-inet}

Workflow:

Fork on Github.

Check out the INET project from GitHub, and import it into the OMNeT++ IDE.

Develop.

Submit pull requests.


\section{Setting Up a New INET-Based Project}
\label{cha:gettingstarted:setting-up-inet-based}

Create new project in the IDE.

NED and source files in the same folder; examples under examples/; etc.

Set INET as referenced project.

Set up version control (git, GitHub).

Develop.


\cleardoublepage

\chapter{Networks}
\label{cha:networks}

%
% This chapter provides practical guidance on how to put together various
% networks from the built-in node models and how to configure them,
% WITHOUT LOOKING AT THE INTERNALS OF THOSE NODES.
%

\section{Overview}
\label{sec:networks:overview}

%TODO: wired, wireless, mixed wired/wireless, various topologies + generated, hierarchical, parametric
%TODO: ethernet networks, mpls networks, vpn, tunneling, PPP networks, sensor networks

INET heavily builds upon the modular architecture of OMNeT++. It provides
numerous domain specific and highly parameterizable components which can be
combined in many ways. The primary means of building large custom network
simulations in INET is the composition of existing models with custom models,
starting from small components and gradually forming ever larger ones up until
the composition of the network. Users are not required to have programming
experience to create simulations unless they also want to implement
their own protocols, for example.

Assembling an INET simulation starts with defining a module representing
the network. Networks are compound modules which contain network nodes,
automatic network configurators, and sometimes additionally transmission
medium, physical environment, various visualizer, and other infrastructure
related modules. Networks also contain connections between network nodes
representing cables. Large hierarchical networks may be further organized
into compound modules to directly express the hierarchy.

There are no predefined networks in INET, because it is very easy to create
one, and because of the vast possibilities. However, the OMNeT++ IDE provides
several topology generator wizards for advanced scenarios.

As INET is an OMNeT++-based framework, users mainly use NED to describe the
model topology, and ini files to provide configuration.\footnote{Some
components require additional configuration to be provided as separate
files, e.g. in XML.}

\section{Built-in Network Nodes and Other Top-Level Modules}
\label{sec:networks:built-in-network-nodes-and-other-top-level-modules}

INET provides several pre-assembled network nodes with carefully selected
components. They support customization via parameters and parametric
submodule types, but they are not meant to be universal. Sometimes it may
be necessary to create special network node models for particular
simulation scenarios. In any case, the following list gives a taste of the
built-in network nodes.

\begin{itemize}
  \item \nedtype{StandardHost} contains the most common Internet protocols:
     \protocol{UDP}, \protocol{TCP}, \protocol{IPv4}, \protocol{IPv6},
     \protocol{Ethernet}, \protocol{IEEE 802.11}. It also supports an
     optional mobility model, optional energy models, and any number of
     applications which are entirely configurable from INI files.
  \item \nedtype{EtherSwitch} models an \protocol{Ethernet} switch containing
     a relay unit and one MAC unit per port.
  \item \nedtype{Router} provides the most common routing protocols:
     \protocol{OSPF}, \protocol{BGP}, \protocol{RIP}, \protocol{PIM}.
  \item \nedtype{AccessPoint} models a Wifi access point with multiple
     \protocol{IEEE 802.11} network interfaces and multiple \protocol{Ethernet}
     ports.
  \item \nedtype{WirelessHost} provides a network node with one (default)
     \protocol{IEEE 802.11} network interface in infrastructure mode,
     suitable for using with an \nedtype{AccessPoint}.
  \item \nedtype{AdhocHost} is a \nedtype{WirelessHost} with the network
     interface configured in ad-hoc mode and forwarding enabled.
  \item \nedtype{AodvRouter} is similar to an \nedtype{AdhocHost} with
     an additional \protocol{AODV} protocol.
\end{itemize}

Network nodes communicate at the network level by exchanging OMNeT++ messages
which are the abstract representations of physical signals on the
transmission medium.  Signals are either sent through OMNeT++ connections
in the wired case, or sent directly to the gate of the receiving network node
in the wireless case. Signals encapsulate INET-specific packets that represent
the transmitted digital data. Packets are further divided into chunks that
provide alternative representations for smaller pieces of data (e.g.
protocol headers, application data).

Additionally, there are components that occur on network level, but they
are not models of physical network nodes. They are necessary
to model other aspects. Some of them are:

\begin{itemize}
  \item A \textit{radio medium} module such as \nedtype{Ieee80211RadioMedium},
     \nedtype{ApskScalarRadioMedium} and \nedtype{UnitDiskRadioMedium}
     (there are a few of them) are a required component of wireless networks.
  \item \nedtype{PhysicalEnvironment} models the effect of the physical
     environment (i.e. obstacles) on radio signal propagation. It is an
     optional component.
  \item \textit{Configurators} such as \nedtype{Ipv4NetworkConfigurator},
     \nedtype{L2NetworkConfigurator} and \nedtype{GenericNetworkConfigurator}
     configure various aspects of the network. For example,
     \nedtype{Ipv4\-Network\-Configurator} assigns IP addresses
     to hosts and routers, and sets up static routing. It is used
     when modeling dynamic IP address assignment (e.g. via DHCP) or
     dynamic routing is not of importance. \nedtype{L2NetworkConfigurator}
     allows one to configure 802.1 LANs and provide STP/RSTP-related
     parameters such as link cost, port priority and the ``is-edge'' flag.
  \item \nedtype{ScenarioManager} allows scripted scenarios, such
     as timed failure and recovery of network nodes.
  \item \textit{Group coordinators} are needed for the operation of some
     group mobility mdels. For example, \nedtype{MoBanCoordinator} is
     the coordinator module for the MoBAN mobility model.
  \item \textit{Visualizers} like \nedtype{PacketDropOsgVisualizer} provide
     graphical rendering of some aspect of the simulation either in
     2D (canvas) or 3D (using OSG or osgEarth). The usual choice is
     \nedtype{IntegratedVisualizer} which bundles together an instance
     of each specific visualizer type in a compound module.
\end{itemize}

\section{Typical Networks}
\label{sec:networks:typical-networks}

\subsection{Wired Networks}
\label{sec:networks:wired-networks}

Wired network connections, for example \protocol{Ethernet} cables, are
represented with standard OMNeT++ connections using the
\nedtype{DatarateChannel} NED type. The channel's \nedtype{datarate} and
\nedtype{delay} parameters must be provided for all wired connections.

The following example shows how straightforward it is to create a model for
a simple wired network. This network contains a server connected to a router
using \protocol{PPP}, which in turn is connected to a switch using
\protocol{Ethernet}. The network also contains a parameterizable number of
clients, all connected to the switch forming a star topology. The utilized
network nodes are all predefined modules in INET. To avoid the manual
configuration of IP addresses and routing tables, an automatic network
configurator is also included.

\nedsnippet{WiredNetworkExample}{Wired network example}

In order to run a simulation using the above network, an OMNeT++ INI file must
be created. The INI file selects the network, sets its number of clients
parameter, and configures a simple \protocol{TCP} application for each
client. The server is configured to have a \protocol{TCP} application which
echos back all data received from the clients individually.

\inisnippet{WiredNetworkConfigurationExample}{Wired network configuration example}

When the above simulation is run, each client application connects to the
server using a \protocol{TCP} socket. Then each one of them sends 1MB of
data, which in turn is echoed back by the server, and the simulation
concludes. The default statistics are written to the \texttt{results}
folder of the simulation for later analysis.

\subsection{Wireless Networks}
\label{sec:networks:wireless-networks}

TODO: AccessPoint, WirelessHost infrastructure mode

Wireless network connections are not modeled with OMNeT++ connections due the
dynamically changing nature of connectivity. For wireless networks, an
additional module, one that represents the transmission medium, is required to
maintain connectivity information.

TODO 

\nedsnippet{WirelessNetworkExample}{Wireless network example}

TODO adjust text: 

In the above network, positions in the display strings provide 
positions for the transmission medium during the computation of 
signal propagation and path loss. 

In addition, \ttt{host1} is configured to periodically send
\protocol{UDP} packets to \ttt{host2} over the AP.

\inisnippet{WirelessNetworkConfigurationExample}{Wireless network configuration example}



\subsection{Mobile Ad hoc Networks}
\label{sec:networks:mobile-ad-hoc-networks}

TODO commentary

\nedsnippet{MobileAdhocNetworkExample}{Mobile ad hoc network example}

TODO

\inisnippet{MobileAdhocNetworkConfigurationExample}{Mobile ad hoc network configuration example}

TODO



\section{Frequent Tasks (How To...)}
\label{sec:networks:frequent-tasks}

Quick and somewhat superficial advice to many practical tasks.

\subsection{Automatic Wired Interfaces}
\label{sec:networks:automatic-wired-interfaces}

In many wired network simulations, the number of wired interfaces need not
be manually configured, because it can be automatically inferred from the
actual number of connections between network nodes.

\nedsnippet{AutomaticWiredInterfacesExample}{Automatic wired interfaces
example}

\subsection{Multiple Wireless Interfaces}
\label{sec:networks:multiple-wireless-interfaces}

All built-in wireless network nodes support multiple wireless interfaces,
but only one is enabled by default.

\inisnippet{MultipleWirelessInterfacesExample}{Multiple wireless interfaces
example}

\subsection{Traffic Generation}
\label{sec:networks:traffic-generation}

TODO scripted, synthetic, CBR, VBR, trace-based, .... 
TODO app[]; other ways 

\subsection{Specifying Addresses}
\label{sec:networks:specifying-addresses}

Nearly all application layer modules, but several other components as well,
have parameters that specify network addresses. They typically accept
addresses given with any of the following syntax variations:

\begin{itemize}
  \item literal IPv4 address: \ttt{"186.54.66.2"}
  \item literal IPv6 address: \ttt{"3011:7cd6:750b:5fd6:aba3:c231:e9f9:6a43"}
  \item module name: \ttt{"server"}, \ttt{"subnet.server[3]"}
  \item interface of a host or router: \ttt{"server/eth0"}, \ttt{"subnet.server[3]/eth0"}
  \item IPv4 or IPv6 address of a host or router: \ttt{"server(ipv4)"},
      \ttt{"subnet.server[3](ipv6)"}
  \item IPv4 or IPv6 address of an interface of a host or router:
      \ttt{"server/eth0(ipv4)"}, \ttt{"subnet.server[3]/eth0(ipv6)"}
\end{itemize}

TODO ini example


\subsection{Node Failure and Recovery}
\label{sec:networks:node-failure-and-recovery}

\subsection{Enabling Dual IP Stack}
\label{sec:networks:enabling-dual-ip-stack}

All built-in network nodes support dual Internet protocol stacks, that is
both \protocol{IPv4} and \protocol{IPv6} are available. They are also
supported by transport layer protocols, link layer protocols, and most
applications. Only \protocol{IPv4} is enabled by default, so in order to
use \protocol{IPv6}, it must be enabled first, and an application
supporting \protocol{IPv6} (e.g., \nedtype{PingApp} must be used). The
following example shows how to configure two ping applications in a single
node where one is using an \protocol{IPv4} and the other is using an
\protocol{IPv6} destination address.

\inisnippet{DualStackExample}{Dual stack example}

\subsection{Enabling Packet Forwarding}
\label{sec:networks:enabling-packet-forwarding}

In general, network nodes don't forward packets by default, only
\nedtype{Router} and the like do. Nevertheless, it's possible to enable
packet forwarding as simply as flipping a switch.

\inisnippet{ForwardingExample}{Forwarding example}

\subsection{Adding Routing Protocols}
\label{sec:networks:adding-routing-protocols}

TODO internet routing and ad hoc routing

\subsection{Node Mobility}
\label{sec:networks:node-mobility}

TODO

\subsection{Topology Generation}
\label{sec:networks:topology-generation}

TODO: wizards, generated topologies

\subsection{Hierarchical Networks}
\label{sec:networks:hierarchical-networks}

TODO: nested compound modules

%%% Local Variables:
%%% mode: latex
%%% TeX-master: "usman"
%%% End:



\cleardoublepage

\chapter{Network Nodes}
\label{cha:network-nodes}

\section{Overview}
\label{sec:nodes:overview}

Hosts, routers, switches, access points, mobile phones, and other network
nodes are represented in INET with compound modules. The previous chapter
has introduced a few node types like \nedtype{StandardHost}, \nedtype{Router},
and showed how to put together networks from them. In this chapter,
we look at the internals of such node models, in order to provide a deeper
understanding of their customization possibilities and to give some guidance
on how custom nodes models can be assembled.

\section{Ingredients}
\label{sec:nodes:ingredients}

Node models are assembled from other modules which represent applications,
communication protocols, network interfaces, routing tables, mobility models,
energy models, and other functionality. These modules fall into the following
broad categories:

\begin{itemize}
  \item \textit{Applications} often model the user behavior as well as the
     application program (e.g., browser), and the application layer protocol
     (e.g., \protocol{HTTP}). Applications typically use transport layer
     protocols (e.g., \protocol{TCP} and/or \protocol{UDP}), but they may
     also directly use lower layer protocols (e.g., \protocol{IP} or
     \protocol{Ethernet}) via sockets.
  \item \textit{Routing protocols} are provided as separate modules:
     \protocol{OSPF}, \protocol{BGP}, or \protocol{AODV} for MANET routing.
     These modules use \protocol{TCP}, \protocol{UDP}, and \protocol{IPv4},
     and manipulate routes in the \nedtype{Ipv4\-RoutingTable} module.
  \item \textit{Transport layer protocols} are connected to applications and
     network layer protocols. They are most often represented by simple
     modules, currently \protocol{TCP}, \protocol{UDP}, and \protocol{SCTP}
     are supported. \protocol{TCP} has several implementations: \nedtype{Tcp}
     is the OMNeT++ native implementation; \nedtype{TcpLwip} module wraps the
     lwIP \protocol{TCP} stack; and \nedtype{TcpNsc} module wraps the
     Network Simulation Cradle library.
  \item \textit{Network layer protocols} are connected to transport layer
     protocols and network interfaces. They are usually modeled as compound
     modules: \nedtype{Ipv4NetworkLayer} for \protocol{IPv4}, and
     \nedtype{Ipv6NetworkLayer} for \protocol{IPv6}. The \nedtype{Ipv4NetworkLayer}
     module contains several protocol modules: \nedtype{Ipv4}, \nedtype{Arp},
     and \nedtype{Icmpv4}.
  \item \textit{Network interfaces} are represented by compound modules
     which are connected to the network layer protocols and other network
     interfaces in the wired case. They are often modeled as compound modules
     containing separate modules for queues, classifiers, MAC, and PHY protocols.
  \item \textit{Link layer protocols} are usually simple modules sitting
     in network interface modules. Some protocols, for example
     \protocol{IEEE 802.11 MAC}, are modeled as a compound module themselves
     due to the complexity of the protocol.
  \item \textit{Physical layer protocols} are compound modules also being part
     of network interface modules.
  \item \textit{Interface table} maintains the set of network interfaces
     (e.g. \texttt{eth0}, \texttt{wlan0}) in the network node. Interfaces
     are registered dynamically during initialization of network interfaces.
  \item \textit{Routing tables} maintain the list of routes for the corresponding
     network protocol (e.g., \nedtype{Ipv4RoutingTable} for \nedtype{Ipv4}).
     Routes are added by automatic network configurators or routing protocols.
     Network protocols use the routing tables to find out the best matching
     route for datagrams.
  \item \textit{Mobility modules} are responsible for moving around the network
     node in the simulated playground. The mobility model is mandatory for
     wireless simulations even if the network node is stationary. The mobility
     module stores the location of the network node which is needed to compute
     wireless propagation and path loss. Different mobility models are provided
     as different modules. Network nodes define their mobility submodule with
     a parametric type, so the mobility model can be changed in the configuration.
  \item \textit{Energy modules} model energy storage mechanisms, energy
     consumption of devices and software processes, energy generation of devices,
     and energy management processes which shutdown and startup network nodes.
  \item \textit{Status} (\nedtype{NodeStatus}) keeps track of the status of the
     network node (up, down, etc.)
  \item \textit{Other modules} with particular functionality such as
     \nedtype{PcapRecorder} are also available.
\end{itemize}

\section{Node Architecture}
\label{sec:nodes:node-architecture}

Within network nodes, OMNeT++ connections are used to represent
communication opportunities between protocols. Packets and
messages sent on these connections represent software or hardware activity.

Although protocols may also be connected to each other directly,
in most cases they are connected via \textit{dispatcher modules}.
Dispatchers (\nedtype{MessageDispatcher}) are small, low-overhead modules
that allow protocol components to be connected in one-to-many and many-to-many
fashion, and ensure that messages and packets sent from one component end up
being delivered to the correct component. Dispatchers need no manual
configuration, as they use discovery and peek into packets.

In there pre-assembled node models, dispatchers allow arbitrary
protocol components to talk directly to each other, i.e. not only
to ones in neighboring layers.

\section{Customizing Nodes}
\label{sec:nodes:customizing-nodes}

The built-in network nodes are written to be as versatile and customizable
as possible. This is achieved in several ways:

\subsection*{Submodule and Gate Vectors}

One way is the use of gate vectors and submodule vectors. The sizes
of vectors may come from parameters or derived by the number of
external connections to the network node. For example, a host may
have an arbitrary number of wireless interfaces, and it will automatically
have as many \protocol{Ethernet} interfaces as the number of \protocol{Ethernet}
devices connected to it.

For example, wireless interfaces for hosts are defined like this:

\begin{ned}
wlan[numWlanInterfaces]: <snip> // wlan interfaces in StandardHost etc al.
\end{ned}

Where \ttt{numWlanInterfaces} is a module parameter that defaults to
either 0 or 1 (this is different for e.g. \nedtype{StandardHost} and
\nedtype{WirelessHost}.) To configure a host to have two interfaces,
add the following line to the ini file:

\begin{inifile}
**.hostA.numWlanInterfaces = 2
\end{inifile}

\subsection*{Conditional Submodules}

Submodules that are not vectors are often conditional. For example,
the \protocol{TCP} protocol module in hosts is conditional on
the \ttt{hasTcp} parameter. Thus, to disable \protocol{TCP} support
in a host (it is enabled by default), use the following ini file line:

\begin{inifile}
**.hostA.hasTcp = false
\end{inifile}

\subsection*{Parametric Types}

Another often used way of customization is parametric types, that is, the
type of a submodule (or a channel) may be specified as a string parameter.
Almost all submodules in the built-in node types have parametric types.
For example, the \protocol{TCP} protocol module is defined like this:

\begin{ned}
tcp: <tcpType> like ITcp if hasTcp;
\end{ned}

The \ttt{tcpType} parameter defaults to the default implementation, \nedtype{Tcp}.
To use another implementation instead, add the following line to the ini file:

\begin{inifile}
**.host*.tcpType = "TcpLwip"  # use lwIP's TCP implementation
\end{inifile}

Submodule vectors with parametric types are defined without the use of a
module parameter to allow elements have different types. An example
is how applications are defined in hosts:

\begin{ned}
app[numApps]: <> like IApp;  // applications in StandardHost et al.
\end{ned}

And applications can be added in the following way:

\begin{inifile}
**.hostA.numApps = 2
**.hostA.apps[0].typename = "UdpBasicApp"
**.hostA.apps[1].typename = "PingApp"
\end{inifile}

\subsection*{Inheritance}

Inheritance can be use to derive new, specialized node types from existing ones.
A derived NED type may add new parameters, gates, submodules or connections,
and may set inherited unassigned parameters to specific values.

For example, \nedtype{WirelessHost} is derived from \nedtype{StandardHost}
in the following way:

\begin{ned}
module WirelessHost extends StandardHost
{
    @display("i=device/wifilaptop");
    numWlanInterfaces = default(1);
}
\end{ned}

\section{Custom Network Nodes}
\label{sec:nodes:custom-network-nodes}

Despite the many pre-assembled network nodes and the several available
customization options, sometimes it is just easier to build a network node
from scratch. The following example shows how easy it is to build a simple
network node.

This network node already contains a configurable application and several
standard protocols. It also demonstrates how to use the packet dispatching
mechanism which is required to connect multiple protocols in a many-to-many
relationship.

\nedsnippet{NetworkNodeExample}{Network node example}



%%% Local Variables:
%%% mode: latex
%%% TeX-master: "usman"
%%% End:


\cleardoublepage

\chapter{Network Interfaces}
\label{cha:network-interfaces}

\section{Overview}
\label{sec:interfaces:overview}

%TODO: MAC address, op mode, duplex mode, data rate, transmission power, queue limits, FCS mode

In INET simulations, network interface modules are the primary means of
communication between network nodes. They represent the required
combination of software and hardware elements from an operating system
point-of-view.

Network interfaces are implemented with OMNeT++ compound modules that
conform to the \nedtype{INetworkInterface} module interface.
Network interfaces can be further categorized as wired and wireless;
they conform to the \nedtype{IWiredInterface} and \nedtype{IWirelessInterface}
NED types, respectively, which are subtypes of \nedtype{INetworkInterface}.

\section{Built-in Network Interfaces}
\label{sec:interfaces:built-in-network-interfaces}

INET provides pre-assembled network interfaces for several standard
protocols, protocol tunneling, hardware emulation, etc. The following list
gives the most commonly used network interfaces.

\begin{itemize}
    \item \nedtype{EthernetInterface} represents an \protocol{Ethernet} interface
    \item \nedtype{PppInterface} is for wired links using \protocol{PPP}
    \item \nedtype{Ieee80211Interface} represents a Wifi (\protocol{IEEE 802.11}) interface
    \item \nedtype{Ieee802154Interface} represents a \protocol{IEEE 802.15.4} interface
    \item \nedtype{BMacInterface}, \nedtype{LMacInterface}, \nedtype{XMacInterface} provide
      low-power wireless sensor MAC protocols along with a simple hypothetical PHY protocol
    \item \nedtype{TunInterface} is a tunneling interface that can be directly used by applications
    \item \nedtype{LoopbackInterface} provides local loopback within the network node
    \item \nedtype{ExtInterface} represents a real-world interface, suitable for hardware-in-the-loop simulations
\end{itemize}

\section{Anatomy of Network Interfaces}
\label{sec:interfaces:anatomy-of-network-interfaces}

Network interfaces in the INET Framework are OMNeT++ compound modules that
contain many more components than just the corresponding layer 2 protocol
implementation. Most of these components are optional, i.e. absent by default,
and can be added via configuration.

Typical ingredients are:

\begin{itemize}
    \item \emph{Layer 2 protocol implementation}. For some interfaces such as
      \nedtype{PppInterface} this is a single module; for others like Ethernet
      and Wifi it consists of separate modules for MAC, LLC, and possibly
      other subcomponents.
    \item \emph{PHY model}. Some interfaces also contain separate
      module(s) that implement the physical layer. For example,
      \nedtype{Ieee80211Interface} contains a radio module.
    \item \emph{Output queue}. This module is optional and absent by default,
      because most MAC protocol implementations already contain an internal queue
      which is more efficient to work with. The possibility to plug in an
      external queue module allows one to experiment with different queueing policies
      and implement QoS, RED, etc.
    \item \emph{Traffic conditioners} allow traffic shaping and policing elements
      to be added to the interface, for example to implement a Diffserv router.
    \item \emph{Hooks} allow extra modules to be inserted in the incoming
      and outgoing paths of packets.
\end{itemize}


\subsection{Internal vs External Output Queue}
\label{sec:interfaces:internal-vs-external-output-queue}

Network interfaces usually have the external queue module defined with a
parametric type like this:

\begin{ned}
queue: <queueType> like IOutputQueue if queueType != "";
\end{ned}

When \fpar{queueType} is empty (this is the default), the external queue
module is absent, and the MAC (or equivalent L2) protocol will use its
internal queue object. Conceptually, the internal queue is of inifinite size,
but for better diagnostics one can often specify a hard limit for the queue
length in a module parameter -- if this is exceeded, the simulation
stops with an error.

When \fpar{queueType} is not empty, it must name a NED type that
implements the \nedtype{IOutputQueue} interface. The external
queue module model allows modeling a finite buffer, or implement
various queueing policies for QoS and/or RED.

The most frequently used module type for external queue is
\nedtype{DropTailQueue}, a finite-size FIFO that drops overflowing
packets). Other queue types that implement queueing policies can be
created by assembling compound modules from DiffServ components
(see chapter \ref{cha:diffserv}). An example of such compound
modules is \nedtype{DiffservQueue}.

An example ini file fragment that installs drop-tail queues of size 10
on PPP interfaces:

\begin{inifile}
**.ppp[*].queueType = "DropTailQueue"
**.ppp[*].queue.frameCapacity = 10
\end{inifile}

\subsection{Traffic Conditioners}
\label{sec:interfaces:traffic-conditioners}

Many network interfaces contain optional traffic conditioner submodules
defined with parametric types, like this:

\begin{ned}
ingressTC: <ingressTCType> like ITrafficConditioner if ingressTCType != "";
egressTC: <egressTCType> like ITrafficConditioner if egressTCType != "";
\end{ned}

Traffic conditioners allow one to implement the policing and shaping actions
of a Diffserv router. They are added to the input or output packets paths
in the network interface. (On the output path they are added before the queue
module.)

Traffic conditioners must implement the \nedtype{ITrafficConditioner} module
interface. Traffic conditioners can be assembled from DiffServ components
(see chapter \ref{cha:diffserv}). There is no preassembled traffic conditioner
in INET, but you can find some in the example simulations.

An example configuration with fictituous types:

\begin{inifile}
**.ppp[*].ingressTCType = "CustomIngressTC"
**.ppp[*].egressTCType = "CustomEgressTC"
\end{inifile}


\subsection{Hooks}
\label{sec:interfaces:hooks}

Several network interfaces allow extra modules to be inserted in the incoming
and outgoing paths of packets at the top of the netwok interface.
Hooks are added as a submodule vector with parametric type, like this:

\begin{ned}
outputHook[numOutputHooks]: <default("Nop")> like IHook if numOutputHooks>0;
inputHook[numInputHooks]: <default("Nop")> like IHook if numInputHooks>0;
\end{ned}

This allows any number of hook modules to be added. The hook modules
are chained in their numeric order.

Modules inserted as hooks may act as probes (for measuring or recording
traffic) or as means of modifying or perturbing the packet flow for
experimentation. Module types implementing the \nedtype{IHook} NED interface
include \nedtype{ThruputMeter}, \nedtype{Delayer}, \nedtype{OrdinalBasedDropper},
and \nedtype{OrdinalBasedDuplicator}.

The following ini file fragment inserts two hook modules into the output
paths of PPP interfaces, a delayer and a throughput meter:

\begin{inifile}
**.ppp[*].numOutputHooks = 2
**.ppp[*].outputHook[0].typename = "Delayer"
**.ppp[*].outputHook[1].typename = "ThruputMeter"
**.ppp[*].outputHook[0].delay = 3ms
\end{inifile}



\section{The Interface Table}
\label{sec:interfaces:the-interface-table}

Network nodes normally contain an \nedtype{InterfaceTable} module.
The interface table is a sort of registry of all the network interfaces
in the host. It does not send or receive messages, other modules access it
via C++ function calls. Contents of the interface table can also
be inspected e.g. in Qtenv.

Network interfaces register themselves in the interface table at the
beginning of the simulation. Registration is usually the task of the
MAC (or equivalent) module.


\section{Wired Network Interfaces}
\label{sec:interfaces:wired-network-interfaces}

Wired interfaces have a pair of special purpose OMNeT++ gates which represent
the capability of having an external physical connection to another network
node (e.g. Ethernet port). In order to make wired communication work,
these gates must be connected with special connections which represent the
physical cable between the physical ports. The connections must use special
OMNeT++ channels (e.g. \nedtype{DatarateChannel}) which determine datarate
and delay parameters.

Wired network interfaces are compound modules that implement the
\nedtype{IWiredInterface} interface. INET has the following
wired network interfaces.

\subsection{PPP}
\label{sec:interfaces:ppp}

Network interfaces for point-to-point links (\nedtype{PppInterface}) are
described in chapter \ref{cha:ppp}. They are typically used in routers.

\subsection{Ethernet}
\label{sec:interfaces:ethernet}

Ethernet interfaces (\nedtype{EthernetInterface}), alongside with models
of Ethernet devices such as switches and hubs, are described in chapter
\ref{cha:ethernet}.

\section{Wireless Network Interfaces}
\label{sec:interfaces:wireless-network-interfaces}

Wireless interfaces use direct sending\footnote{OMNeT++ \ttt{sendDirect()} calls}
for communication instead of links, so their compound modules do not have
output gates at the physical layer, only an input gate dedicated to receiving.
Another difference from the wired case is that wireless interfaces
require (and collaborate with) a \textit{transmission medium} module
at the network level. The medium module represents the shared transmission
medium (electromagnetic field or acoustic medium), is responsible for
modeling physical effects like signal attenuation, and maintains
connectivity information. Also, while wired interfaces can do without
explicit modeling of the physical layer, a PHY module is an indispensable
part of a wireless interface.

Wireless network interfaces are compound modules that implement the
\nedtype{IWirelessInterface} interface. In the following sections we
give an overview of the wireless interfaces available in INET.

\subsection{Generic Wireless Interface}
\label{sec:interfaces:generic-wireless-interface}

The \nedtype{WirelessInterface} compound module is a generic implementation
of \nedtype{IWirelessInterface}. In this network interface, the types of the
MAC protocol and the PHY layer (the radio) are parameters:

\begin{ned}
mac: <macType> like IMacProtocol;
radio: <radioType> like IRadio if radioType != "";
\end{ned}

There are specialized versions of \nedtype{WirelessInterface} where
the MAC and the radio modules are fixed to a particular value.
One example is \nedtype{BMacInterface}, which contains a \nedtype{BMac}
and an \nedtype{ApskRadio}.

\subsection{IEEE 802.11}
\label{sec:interfaces:ieee-80211}

IEEE 802.11 or Wifi network interfaces (\nedtype{Ieee80211Interface}),
alongside with models of devices acting as access points (AP),
are covered in chapter \ref{cha:80211}.

\subsection{IEEE 802.15.4}
\label{sec:interfaces:ieee-802154}

\nedtype{Ieee802154Interface} is covered in a separate chapter, see \ref{cha:802154}.

\subsection{Wireless Sensor Networks}
\label{sec:interfaces:wireless-sensor-networks}

MAC protocols for wireless sensor networks (WSNs) and the corresponding
network interfaces are covered in chapter \ref{cha:sensor-macs}.

\subsection{CSMA/CA}
\label{sec:interfaces:csma/ca}

\nedtype{CsmaCaMac} implements an imaginary CSMA/CA-based MAC protocol with
optional acknowledgements and a retry mechanism. With the appropriate settings,
it can approximate basic 802.11b ad-hoc mode operation.

\nedtype{CsmaCaMac} provides a lot of room for experimentation:
acknowledgements can be turned on/off, and operation parameters like
inter-frame gap sizes, backoff behaviour (slot time, minimum and maximum
number of slots), maximum retry count, header and ACK frame sizes, bit rate,
etc. can be configured via NED parameters.

\nedtype{CsmaCaInterface} interface is a \nedtype{WirelessInterface} with
the MAC type set to \nedtype{CsmaCaMac}.

\subsection{Acking MAC}
\label{sec:interfaces:acking-mac}

Not every simulation requires a detailed simulation of the lower layers.
\nedtype{AckingWirelessInterface} is a highly abstracted wireless interface
that offers simplicity for scenarios where Layer 1 and 2 effects can be
completely ignored, for example testing the basic functionality of a
wireless ad-hoc routing protocol.

\nedtype{AckingWirelessInterface} is a \nedtype{WirelessInterface}
parameterized to contain a unit disk radio (\nedtype{UnitDiskRadio})
and a trivial MAC protocol (\nedtype{AckingMac}).

The most important parameter \nedtype{UnitDiskRadio} accepts is the
transmission range. When a radio transmits a frame, all other radios
within transmission range are able to receive the frame correctly,
and radios that are out of range will not be affected at all.
Interference modeling (collisions) is optional, and it is recommended
to turn it off with \nedtype{AckingMac}.

\nedtype{AckingMac} implements a trivial MAC protocol that has packet
encapsulation and decapsulation, but no real medium access procedure.
Frames are simply transmitted on the wireless channel as soon as the
transmitter becomes idle. There is no carrier sense, collision avoidance,
or collison detection. \nedtype{AckingMac} also provides an optional
out-of-band acknowledgement mechanism (using C++ function calls,
not actual wirelessly sent frames), which is turned on by default.
There is no retransmission: if the acknowledgement does not arrive
after the first transmission, the MAC gives up and counts the packet
as failed transmission.

\subsection{Shortcut}
\label{sec:interfaces:shortcut}

\nedtype{ShortcutMac} implements error-free ``teleportation'' of packets
to the peer MAC entity, with some delay computed from a transmission
duration and a propagation delay. The physical layer is completely bypassed.
The corresponding network interface type, \nedtype{ShortcutInterface},
does not even have a radio model.

\nedtype{ShortcutInterface} is useful for modeling wireless networks
where full connectivity is assumed, and Layer 1 and Layer 2 effects
can be completely ignored.

\section{Special-Purpose Network Interfaces}
\label{sec:interfaces:special-purpose-network-interfaces}


\subsection{Tunnelling}
\label{sec:interfaces:tunnelling}

\nedtype{TunInterface} is a virtual network interface that can be used
for creating tunnels, but it is more powerful than that.
It lets an application-layer module capture packets sent to
the TUN interface and do whatever it pleases with it (including
sending it to a peer entity in an UDP or plain IPv4 packet.)

To set up a tunnel, add an instance of \nedtype{TunnelApp} to
the node, and specify the protocol (IPv4 or UDP) and the remote
endpoint of the tunnel (address and port) in parameters.

TODO example: see examples/inet/tunnel

\subsection{Local Loopback}
\label{sec:interfaces:local-loopback}

\nedtype{LoopbackInterface} provides local loopback within the network node.

\subsection{External Interface}
\label{sec:interfaces:external-interface}

\nedtype{ExtInterface} represents a real-world interface, suitable for
hardware-in-the-loop simulations. External interfaces are explained in
chapter \ref{cha:emulation}.

\section{Custom Network Interfaces}
\label{sec:interfaces:custom-network-interfaces}

It's also possible to build custom network interfaces, the following
example shows how to build a custom wireless interface.

\nedsnippet{WirelessInterfaceExample}{Wireless interface example}

The above network interface contains very simple hypothetical MAC and PHY
protocols. The MAC protocol only provides acknowledgment without other
services (e.g., carrier sense, collision avoidance, collision detection),
the PHY protocol uses one of the predefined APSK modulations for the whole
signal (preamble, header, and data) without other services (e.g.,
scrambling, interleaving, forward error correction).


%%% Local Variables:
%%% mode: latex
%%% TeX-master: "usman"
%%% End:



\cleardoublepage

\chapter{Applications}
\label{cha:apps}

TODO



\cleardoublepage

\chapter{Transport Protocols}
\label{cha:transport-protocols}

\section{Overview}
\label{sec:transport:overview}

In the OSI reference model, the protocols of the transport layer provide
host-to-host communication services for applications. They provide services such
as connection-oriented communication, reliability, flow control, and
multiplexing.

INET currently provides support for the TCP, UDP, SCTP and RTP transport layer
protocols. INET nodes like \nedtype{StandardHost} contain optional and
replaceable instances of these protocols, like this:

\begin{ned}
tcp: <tcpType> like ITcp if hasTcp;
udp: <udpType> like IUdp if hasUdp;
sctp: <sctpType> like ISctp if hasSctp;
\end{ned}

As RTP is more specialized that the other ones (multimedia streaming), INET
provides a separate node type, \nedtype{RtpHost}, for modeling RTP traffic.

\section{TCP}
\label{sec:transport:tcp}

\subsection{Overview}
\label{sec:transport:tcp-overview}

TCP protocol is the most widely used protocol of the Internet. It provides
reliable, ordered delivery of stream of bytes from one application on one
computer to another application on another computer. The baseline TCP protocol
is described in RFC793, but other tens of RFCs contains modifications and
extensions to the TCP. As a result, TCP is a complex protocol and sometimes it
is hard to see how the different requirements interact with each other.

INET contains three implementations of the TCP protocol:

\begin{itemize}
  \item \nedtype{Tcp} is the primary implementation, designed for readability,
    extensibility, and experimentation.
  \item \nedtype{TcpLwip} is a wrapper around the lwIP (Lightweight IP) library,
    a widely used open source TCP/IP stack designed for embedded systems.
  \item \nedtype{TcpNsc} wraps Network Simulation Cradle (NSC), a library
    that allows real world TCP/IP network stacks to be used inside a
    network simulator.
\end{itemize}

All three module types implement the \nedtype{ITcp} interface and communicate
with other layers through the same interface, so they can be interchanged and
also mixed in the same network.


\subsection{Tcp}
\label{sec:transport:inet-tcp}

The \nedtype{Tcp} simple module is the main implementation of the TCP protocol
in the INET framework.

\nedtype{Tcp} implements the following:

\begin{compactitem}
  \item TCP state machine
  \item initial sequence number selection according to the system clock.
  \item window-based flow control
  \item Window Scale option
  \item Persistence timer
  \item Keepalive timer
  \item Transmission policies
  \item RTT measurement for retransmission timeout (RTO) computation
  \item Delayed ACK algorithm
  \item Nagle's algorithm
  \item Silly window avoidance
  \item Timestamp option
  \item Congestion control schemes: Tahoe, Reno, New Reno, Westwood, Vegas, etc.
  \item Slow Start and Congestion Avoidance
  \item Fast Retransmit and Fast Recovery
  \item Loss Recovery Using Limited Transmit
  \item Selective Acknowledgments (SACK)
  \item SACK based loss recovery
\end{compactitem}

Several protocol features can be turned on/off with parameters like
\fpar{delayedAcksEnabled}, \fpar{nagleEnabled}, \fpar{limitedTransmitEnabled},
\fpar{increasedIWEnabled}, \fpar{sackSupport}, \fpar{windowScalingSupport}, or
\fpar{timestampSupport}.

The congestion control algorithm can be selected with the \fpar{tcpAlgorithmClass}
parameter. For example, the following ini file fragment selects TCP Vegas:

\begin{inifile}
**.tcp.tcpAlgorithmClass = "TcpVegas"
\end{inifile}

Values like \ttt{"TcpVegas"} name C++ classes. Indeed, \nedtype{Tcp} can be
extended with new congestion control schemes by implementing and registering
them in C++.



\subsection{TcpLwip}
\label{sec:transport:tcplwip}

\textit{lwIP} is a light-weight implementation of the TCP/IP protocol suite
that was originally written by Adam Dunkels of the Swedish Institute of
Computer Science. The current development homepage is
\url{http://savannah.nongnu.org/projects/lwip/}.

The implementation targets embedded devices: it has very limited resource usage
(it works ``with tens of kilobytes of RAM and around 40 kilobytes of ROM''), and
does not require an underlying OS.

The \nedtype{TcpLwip} simple module is based on the 1.3.2 version of
the lwIP sources.

Features:

\begin{compactitem}
\item delayed ACK
\item Nagle's algorithm
\item round trip time estimation
\item adaptive retransmission timeout
\item SWS avoidance
\item slow start threshold
\item fast retransmit
\item fast recovery
\item persist timer
\item keep-alive timer
\end{compactitem}

\subsubsection*{Limitations}

\begin{itemize}
  \item only MSS and TS TCP options are supported. The TS option is turned off
        by default, but can be enabled by defining LWIP\_TCP\_TIMESTAMPS to 1
        in \ffilename{lwipopts.h}.
  \item \fvar{fork} must be \fkeyword{true} in the passive open command
  \item The status request command (TCP\_C\_STATUS) only reports the
          local and remote addresses/ports of the connection and
          the MSS, SND.NXT, SND.WND, SND.WL1, SND.WL2, RCV.NXT, RCV.WND variables.
\end{itemize}

\subsection{TcpNsc}
\label{sec:transport:tcpnsc}

Network Simulation Cradle (NSC) is a tool that allow real-world TCP/IP network stacks
to be used in simulated networks. The NSC project is created by Sam Jansen
and available on \url{http://research.wand.net.nz/software/nsc.php}. NSC currently
contains Linux, FreeBSD, OpenBSD and lwIP network stacks, although on 64-bit
systems only Linux implementations can be built.

To use the \nedtype{TcpNsc} module you should download the
\ffilename{nsc-0.5.2.tar.bz2} package and follow the instructions
in the \ffilename{<inet\_root>/3rdparty/README} file to build it.

\begin{warning}
Before generating the INET module, check that the \emph{opp\_makemake} call
in the make file (\ffilename{<inet\_root>/Makefile}) includes the
\emph{-DWITH\_TCP\_NSC} argument. Without this option the \nedtype{TcpNsc}
module is not built. If you build the INET library from the IDE, it is enough
to enable the \emph{TCP (NSC)} project feature.
\end{warning}

\subsubsection*{Parameters}

The module has the following parameters:

\begin{itemize}
  \item \fpar{stackName}: the name of the TCP implementation to be used.
    Possible values are: \ttt{liblinux2.6.10.so}, \ttt{liblinux2.6.18.so},
    \ttt{liblinux2.6.26.so}, \ttt{libopenbsd3.5.so}, \ttt{libfreebsd5.3.so} and
    \ttt{liblwip.so}. (On the 64 bit systems, the \ttt{liblinux2.6.26.so} and
    \ttt{liblinux2.6.16.so} are available only).
  \item \fpar{stackBufferSize}: the size of the receive and send buffer of
    one connection for selected TCP implementation.
    The NSC sets the \fvar{wmem\_max}, \fvar{rmem\_max}, \fvar{tcp\_rmem}, \fvar{tcp\_wmem}
    parameters to this value on linux TCP implementations. For details, you can see
    the NSC documentation.
\end{itemize}

\subsubsection*{Limitations}

\begin{itemize}
  \item Because the kernel code is not reentrant, NSC creates a record containing
    the global variables of the stack implementation. By default there is room
    for 50 instance in this table, so you can not create more then 50 instance
    of \nedtype{TcpNsc}. You can increase the \fvar{NUM\_STACKS} constant
    in \ffilename{num\_stacks.h} and recompile NSC to overcome this limitation.
  \item The \nedtype{TcpNsc} module does not supprt TCP\_TRANSFER\_OBJECT
    data transfer mode.
  \item The MTU of the network stack fixed to 1500, therefore MSS is 1460.
  \item TCP\_C\_STATUS command reports only local/remote addresses/ports and
    current window of the connection.
\end{itemize}

% local address: 1.0.0.253, gateway address: 1.0.0.254, remote addresses: 2.0.0.1, 2.0.0.2, ...

% FIXME connections are identified by connId, not by (appGateIndex,connId) as in TCP module.
% FIXME TCP_NSC_Connection::getSocket() and TCP_NSC_Connection::do_checkedclose() are declared but not implemented


\section{UDP}
\label{sec:transport:udp}

The UDP protocol is a very simple datagram transport protocol, which
basically makes the services of the network layer available to the applications.
It performs packet multiplexing and demultiplexing to ports and some basic
error detection only.

The \nedtype{Udp} simple module implements the UDP protocol.
There is a module interface (\nedtype{IUdp}) that defines the gates of the
\nedtype{Udp} component. In the \nedtype{StandardHost} node, the UDP component
can be any module implementing that interface.

\section{SCTP}
\label{sec:transport:sctp}

The \nedtype{Sctp} module implements the Stream Control Transmission Protocol
(SCTP). Like TCP, SCTP provides reliable ordered data delivery over an ureliable
network. The most prominent feature of SCTP is the capability of transmitting
multiple streams of data at the same time between two end points that have
established a connection.

TODO some details (features, etc.)

\section{RTP}
\label{sec:transport:rtp}

The Real-time Transport Protocol (RTP) is a transport layer protocol for
delivering audio and video over IP networks. RTP is used extensively in
communication and entertainment systems that involve streaming media, such as
telephony, video teleconference applications including WebRTC, television
services and web-based push-to-talk features.

The RTP Control Protocol (RTCP) is a sister protocol of the Real-time Transport
Protocol (RTP). RTCP provides out-of-band statistics and control information for
an RTP session.

INET provides the following modules:

\begin{itemize}
  \item \nedtype{Rtp} implements the RTP protocol
  \item \nedtype{Rtcp} implements the RTCP protocol
\end{itemize}

TODO some details (how to use, etc.)


\section{//////////// TCP PROTOCOL DESCRIPTION /////////////////}
\label{tcp-protocol-desc}

\subsection{Topics}
\label{tcp-protocol-desc-topics}

The TCP modules of the INET framework implements the following RFCs:

\begin{tabular}{ll}
RFC 793 & Transmission Control Protocol \\
RFC 896 & Congestion Control in IP/TCP Internetworks \\
RFC 1122 & Requirements for Internet Hosts -- Communication Layers \\
RFC 1323 & TCP Extensions for High Performance \\
RFC 2018 & TCP Selective Acknowledgment Options \\
RFC 2581 & TCP Congestion Control \\
RFC 2883 & An Extension to the Selective Acknowledgement (SACK) Option for TCP \\
RFC 3042 & Enhancing TCP's Loss Recovery Using Limited Transmit \\
RFC 3390 & Increasing TCP's Initial Window \\
RFC 3517 & A Conservative Selective Acknowledgment (SACK)-based Loss Recovery \newline
                 Algorithm for TCP \\
RFC 3782 & The NewReno Modification to TCP's Fast Recovery Algorithm \\
\end{tabular}

In this section we describe the features of the TCP protocol specified by these RFCs,
the following sections deal with the implementation of the TCP in the INET framework.

\subsubsection{TCP segments}
\label{sec:transport:tcp-segments}

The TCP module transmits a stream of the data over the unreliable, datagram service
that the IP layer provides. When the application writes a chunk of data into the socket,
the TCP module breaks it down to packets and hands it over the IP. On the receiver side,
it collects the recieved packets, order them, and acknowledges the reception. The packets
that are not acknowledged in time are retransmitted by the sender.

The TCP procotol can address each byte of the data stream by \emph{sequence numbers}.
The sequence number is a 32-bit unsigned integer, if the end of its range is reached,
it is wrapped around.


\subsubsection{TCP connections}
\label{sec:transport:tcp-connections}

When two applications are communicating via TCP, one of the applications is the client,
the other is the server. The server usually starts a socket with a well known local port
and waits until a request comes from clients. The client applications are issue connection
requests to the port and address of the service they want to use.

After the connection is established both the client and the server can send and receive data.
When no more data is to be sent, the application closes the socket. The application can still
receive data from the other direction. The connection is closed when both communication partner
closed its socket.

...

When opening the connection an initial sequence number is choosen and communicated to the
other TCP in the SYN segment. This sequence number can not be a constant value (e.g. 0),
because then data segments from a previous incarnation of the connection (i.e. a connection
with same addresses and ports) could be erronously accepted in this connection. Therefore
most TCP implementation choose the initial sequence number according to the system clock.

\begin{figure}[htbp]
  \begin{center}
    \includesvg[width=\textwidth]{figures/tcpstate}
    \caption{TCP state diagram}
    \label{fig:tcp_states}
  \end{center}
\end{figure}

\subsubsection{Flow control}
\label{sec:transport:tcp-flow-control}

The TCP module of the receiver buffers the data of incoming segments.
This buffer has a limited capacity, so it is desirable to notify the sender
about how much data the client can accept. The sender stops the transmission
if this space exhausted.

In TCP every ACK segment holds a Window field; this is the available space
in the receiver buffer. When the sender reads the Window, it can send at most
Window unacknowledged bytes.

\subsubsection*{Window Scale option}

% RFC1323
The TCP segment contains a 16-bit field for the Window, thus allowing at most
65535 byte windows. If the network bandwidth and latency is large, it is surely
too small. The sender should be able to send bandwitdh*latency bytes without
receiving ACKs.

For this purpose the Window Scale (WS) option had been introduced in RFC1323.
This option specifies a scale factor used to interpret the value of the Window field.
The format is the option is:

\begin{pdfonly}
\begin{center}
\begin{bytefield}{24}
\bitbox{8}{Kind=3} &
\bitbox{8}{Length=3} &
\bitbox{8}{shift.cnt}
\end{bytefield}
\end{center}
\end{pdfonly}

\begin{htmlonly}
[WARNING: Cannot translate LaTeX bitbox to HTML!]
\end{htmlonly}


If the TCP want to enable window sizes greater than 65535, it should send
a WS option in the SYN segment or SYN/ACK segment (if received a SYN with WS
option). Both sides must send the option in the SYN segment to enable window scaling,
but the scale in one direction might differ from the scale in the other direction.
The $shift.cnt$ field is the 2-base logarithm of the window scale of the sender.
Valid values of $shift.cnt$ are in the $[0,14]$ range.

\subsubsection*{Persistence timer}

When the reciever buffer is full, it sends a 0 length window in the ACK segment
to stop the sender. Later if the application reads the data,
it will repeat the last ACK with an updated window to resume data sending.
If this ACK segment is lost, then the sender is not notified, so a deadlock
happens.

To avoid this situation the sender starts a Persistence Timer when it received
a 0 size window. If the timer expires before the window is increased it send
a probe segment with 1 byte of data. It will receive the current window of the
receiver in the response to this segment.

\subsubsection*{Keepalive timer}

TCP keepalive timer is used to detect dead connections.

\subsubsection{Transmission policies}
\label{sec:transport:transmission-policies}

\subsubsection*{Retransmissions}

% source: RFC1222 4.3.2.1 and Tannenbaum 6.5.10

When the sender TCP sends a TCP segment it starts a
retransmission timer.
If the ACK arrives before the timer expires it is stopped,
otherwise it triggers a retransmission of the segment.

If the retransmission timeout (RTO) is too high, then lost segments
causes high delays, if it is too low, then the receiver gets
too many useless duplicated segments. For optimal behaviour, the
timeout must be dynamically determined.

Jacobson suggested to measure the RTT mean and deviation
and apply the timeout:

$$ RTO = RTT + 4 * D $$

Here RTT and D are the measured smoothed roundtrip time and its
smoothed mean deviation. They are initialized to 0 and updated each time an
ACK segment received according to the following formulas:

$$ RTT = \alpha*RTT + (1-\alpha) * M $$

$$ D = \alpha*D + (1-\alpha)*|RTT-M| $$

where $M$ is the time between the segments send and the acknowledgment
arrival. Here the $\alpha$ smoothing factor is typically $7/8$.

One problem may occur when computing the round trip: if the
retransmission timer timed out and the segment is sent again,
then it is unclear that the received ACK is a response to the
first transmission or to the second one. To avoid confusing the
RTT calculation, the segments that have been retransmitted
do not update the RTT. This is known as Karn's modification.
He also suggested to double the $RTO$ on each failure until the
segments gets through (``exponential backoff'').

\subsubsection*{Delayed ACK algorithm}

% RFC1122 4.2.3.2

A host that is receiving a stream of TCP data segments can
increase efficiency in both the Internet and the hosts by
sending fewer than one ACK (acknowledgment) segment per data
segment received; this is known as a "delayed ACK" [TCP:5].

Delay is max. 500ms.

A delayed ACK gives the application an opportunity to
update the window and perhaps to send an immediate
response.  In particular, in the case of character-mode
remote login, a delayed ACK can reduce the number of
segments sent by the server by a factor of 3 (ACK,
window update, and echo character all combined in one
segment).

In addition, on some large multi-user hosts, a delayed
ACK can substantially reduce protocol processing
overhead by reducing the total number of packets to be
processed [TCP:5].  However, excessive delays on ACK's
can disturb the round-trip timing and packet "clocking"
algorithms [TCP:7].

% RFC2581 3.2

a TCP receiver SHOULD send an immediate ACK
when the incoming segment fills in all or part of a gap in the
sequence space.

\subsubsection*{Nagle's algorithm}

RFC896 describes the ``small packet problem": when the application
sends single-byte messages to the TCP, and it transmitted immediatly
in a 41 byte TCP/IP packet (20 bytes IP header, 20 bytes TCP header,
1 byte payload), the result is a 4000\% overhead that can cause
congestion in the network.

The solution to this problem is to delay the transmission until
enough data received from the application and send all collected
data in one packet. Nagle proposed that
when a TCP connection has outstanding data that has not
yet been acknowledged, small segments should not be sent
until the outstanding data is acknowledged.

\subsubsection*{Silly window avoidance}

The Silly Window Syndrome (SWS) is described in RFC813. It occurs when
a TCP receiver advertises a small window and the TCP sender immediately
sends data to fill the window. Let's take the example when the sender
process writes a file into the TCP stream in big chunks, while the
receiver process reads the bytes one by one. The first few bytes
are transmitted as whole segments until the receiver buffer
becomes full. Then the application reads one
byte, and a window size 1 is offered to the sender. The sender sends
a segment with 1 byte payload immediately, the receiver buffer becomes
full, and after reading 1 byte, the offered window is 1 byte again.
Thus almost the whole file is transmitted in very small segments.

In order to avoid SWS, both sender and receiver must try to avoid this
situation. The receiver must not advertise small windows and the sender
must not send small segments when only a small window is advertised.

In RFC813 it is offered that
\begin{enumerate}
  \item the receiver should not advertise windows that is smaller than the maximum
        segment size of the connection
  \item the sender should wait until the window is large enough for a maximum sized
        segment.
\end{enumerate}

\subsubsection*{Timestamp option}

Efficient retransmissions depends on precious RTT measurements.
Packet losses can reduce the precision of these measurements radically.
When a segment lost, the ACKs received in that window can not be used;
thus reducing the sample rate to one RTT data per window. This is
unacceptable if the window is large.

The proposed solution to the problem is to use a separate timestamp
field to connect the request and the response on the sender side.
The timestamp is transmitted as a TCP option. The option contains two
32-bit timestamps:

\begin{pdfonly}
\begin{center}
\begin{bytefield}{80}
\bitbox{8}{Kind=5} &
\bitbox{8}{Length=10} &
\bitbox{32}{TS Value} &
\bitbox{32}{TS Echo Reply} &
\end{bytefield}
\end{center}
\end{pdfonly}

\begin{htmlonly}
[WARNING: Cannot translate LaTeX bitbox to HTML!]
\end{htmlonly}

Here the TS Value (TSVal) field is the current value of the timestamp
clock of the TCP sending the option, TS Echo Reply (TSecr) field is
0 or echoes the timestamp value of that was sent by the remote TCP.
The TSscr field is valid only in ACK segments that acknowledges new
data. Both parties should send the TS option in their SYN segment
in order to allow the TS option in data segments.

The timestamp option can also be used for PAWS (protection against wrapped
sequence numbers).


\subsubsection{Congestion control}
\label{sec:tcp-transport:congestion-control}

Flow control allows the sender to slow down the transmission when the
receiver can not accept them because of memory limitations. However
there are other situations when a slow down is desirable. If the sender
transmits a lot of data into the network it can overload the processing
capacities of the network nodes, so packets are lost in the network
layer.

For this purpose another window is maintained at the sender side, the
congestion window (CWND). The congestion window is a sender-side limit
on the amount of data the sender can transmit into the network before
receiving ACK. More precisely, the sender can send at most $max(CWND, WND)$
bytes above SND.UNA, therefore $SND.NXT < SND.UNA + max(CWND, WND)$ is
guaranteed.

The size of the congestion window is dinamically determined by monitoring
the state of the network.

% RFC2581
%
% Definitions:
% SMSS: sender maximum segment size
% RMSS: receiver maximum segment size (default 536)
% rwnd: most recently advertised receiver window
% IW: initial sender's congestion window
% LW: loss window, size of congestion window after a TCP sender detects loss
% RW: restart window, size of congestion window after a TCP restarts transmission after an idle period
% fligth size: amount of data has been sent but not yet acknowledged
% cwnd: congestion window, sender-size limit on the amount of data the sender
%       can transmit into the network before receiving an ACK
% rwnd: receiver advertised window, receiver-side limit on the amount of outstanding data
% sstresh: whether slow start or congestion avoidance used
%
% IW <= 2*MSS


\subsubsection*{Slow Start and Congestion Avoidance}

There are two algorithm that updates the congestion window, ``Slow Start''
and ``Congestion Avoidance''. They are specified in RFC2581.

\begin{pseudocode}
$cwnd \gets 2*SMSS$
$ssthresh \gets$ upper bound of the window (e.g. $65536$)
whenever ACK received
  if $cwnd < ssthresh$
    $cwnd \gets cwnd + SMSS$
  otherwise
    $cwnd \gets cwnd + SMSS*SMSS/cwnd$
whenever packet loss detected
  $cwnd \gets SMSS$
  $ssthresh \gets max(FlightSize/2, 2*SMSS)$
\end{pseudocode}

Slow Start means that when the connection opened the sender initially
sends the data with a low rate. This means that the initial
window (IW) is at most 2 MSS, but no more than 2 segments. If there was no packet loss,
then the congestion window is increased rapidly, it is doubled in each flight.
When a packet loss is detected, the congestion window is reset to 1 MSS (loss window, LW)
and the ``Slow Start'' is applied again.

\begin{note}
RFC3390 increased the IW to roughly 4K bytes: $min(4*MSS, max(2*MSS, 4380))$.
\end{note}

When the congestion window reaches a certain limit (slow start threshold),
the ``Congestion Avoidance'' algorithm is applied. During ``Congestion Avoidance''
the window is incremented by 1 MSS per round-trip-time (RTT). This is usually
implemented by updating the window according to the $ cwnd += SMSS*SMSS/cwnd $
formula on every non-duplicate ACK.

The Slow Start Threshold is updated when a packet loss is detected.
It is set to $max(FlightSize/2, 2*SMSS)$.

How the sender estimates the flight size? The data sent, but not yet acknowledged.

How the sender detect packet loss? Retransmission timer expired.


\subsubsection*{Fast Retransmit and Fast Recovery}

RFC2581 specifies two additional methods to increase the efficiency
of congestion control: ``Fast Retransmit'' and ``Fast Recovery''.

``Fast Retransmit'' requires that the receiver signal the event,
when an out-of-order segment arrives. It is achieved by sending
an immediate duplicate ACK. The receiver also sends an immediate
ACK when the incoming segment fills in a gap or part of a gap.

When the sender receives the duplicated ACK it knows that some
segment after that sequence number is received out-of-order or
that the network duplicated the ACK. If 3 duplicated ACK received
then it is more likely that a segment was dropped or delayed.
In this case the sender starts to retransmit the segments
immediately.

``Fast Recovery'' means that ``Slow Start'' is not applied
when the loss is detected as 3 duplicate ACKs. The arrival
of the duplicate ACKs indicates that the network is not fully
congested, segments after the lost segment arrived, as well
the ACKs.

% Details?
%
%    1.  When the third duplicate ACK is received, set ssthresh to no more
%        than the value given in equation 3.
%
%    2.  Retransmit the lost segment and set cwnd to ssthresh plus 3*SMSS.
%        This artificially "inflates" the congestion window by the number
%        of segments (three) that have left the network and which the
%        receiver has buffered.
%
%    3.  For each additional duplicate ACK received, increment cwnd by
%        SMSS.  This artificially inflates the congestion window in order
%        to reflect the additional segment that has left the network.
%
%    4.  Transmit a segment, if allowed by the new value of cwnd and the
%        receiver's advertised window.
%
%    5.  When the next ACK arrives that acknowledges new data, set cwnd to
%        ssthresh (the value set in step 1).  This is termed "deflating"
%        the window.
%
%        This ACK should be the acknowledgment elicited by the
%        retransmission from step 1, one RTT after the retransmission
%        (though it may arrive sooner in the presence of significant out-
%        of-order delivery of data segments at the receiver).
%        Additionally, this ACK should acknowledge all the intermediate
%        segments sent between the lost segment and the receipt of the
%        third duplicate ACK, if none of these were lost.

\subsubsection*{Loss Recovery Using Limited Transmit}

If there is not enough data to be send after a lost segment,
then the Fast Retransmit algorithm is not activated, but the
costly retranmission timeout used.

RFC3042 suggests that the sender TCP should send a new data segment
in response to each of the first two duplicate acknowledgement. Transmitting
these segments increases the probability that TCP can recover from a single
lost segment using the fast retransmit algorithm, rather than using a costly
retransmission timeout.

\subsubsection*{Selective Acknowledgments}

% RFC2018

With selective
acknowledgments (SACK), the data receiver can inform the sender about all
segments that have arrived successfully, so the sender need
retransmit only the segments that have actually been lost.

With the help of this information the sender can detect
\begin{itemize}
  \item replication by the network
  \item false retransmit due to reordering
  \item retransmit timeout due to ACK loss
  \item early retransmit timeout
\end{itemize}


In the congestion control algorithms described so far
the sender has only rudimentary information about which
segments arrived at the receiver. On the other hand
the algorithms are implemented completely on the sender side,
they only require that the client sends immediate ACKs on
duplicate segments. Therefore they can work in a heterogenous
environment, e.g. a client with Tahoe TCP can communicate with
a NewReno server. On the other hand SACK must be supported by
both endpoint of the connection to be used.

If a TCP supports SACK it includes the \emph{SACK-Permitted} option
in the SYN/SYN-ACK segment when initiating the connection.
The SACK extension enabled for the connection if the \emph{SACK-Permitted}
option was sent and received by both ends. The option occupies
2 octets in the TCP header:

\begin{pdfonly}
\begin{center}
\begin{bytefield}{16}
\bitbox{8}{Kind=4} &
\bitbox{8}{Length=2}
\end{bytefield}
\end{center}
\end{pdfonly}

\begin{htmlonly}
[WARNING: Cannot translate LaTeX bitbox to HTML!]
\end{htmlonly}

If the SACK is enabled then the data receiver adds SACK option
to the ACK segments. The SACK option informs the sender about
non-contiguous blocks of data that have been received and queued.
The meaning of the \emph{Acknowledgement Number} is unchanged,
it is still the cumulative sequence number. Octets received
before the \emph{Acknowledgement Number} are kept by the receiver,
and can be deleted from the sender's buffer. However the receiver
is allowed to drop the segments that was only reported in the SACK
option.

The \emph{SACK} option contains the following fields:

\begin{pdfonly}
\begin{center}
\begin{bytefield}{32}
\bitbox[]{16}{} &
\bitbox{8}{Kind=5} &
\bitbox{8}{Length} \\
\bitbox{32}{Left Edge of 1st Block} \\
\bitbox{32}{Right Edge of 1st Block} \\
\wordbox[]{1}{$\vdots$ \\[1ex]} \\
\bitbox{32}{Left Edge of nth Block} \\
\bitbox{32}{Right Edge of nth Block}
\end{bytefield}
\end{center}
\end{pdfonly}

\begin{htmlonly}
[WARNING: Cannot translate LaTeX bitbox to HTML!]
\end{htmlonly}

Each block represents received bytes of data that are
contiguous and isolated with one exception: if a segment
received that was already ACKed (i.e. below $RCV.NXT$),
it is included as the first block of the \emph{SACK} option.
The purpose is to inform the sender about a spurious retransmission.

Each block in the option occupies 8 octets. The TCP header
allows 40 bytes for options, so at most 4 blocks can be
reported in the \emph{SACK} option (or 3 if TS option is also used).
The first block is used for reporting the most recently received
data, the following blocks repeats the most recently reported
SACK blocks. This way each segment is reported at least 3 times,
so the sender receives the information even if some ACK segment is
lost.


\textbf{SACK based loss recovery}

% RFC3517: loss recovery based on SACK

Now lets see how the sender can use the information in the
\emph{SACK} option. First notice that it can give a better
estimation of the amount of data outstanding in the network
(called $pipe$ in RFC3517).
If $highACK$ is the highest ACKed sequence number, and
$highData$ of the highest sequence number transmitted,
then the bytes between $highACK$ and $highData$ can be
in the network. However $ pipe \neq highData - highACK $
if there are lost and retransmitted segments:

$$ pipe = highData - highACK - lostBytes + retransmittedBytes $$

A segment is supposed to be lost if it was not received
but 3 segments recevied that comes after this segment in the sequence
number space.
This condition is detected by the sender by receiving
either 3 discontiguous SACKed blocks, or at least
$3*SMSS$ SACKed bytes above the sequence numbers of the
lost segment.

The transmission of data starts with a \emph{Slow Start} phase.
If the loss is detected by 3 duplicate ACK, the sender
goes into the recovery state: it sets
$cwnd$ and $ssthresh$ to $FlightSize / 2$.
It also remembers the $highData$ variable, because
the recovery state is left when this sequence number
is acknowledged.

In the recovery state it sends data
until there is space in the congestion window (i.e. $cwnd-pipe >= 1 SMSS$)
The data of the segment is choosen by the following rules (first rule that applies):

\begin{enumerate}
  \item send segments that is lost and not yet retransmitted
  \item send segments that is not yet transmitted
  \item send segments that is not yet retransmitted and possibly fills a gap
        (there is SACKed data above it)
\end{enumerate}

If there is no data to send, then the sender waits for the next ACK, updates
its variables based on the data of the received ACK, and then try to transmit
according to the above rules.

If an RTO occurs, the sender drops the collected SACK information and
initiates a Slow Start. This is to avoid a deadlock when the receiver
dropped a previously SACKed segment.

% highACK: highest ACKed sequence number
%
% highData: highest sequence number transmitted
%
% highRxt: highest sequence number which has been retransmitted
%
%
% Normal phase: before the first loss, until 3 duplicate ACK
%
% Loss recovery phase: until ACK for RecoveryPoint received
%
% On the transition to loss recovery phase
% \begin{enumerate}
%   \item RecoveryPoint=HighData
%   \item ssthresh=cwnd=FlightSize/2
%   \item compute \emph{pipe}
% \end{enumerate}
%
% In the loss recovery phase, for each incoming ACK:
%
% \begin{enumerate}
%   %\Alph
%   \item if cumulative ACK above RecoveryPoint, leave loss recovery phase
%   \item update SACK info and compute pipe
%   \item if $cwnd-pipe >= 1 SMSS$ send one or more segments (if there is data to send)
%   \item update HighRxt, HighData according to the sent bytes
%   \item increment $pipe$ by the amount of data sent
%   \item if $cwnd-pipe >= 1 SMSS$, continue sending
% \end{enumerate}
%
% Which bytes to be send are determined as follows:
%
% \begin{enumerate}
%   \item if there is a byte which is lost and not yet retransmitted, send that in 1 segment
%   \item otherwise if there is unsent data, send that in 1 segment
%   \item otherwise if there is not yet retransmitted data, and above that there is SACKed data, send that
%   \item otherwise there is no data to send
% \end{enumerate}

%%% Local Variables:
%%% mode: latex
%%% TeX-master: "usman"
%%% End:


\cleardoublepage

\chapter{Ipv4}
\label{cha:ipv4}

\section{The IPv4 Module}

\subsection{Processing time}

The C++ class of the \nedtype{Ipv4} module is derived from \cppclass{QueueBase}.
There is a processing time associated with each incoming packet.
This processing time is specified by the \fpar{procDelay} module parameter.
If a packet arrives, when the processing of a previous has not been
finished, it is placed in a FIFO queue.

The current performance model assumes that each datagram is processed
within the same time, and there is no priority between the datagrams.
If you need a more sophisticated performance model, you may change
the module implementation (the IP class), and:
\begin{enumerate}
  \item override the \ffunc{startService()} method which determines processing
        time for a packet, or
  \item use a different base class.
\end{enumerate}

\subsection{Interface with higher layer}

Higher layer protocols should be connected to the \ttt{transportIn}/\ttt{transportOut}
gates of the \nedtype{Ipv4} module.

\subsubsection*{Sending packets}

Higher layer protocols can send a packet by attaching a \cppclass{Ipv4ControlInfo}
object to their packet and sending it to the \nedtype{Ipv4} module.

% receiving IP datagrams from higher layer?

The following fields must be set in the control info:
\begin{itemize}
  \item \fvar{procotol}: the \ttt{Protocol} field of the IP datagram. Valid values
        are defined in the \ttt{IpProtocolId} enumeration.
  \item \fvar{destAddr}: the \ttt{Destination Address} of the IP datagram.
\end{itemize}

Optionally the following fields can be set too:
\begin{itemize}
\item \fvar{scrAddr}: \ttt{Source Address} of the IP datagram. If given it must match with the
      address of one of the interfaces of the node, but the datagram is not necessarily
      routed through that interface. If left unspecified, then the address of the
      outgoing interface will be used.
\item \fvar{timeToLive}: TTL of the IP datagram or -1 (unspecified). If unspecified then the TTL
      of the datagram will be 1 for destination addresses in the
      224.0.0.0 -- 224.0.0.255 range. (Datagrams with these special multicast addresses
      do not need to go further that one hop, routers does not forward these datagrams.)
      Otherwise the TTL field is determined by the \fpar{defaultTimeToLive} or
      \fpar{defaultMCTimeToLive} module parameters depending whether the destination
      address is a multicast address or not.
\item \fvar{dontFragment}: the \ttt{Don't Fragment} flag of the outgoing datagram (default is \ttt{false})
\item \fvar{diffServCodePoint}: the \ttt{Type of Service} field of the outgoing datagram.
      (ToS is called \ttt{diffServCodePoint} in \msgtype{IPv4Datagram} too.)
\item \fvar{multicastLoop}: if \ttt{true}, then a copy of the multicast datagrams
      are sent to the loopback interface, so applications on the same host can receive it.
\item \fvar{interfaceId}: id of outgoing interface (can be used to limit broadcast or restrict routing).
\item \fvar{nextHopAddr}: explicit routing info, used by Manet DSR routing. If specified, then
      \ttt{interfaceId} must also be specified. Ignored in Manet routing is disabled.
\end{itemize}

The IP module encapsulates the transport layer datagram into an \msgtype{IPv4Datagram}
and fills in the header fields according to the control info. The \ttt{Identification}
field is generated by incrementing a counter.

The generated IP datagram is passed to the routing algorithm. The routing decides if the
datagram should be delivered locally, or passed to one of the network interfaces
with a specified next hop address, or broadcasted on one or all of the network interfaces.
The details of the routing is described in the next subsection (\ref{subsec:ip_routing})
in detail.

Before sending the datagram on a specific interface, the \nedtype{Ipv4} module
checks if the packet length is smaller than the \ttt{MTU} of the interface.
If not, then the datagram is fragmented. When the \ttt{Don't Fragment} flag
forbids fragmentation, an \ttt{Destination Unreachable} ICMP error is generated
with the \ttt{Fragmentation Error (5)} error code.
\begin{note}
Each fragment will encapsulate the whole higher layer datagram, although the
length of the IP datagram corresponds to the fragment length.
\end{note}

The fragments are sent to the \nedtype{Arp} module through the \ttt{queueOut} gate.
The \nedtype{Arp} module forwards the datagram immediately to point-to-point interface
cards. If the outgoing interface is a 802.x card, then before forwarding the datagram
it performs address resolution to obtain the MAC address of the destination.

\subsubsection*{Receiving packets}

The \nedtype{Ipv4} module of hosts processes the datagrams received from the network
in three steps:
\begin{enumerate}
  \item Reassemble fragments
  \item Decapsulate the transport layer datagram
  \item Dispatch the datagram to the appropriate transport protocol
\end{enumerate}

When a fragment received, it is added to the fragment buffer of the IP.
If the fragment was the last fragment of a datagram, the processing of
the datagram continues with step 2. The fragment buffer stores the reception
time of each fragment. Fragments older than \fpar{fragmentTimeout} are
purged from the buffer. The default value of the timeout is 60s. The
timeout is only checked when a fragment is received, and at least 10s
elapsed since the last check.

An \msgtype{Ipv4ControlInfo} attached to the decapsulated transport layer packet.
The control info contains fields copied from the IP header (source and destination
address, protocol, TTL, ToS) as well as the interface id through it was received.
The control info also stores the original IP datagram, because the transport
layer might signal an ICMP error, and the ICMP packet must encapsulate the
erronous IP datagram.
\begin{note}
IP datagrams containing a DSR packet are not decapsulated, the unchanged IP
datagram is passed to the DSR module instead.
\end{note}

After decapsulation, the transport layer packet will be passed to the appropriate
transport protocol. It must be connected to one of the \ttt{transportOut[]} gate.
The \nedtype{Ipv4} module finds the gate using the \ttt{protocol id}$\rightarrow$
\ttt{gate index} mapping given in the \fpar{protocolMapping} string parameter.
The value must be a comma separated list of ''<protocol\_id>:<gate\_index>'' items.
For example the following line in the ini file maps TCP (6) to gate 0, UDP (17)
to gate 1, ICMP (1) to gate 2, IGMP (2) to gate 3, and RVSP (46) to gate 4.
\begin{inifile}
**.ip.protocolMapping="6:0,17:1,1:2,2:3,46:4"
\end{inifile}
If the protocol of the received IP datagram is not mapped, or the gate
is not connected, the datagram will be silently dropped.
% FIXME (#463) should send DESTINATION_UNREACHABLE/PROTOCOL_UNREACHABLE
% FIXME reassembleAndDeliver() checks that transport gate is connected, but handleReceivedICMP() does not check it

Some protocols are handled differently:
\begin{itemize}
  \item \ttt{Icmp}: ICMP errors are delivered to the protocol
        whose packet triggered the error. Only ICMP query
        requests and responses are sent to the \nedtype{Icmp} module.
  \item \ttt{IP}: sent through \ttt{preRoutingOut} gate. (bug!)
  \item \ttt{DSR}: ??? (subsection about Manet routing?)
\end{itemize}

% FIXME (#462) reassembleAndDeliver(): packets with IP protocol are sent through 'preRoutingOut' gate,
%       but there is no such gate in the IPv4 module.


\subsection{Routing, and interfacing with lower layers}
\label{subsec:ip_routing}

The output of the network interfaces are connected to the
\ttt{queueIn} gates of the \nedtype{Ipv4} module. The incoming
packets are either IP datagrams or ARP responses. The IP datagrams
are processed by the \nedtype{Ipv4} module, the ARP
responses are forwarded to the \nedtype{Arp}.

The \nedtype{Ipv4} module first checks the error bit of the
incoming IP datagrams. There is a $header length/packet length$
probability that the IP header contains the error (assuming
1~bit error). With this probability an ICMP \ttt{Parameter Problem}
generated, and the datagram is dropped.

% FIXME (#466) if IP datagram hasBitError(), but it is decided not be in the IP header,
%       then the decapsulated packet should have the bit error.

When the datagram does not contain error in the IP header,
a routing decision is made. As a result of the routing
the datagram is either delivered locally,
or sent out one or more output interface.
When it is sent out, the routing algorithm must compute the
next hop of its route. The details are differ, depending on
that the destination address is multicast address or not.

When the datagram is decided to be sent up, it is processed
as described in the previous subsection (Receiving packets).
If it is decided to be sent out through some interface, it
is actually sent to the \nedtype{Arp} module through the
\ttt{queueOut} gate. An \msgtype{IPv4RoutingDecision} control
info is attached to the outgoing packet, containing the
outgoing interface id, and the IP address of the next hop.
The \nedtype{Arp} module resolve the IP address to a hardware
address if needed, and forwards the datagram to next hop.

\subsubsection*{Unicast/broadcast routing}

When the higher layer generated the datagram, it will be processed
in these steps:
\begin{enumerate}
  \item If the destination is the address of a local interface,
  then the datagram is locally delivered.
  \item If the destination is the limited broadcast address, or a
  local broadcast address, then it will be broadcasted on one or more
  interface. If the higher layer specified am outgoing interface
  (\fvar{interfaceId} in the control info), then it will be broadcasted
  on that interface only. Otherwise if the \fpar{forceBroadcast} module
  parameter is \ttt{true}, then it will broadcasted on all interfaces
  including the loopback interface. The default value of the
  \fpar{forceBroadcast} is \ttt{false}.
  \item If the higher layer provided the routing decision (Manet routing),
  then the datagram will be sent through the specified interface to the
  specified next hop.
  \item Otherwise IP finds the outgoing interface and the address of the
  next hop by consulting the routing table, and sends the datagram
  to the next hop. If no route
  found, then a \ttt{Destination Unreachable} ICMP error is generated.
\end{enumerate}


Incoming datagrams having unicast or broadcast destination addresses are
routed in the following steps:

\begin{enumerate}
  \item Deliver datagram locally. If the destination address is a local
  address, the limited broadcast address (255.255.255.255), or a local
  broadcast address, then it will be sent to the transport layer.
  \item Drop packets received from the network when IP forwarding is disabled.
  \item Forward the datagram to the next hop. The next hop is
    determined by looking up the best route to the destination from the
    routing table. If the gateway is set in the route, then the datagram
    will be forwarded to the gateway, otherwise it is sent directly to the
    destination. If no route is found, then
    a \ttt{Destination Unreachable} ICMP error is sent to the source of the
    datagram.
\end{enumerate}

\subsubsection*{Multicast routing}

Outgoing multicast datagrams are handled as follows:
\begin{enumerate}
  \item If the higher layer set the \fvar{multicastLoop} variable
  to \ttt{true}, the IP will send up a copy of the datagram
  through the loopback interface.
  \item Determine the outgoing interface for the multicast datagram,
  and send out the datagram through that interface. The outgoing
  interface is determined by the following rules:
  \begin{compactenum}
    \item if the HL specified the outgoing interface in the control
    info, the it will be used
    \item otherwise use the interface of the route configured in the
    routing table for the destination address
    \item if no route found, then use the interface whose address
    matches the source address of the datagram
    \item if the HL did not specify the source address, then use
    the first multicast capable interface
    \item if no such interface found, then the datagram is unroutable
    and droppped
  \end{compactenum}
\end{enumerate}


Incoming multicast datagrams are forwarded according to their source address
(Reverse Path Forwarding), i.e. datagrams are sent away from their sources instead
towards their destinations. The multicast routing table maintains a spanning tree
for each source network and multicast group. The source network is the root of the
tree, and there is a path to each LAN that has members of the multicast group.
Each node expects the multicast datagram to arrive from their parent and forwards
them towards their children. Multicast forwarding loops are avoided by dropping the
datagrams not arrived on the parent interface.

More specifically, the routing routine for multicast datagrams performs these steps:
\begin{enumerate}
  \item Deliver a copy of the datagram locally. If the interface on which
  the datagram arrived belongs to the multicast group specified by the
  destination address, it is sent up to the transport layer.
  \item Discard incoming packets that can not be delivered locally and
  can not be forwarded.
  A non-local packet can not be forwarded if multicast forwarding is disabled,
  the destination is a link local multicast address (224.0.0.x), or
  the TTL field reached 0.
  \item Discard the packet if no multicast route found, or
  if it did not arrive on the parent interface of the route
  (to avoid multicast loops). If the parent is not set in the route,
  then the shortest path interface to the source is assumed.
  \item Forward the multicast datagram.
  A copy of the datagram is sent on each child interface described by
  multicast routes (except the incoming interface). Interfaces may have
  a \fvar{ttlThreshold} parameter, that limits the scope of the multicast:
  only datagrams with higher TTL are forwarded.
\end{enumerate}

\section{The IPv4RoutingTable Module}

Interfaces are dynamically registered: at the start of the simulation,
every L2 module adds its own interface entry to the table.

The route table is read from a file; the file can
also fill in or overwrite interface settings. The route table can also
be read and modified during simulation, typically by routing protocol
implementations (e.g. OSPF).

Entries in the route table are represented by \cppclass{Ipv4Route} objects.
\cppclass{Ipv4Route} objects can be polymorphic: if a routing protocol needs
to store additional data, it can simply subclass from \cppclass{Ipv4Route},
and add the derived object to the table. The \cppclass{Ipv4Route} object
has the following fields:
\begin{itemize}
  \item \ttt{host} is the IP address of the target of the route (can be a host or network).
                   When an entry searched for a given destination address, the destination
                   address is compared with this \ttt{host} address using the \ttt{netmask}
                   below, and the longest match wins.
  \item \ttt{netmask} used when comparing \ttt{host} with the detination address.
                     It is 0.0.0.0 for the default route, 255.255.255.255 for
                     host routes (exact match), or the network or subnet mask
                     for network routes.
  \item \ttt{gateway} is the IP address of the gateway for indirect routes, or
                      0.0.0.0 for direct routes. Note that 0.0.0.0 can be used
                      even if the destination is not directly connected to this
                      node, but can be found using proxy ARP.
  \item \ttt{interface} the outgoing interface to be used with this route.
  \item \ttt{type} \ttt{DIRECT} or \ttt{REMOTE}. For direct routes, the next hop
                   address is the destination address, for remote routes it is
                   the gateway address.
  \item \ttt{source} \ttt{MANUAL}, \ttt{IFACENETMASK}, \ttt{RIP}, \ttt{OSPF},
        \ttt{BGP}, \ttt{ZEBRA}, \ttt{MANET}, or \ttt{MANET2}. \ttt{MANUAL} means
        that the route was added by a routing file, or a network configurator.
        \ttt{IFACENETMASK} routes are added for each interface of the node.
        Other values means that the route is managed by the specific routing
        daemon.
  \item \ttt{metric} the ``cost'' of the route. Currently not used when choosing
                     the best route.
\end{itemize}

% TODO describe Reverse Path Forwarding

In multicast routers the routing table contains multicast routes too.
A multicast route is represented by an instance of the \cppclass{Ipv4MulticastRoute}
class. The \cppclass{Ipv4MulticastRoute} instance stores the following fields:
\begin{itemize}
  \item \fvar{origin} IP address of the network of the source of the datagram
  \item \fvar{originNetmask} netmask of the source network
  \item \fvar{group} the multicast group to be matched the destination of the
  datagram. If unspecified, then the route matches with
  \item \fvar{parent} interface towards the parent link in the multicast tree.
  Only those datagrams are forwarded that arrived on the parent interface.
  \item \fvar{children} the interfaces on which the multicast datagram to be forwarded.
  Each entry contains a flag indicating if this interface is a leaf in the multicast
  tree. The datagram is forwarded to leaf interfaces only if there are known members
  of the group in the attached LAN.
  \item \fvar{source} enumerated value identifying the creator of the entry. \ttt{MANUAL}
  for static routes, \ttt{DVRMP} for the DVMRP routers, \ttt{PIM\_SM} for PIM SM routers.
  \item \fvar{metric} the ``cost`` of the route.
\end{itemize}

When there are several multicast routes matching the source and destination
of the datagram, then the forwarding algorithm chooses the one with the
\begin{enumerate}
  \item the longest matching source
  \item the more specific group
  \item the smallest metric.
\end{enumerate}


\section{The ICMP Module}

The \nedtype{Icmp} module has two methods which can be used by other modules
to send ICMP error messages:
\begin{itemize}
  \item \ffunc[sendErrorMessage]{sendErrorMessage(IPv4Datagram*, ICMPType, ICMPCode)}
        used by the network layer to report erronous IPv4 datagrams. The ICMP header
        fields are set to the given type and code, and the ICMP message will encapsulate
        the given datagram.
  \item \ffunc[sendErrorMessage]{sendErrorMessage(cPacket*, IPv4ControlInfo*, ICMPType, ICMPCode)}
        used by the transport layer components to report erronous packets. The transport
        packet will be encapsulated into an IP datagram before wrapping it into the ICMP message.
\end{itemize}

The \nedtype{Icmp} module can be accessed from other modules of the node by calling
\ffunc{ICMPAccess::get()}.

When an incoming ICMP error message is received, the \nedtype{Icmp} module
sends it out on the \ttt{errorOut} gate unchanged. It is assumed that an
external module is connected to \ttt{errOut} that can process the error
packet. There is a simple module (\nedtype{ErrorHandling}) that simply
logs the error and drops the message. Note that the \nedtype{Ipv4} module
does not send REDIRECT, DESTINATION\_UNREACHABLE,
TIME\_EXCEEDED and PARAMETER\_PROBLEM messages to the \nedtype{Icmp} module,
it will send them to the transport layer module that sent the bogus
packet encapsulated in the ICMP message.
\begin{note}
ICMP protocol encapsulates only the IP header + 8 byte following the IP header
from the bogus IP packet. The ICMP packet length computed from this truncated
packet, despite it encapsulates the whole IP message object.
As a consequence, calling \ffunc{decapsulate()} on the ICMP message
will cause an ``packet length became negative'' error. To avoid this,
use \ffunc{getEncapsulatedMsg()} to access the IP packet that caused the ICMP
error.
\end{note}

The \nedtype{Icmp} module receives ping commands on the \ttt{pingIn}
gate from the application. The ping command can be any packet
having an \cppclass{Ipv4ControlInfo} control info. The packet
will be encapsulated with an \msgtype{ICMPMessage} and
handed over to the IP.

If \nedtype{Icmp} receives an echo request from IP, the original
message object will be returned as the echo reply. Of course,
before sending back the object to IP, the source and destination
addresses are swapped and the message type changed to ICMP\_ECHO\_REPLY.

When an ICMP echo reply received, the application message decapsulated
from it and passed to the application through the \ttt{pingOut} gate.
The \cppclass{Ipv4ControlInfo} also copied from the \msgtype{ICMPMessage}
to the application message.

% FIXME ICMP TIMESTAMP requests are processed as ECHO requests



\cleardoublepage

\chapter{IPv6 and Mobile IPv6}
\label{cha:ipv6}

TODO how to send and receive

TODO C++ interfaces among members of the protocol family

TODO how to extend



\cleardoublepage

\chapter{Other Network Protocols}
\label{cha:other-network-protocols}

\section{Overview}
\label{sec:networkprotocols:overview}

Network layer protocols in INET are not restricted to IPv4 and IPv6. INET nodes such as
\nedtype{Router} and \nedtype{StandardHost} can be configured to use an alternative
network layer protocols instead of, or in addition to, IPv4 and IPv6.

Node models contain three optional network layers that can be individually
turned on or off:

\begin{ned}
ipv4: <ipv4NetworkLayerType> like INetworkLayer if hasIpv4;
ipv6: <ipv6NetworkLayerType> like INetworkLayer if hasIpv6;
generic: <networkLayerType> like INetworkLayer if hasGn;
\end{ned}

In the default configuration, only IPv4 is turned on. If you want to use an
alternative network layer protocol instead of IPv4/IPv6, your configuration will
look something like this:

\begin{inifile}
**.hasIpv4 = false
**.hasIpv6 = false
**.hasGn = true
**.networkLayerType = "WiseRouteNetworkLayer"
\end{inifile}

The list of alternative network layers includes:

\begin{itemize}
  \item \nedtype{SimpleNetworkLayer} is a generic network layer where the
    concrete protocol type is a parameter
  \item \nedtype{NextHopNetworkLayer} is a network layer specialized
    for the ``Next-Hop Forwarding Protocol'', an abstract implementation of the
    next-hop routing concept
  \item \nedtype{WiseRouteNetworkLayer} is specialized for the Wise Route protocol
\end{itemize}

The list of network layer protocols that can be plugged into
\nedtype{SimpleNetworkLayer} includes:

\begin{itemize}
  \item \nedtype{Flooding} implements controlled flooding
  \item \nedtype{WiseRoute} implements Wise Route, a convergecasting protocol for wireless sensor networks
  \item \nedtype{ProbabilisticBroadcast} implements a multi-hop ad-hoc data dissemination protocol
  \item \nedtype{AdaptiveProbabilisticBroadcast} is a variant of the previous one
\end{itemize}

In addition to the network layer protocol, \nedtype{SimpleNetworkLayer}
includes an instance of \ttt{GlobalArp} for address resolution,
and an instance of \nedtype{EchoProtocol}, a module type that
implements a simple \textit{ping}-like protocol.

All the above network protocols can work with IPv4 addresses, IPv6 addresses,
use MAC address as network address (this is sometimes useful in WSNs),
or use sythetic addresses that only make sense within the simulation.
\footnote{This is possible because the implementation of these modules
simply use the \ttt{L3Address} C++ class, which is a variant type capable of
holding several types of L3 addresses.}

In apps, you need to specify which network layer protocol you want to use.
For example:

\begin{inifile}
**.app[*].networkProtocol = "flood"
\end{inifile}

TODO all apps support that?

\section{Protocols}
\label{sec:networkprotocols:protocols}

\subsection{Flooding}
\label{sec:networkprotocols:flooding}

\nedtype{Flooding} is a simple flooding protocol for network-level broadcast.
It remembers already broadcast messages, and does not rebroadcast
them if it gets another copy of that message.

\subsection{ProbabilisticBroadcast}
\label{sec:networkprotocols:probabilisticbroadcast}

\nedtype{ProbabilisticBroadcast} is a multi-hop ad-hoc data dissemination
protocol based on probabilistic broadcast.

This method reduces the number of packets sent on the channel (reducing the
broadcast storm problem) at the risk of some nodes not receiving the data.
It is particularly interesting for mobile networks.

The transmission probability for each attempt, the time between two transmission
attempts, the maximum number of broadcast transmissions of a packet, and
some other settings are parameters.

\subsection{AdaptiveProbabilisticBroadcast}
\label{sec:networkprotocols:adaptiveprobabilisticbroadcast}

\nedtype{AdaptiveProbabilisticBroadcast} is a variant of
\nedtype{ProbabilisticBroadcast} that automatically adjusts transmission
probabilities depending on the estimated number of neighbours.

\subsection{WiseRoute}
\label{sec:networkprotocols:wiseroute}

\nedtype{WiseRoute} implements Wise Route, a simple loop-free routing algorithm
that builds a routing tree from a central network point, designed for sensor
networks and convergecast traffic (WIreless SEnsor routing).

The sink (the device at the center of the network) broadcasts a route building
message. Each network node that receives it selects the sink as parent in the
routing tree, and rebroadcasts the route building message. This procedure
maximizes the probability that all network nodes can join the network, and
avoids loops.

The \fpar{sinkAddress} parameter specifies the sink network address,
\fpar{rssiThreshold} is a threshold to avoid using bad links (with too low RSSI
values) for routing, and \fpar{routeFloodsInterval} should be set to zero for
all nodes except the sink. Each \fpar{routeFloodsInterval}, the sink restarts
the tree building procedure. Set it to a large value if you do not want the tree
to be rebuilt.

\subsection{NextHopForwarding}
\label{sec:networkprotocols:nexthopforwarding}

The \nedtype{NextHopForwarding} module is an implementation of the next-hop
forwarding concept. (It can be thought of as an abstracted version of IP.)

The protocol needs an additional module, a \nedtype{NextHopRoutingTable} for its
operation. The routing table module is included in the
\nedtype{NextHopNetworkLayer} compound module.


\section{Address Types}
\label{sec:networkprotocols:address-types}

The following address types are available:

\begin{itemize}
  \item IPv4 address
  \item IPv6 address
  \item MAC address
  \item module ID
  \item module path
\end{itemize}

Protocols described in this chapter work with ``generic'' L3 addresses,
they can use any address type.

When choosing IPv4 addresses, an \nedtype{IPv4NetworkConfigurator} global
instance can be used to assign addresses to network interfaces. (Note that
\nedtype{IPv4NetworkConfigurator} also needs a per-node instance
of \nedtype{Ipv4NodeConfigurator} for it to work.)

\section{Address Resolution}
\label{sec:networkprotocols:address-resolution}

Address resolution is done by \nedtype{GlobalArp}.
If the address type is IPv4, \nedtype{Arp} can be used instead of
\nedtype{GlobalArp}.


\section{/////////////////////////////////////////////////////////}
\label{sec:networkprotocols:junk}


\section{InternetCloud}
\label{sec:networkprotocols:internetcloud}

This module is an IPv4 router that can delay or drop packets (while retaining
their order) based on which interface card the packet arrived on and 
on which interface It is leaving the cloud. The delayer module is replacable.

By default the delayer module is ~MatrixCloudDelayer which lets you configure
the delay, drop and datarate parameters in an XML file. Packet flows, as defined
by incoming and outgoing interface pairs, are independent of each other.

The ~InternetCloud module can be used only to model the delay between two hops, but
it is possible to build more complex networks using several ~InternetCloud modules.

\section{PIM}
\label{sec:networkprotocols:pim}

Protocol Independent Multicast -- not a network protocol

models: \nedtype{PimSm}, \nedtype{PimDm}; \nedtype{Pim} is a compound module



%%% Local Variables:
%%% mode: latex
%%% TeX-master: "usman"
%%% End:


\cleardoublepage

\chapter{Internet Routing}
\label{cha:routing}

\section{Overview}
\label{sec:routing:overview}

INET Framework has models for several internet routing protocols, including
RIP, OSPF and BGP.

The easiest way to add routing to a network is to use the \nedtype{Router}
NED type for routers. \nedtype{Router} contains a conditional instance
for each of the above protocols. These submodules can be enabled by
setting the \ttt{hasRip}, \ttt{hasOspf} and/or \ttt{hasBgp} parameters to
\ttt{true}.

Example:

\begin{inifile}
**.hasRip = true
\end{inifile}

There are also NED types called \nedtype{RipRouter}, \nedtype{OspfRouter},
\nedtype{BgpRouter}, which are all \nedtype{Router}'s with appropriate
routing protocol enabled.

\section{RIP}
\label{sec:routing:rip}

RIP (Routing Information Protocol) is a distance-vector routing protocol which
employs the hop count as a routing metric. RIP prevents routing loops by
implementing a limit on the number of hops allowed in a path from source to
destination.  RIP uses the \textit{split horizon with poison reverse} technique
to work around the ``count-to-infinity'' problem.

The \nedtype{Rip} module implements distance vector routing as specified in RFC
2453 (RIPv2) and RFC 2080 (RIPng). The behavior can be selected by setting the
\fpar{mode} parameter to either \ttt{"RIPv2"} or \ttt{"RIPng"}. Protocol
configuration such as link metrics and per-interface operation mode (such as
whether RIP is enabled on the interface, and whether to use split horizon)
can be specified in XML using the \ttt{ripConfig} parameter.

The following example configures a \nedtype{Router} module to use RIPv2:

\begin{inifile}
**.hasRip = true
**.mode = "RIPv2"
**.ripConfig = xmldoc("RIPConfig.xml")
\end{inifile}

The configuration file specifies the per interface parameters.
Each \ttt{<interface>} element configures one or more interfaces;
the \ttt{hosts}, \ttt{names}, \ttt{towards}, \ttt{among} attributes
select the configured interfaces (in a similar way as with
\nedtype{Ipv4NetworkConfigurator} \ref{cha:network-autoconfiguration}).

Additional attributes:

\begin{itemize}
  \item \ttt{metric}: metric assigned to the link, default value is 1.
        This value is added to the metric of a learned route,
        received on this interface. It must be an integer in
        the [1,15] interval.
  \item \ttt{mode}: mode of the interface.
\end{itemize}

The mode attribute can be one of the following:

\begin{itemize}
  \item \ttt{'NoRIP'}: no RIP messages are sent or received on this interface.
  \item \ttt{'NoSplitHorizon'}: no split horizon filtering; send all routes to
        neighbors.
  \item \ttt{'SplitHorizon'}: do not sent routes whose next hop is the neighbor.
  \item \ttt{'SplitHorizonPoisenedReverse'} (default): if the next hop is the neighbor, then
  set the metric of the route to infinity.
\end{itemize}

The following example sets the link metric between router
\ttt{R1} and \ttt{RB} to 2, while all other links will have a metric of 1.

\begin{XML}
<RIPConfig>
  <interface among="R1 RB" metric="2"/>
  <interface among="R? R?" metric="1"/>
</RIPConfig>
\end{XML}

\section{OSPF}
\label{sec:routing:ospf}

OSPF (Open Shortest Path First) is a routing protocol for IP networks.
It uses a link state routing (LSR) algorithm and falls into the group
of interior gateway protocols (IGPs), operating within a single
autonomous system (AS).

\nedtype{OspfRouter} is a \nedtype{Router} with the OSPF protocol enabled.

The \nedtype{Ospf} module implements OSPF protocol version 2. Areas and routers
can be configured using an XML file specified by the \ttt{ospfConfig} parameter.
Various parameters for the network interfaces can be specified also in the XML
file or as a parameter of the \nedtype{Ospf} module.

\begin{inifile}
**.ospf.ospfConfig = xmldoc("ASConfig.xml")
**.ospf.helloInterval = 12s
**.ospf.retransmissionInterval = 6s
\end{inifile}

The \ttt{<OSPFASConfig>} root element may contain \ttt{<Area>} and \ttt{<Router>}
elements with various attributes specifying the parameters for the network
interfaces. Most importantly \ttt{<Area>} contains \ttt{<AddressRange>} elements
enumerating the network addresses that should be advertized by the protocol.
Also \ttt{<Router>} elements may contain data for configuring various pont-to-point
or broadcast interfaces.

\begin{XML}
<?xml version="1.0"?>
<OSPFASConfig xmlns:xsi="http://www.w3.org/2001/XMLSchema-instance" xsi:schemaLocation="OSPF.xsd">
  <!-- Areas -->
  <Area id="0.0.0.0">
    <AddressRange address="H1" mask="H1" status="Advertise" />
    <AddressRange address="H2" mask="H2" status="Advertise" />
    <AddressRange address="R1>R2" mask="R1>R2" status="Advertise" />
    <AddressRange address="R2>R1" mask="R2>R1" status="Advertise" />
  </Area>
  <!-- Routers -->
  <Router name="R1" RFC1583Compatible="true">
    <BroadcastInterface ifName="eth0" areaID="0.0.0.0" interfaceOutputCost="1" routerPriority="1" />
    <PointToPointInterface ifName="eth1" areaID="0.0.0.0" interfaceOutputCost="2" />
  </Router>
  <Router name="R2" RFC1583Compatible="true">
    <PointToPointInterface ifName="eth0" areaID="0.0.0.0" interfaceOutputCost="2" />
    <BroadcastInterface ifName="eth1" areaID="0.0.0.0" interfaceOutputCost="1" routerPriority="2" />
  </Router>
</OSPFASConfig>
\end{XML}

\section{BGP}
\label{sec:routing:bgp}

BGP (Border Gateway Protocol) is a standardized exterior gateway protocol
designed to exchange routing and reachability information among
autonomous systems (AS) on the Internet.

\nedtype{BgpRouter} is a \nedtype{Router} with the BGP protocol enabled.

The \nedtype{Bgp} module implements BGP Version 4. The model implements
RFC 4271, with some limitations. Autonomous Systems and rules can be
configured in an XML file that can be specified in the \ttt{bgpConfig}
parameter.

\begin{inifile}
**.bgpConfig = xmldoc("BGPConfig.xml")
\end{inifile}

The configuration file may contain \ttt{<TimerParams>}, \ttt{<AS>} and
\ttt{Session} elements at the top level.

\begin{itemize}
  \item \ttt{<TimerParams>}: allows specifying various timing parameters
  for the routers.
  \item \ttt{<AS>}: defines Autonomous Systems, routers and rules to be applied.
  \item \ttt{<Session>}: specifies open sessions between edge routers. It must
  contain exactly two \ttt{<Router exterAddr="x.x.x.x"/>} elements.
\end{itemize}

\begin{XML}
<BGPConfig xmlns:xsi="http://www.w3.org/2001/XMLSchema-instance"
  xsi:schemaLocation="BGP.xsd">

  <TimerParams>
    <connectRetryTime> 120 </connectRetryTime>
    <holdTime> 180 </holdTime>
    <keepAliveTime> 60 </keepAliveTime>
    <startDelay> 15 </startDelay>
  </TimerParams>

  <AS id="60111">
    <Router interAddr="172.1.10.255"/> <!--Router A1-->
    <Router interAddr="172.1.20.255"/> <!--Router A2-->
  </AS>

  <AS id="60222">
    <Router interAddr="172.10.4.255"/> <!--Router B-->
  </AS>

  <AS id="60333">
    <Router interAddr="172.13.1.255"/> <!--Router C1-->
    <Router interAddr="172.13.2.255"/> <!--Router C2-->
    <Router interAddr="172.13.3.255"/> <!--Router C3-->
    <Router interAddr="172.13.4.255"/> <!--Router C4-->
    <DenyRouteOUT Address="172.10.8.0" Netmask="255.255.255.0"/>
    <DenyASOUT> 60111 </DenyASOUT>
  </AS>

  <Session id="1">
    <Router exterAddr="10.10.10.1" > </Router> <!--Router A1-->
    <Router exterAddr="10.10.10.2" > </Router> <!--Router C1-->
  </Session>

  <Session id="2">
    <Router exterAddr="10.10.20.1" > </Router> <!--Router A2-->
    <Router exterAddr="10.10.20.2" > </Router> <!--Router B-->
  </Session>

  <Session id="3">
    <Router exterAddr="10.10.30.1" > </Router> <!--Router B-->
    <Router exterAddr="10.10.30.2" > </Router> <!--Router C2-->
  </Session>
</BGPConfig>
\end{XML}

Inside \ttt{<AS>} elements various rules can be sepecified:

\begin{itemize}
  \item DenyRoute: deny route in IN and OUT traffic (\ttt{Address} and
        \ttt{Netmask} attributes must be specified.)
  \item DenyRouteIN : deny route in IN traffic (\ttt{Address} and
        \ttt{Netmask} attributes must be specified.)
  \item DenyRouteOUT: deny route in OUT traffic (\ttt{Address} and
        \ttt{Netmask} attributes must be specified.)
  \item DenyAS: deny routes learned by AS in IN  and OUT traffic (AS id must be
        specified as the body of the element.)
  \item DenyASIN : deny routes learned by AS in IN traffic (AS id must be
        specified as the body of the element.)
  \item DenyASOUT: deny routes learned by AS in OUT traffic (AS id must be
        specified as the body of the element.)
\end{itemize}

%%% Local Variables:
%%% mode: latex
%%% TeX-master: "usman"
%%% End:


\cleardoublepage

\chapter{Ad Hoc Routing}
\label{cha:adhoc-routing}

\section{Overview}
\label{sec:adhocrouting:overview}

In ad hoc networks, nodes are not familiar with the topology of
their networks. Instead, they have to discover it: typically,
a new node announces its presence and listens for announcements
broadcast by its neighbors. Each node learns about others nearby
and how to reach them, and may announce that it too can reach them.
The difficulty of routing may be compounded by the fact that
nodes may be mobile, which results in a changing topology.

Ad hoc routing protocols fall in two broad categories: proactive
and reactive. \textit{Proactive} or \textit{table-driven} protocols
maintain fresh lists of destinations and their routes by periodically
distributing routing tables throughout the network.
\textit{Reactive} or \textit{on-demand} protocols find a route on demand
by flooding the network with Route Request packets.

The INET Framework contains the implementation of several ad hoc routing
protocols including AODV, DSDV, DYMO and GPSR.

The easiest way to add routing to an ad hoc network is to use the
\nedtype{ManetRouter} NED type for nodes. \nedtype{ManetRouter}
contains a submodule named \ttt{routing} whose type is a parameter,
so it can be configured to be an AODV router, a DYMO router, or a
router of any other supported routing protocol. For example, you
can configure \nedtype{ManetRouter} nodes in the network to use
AODV with the following ini file line:

\begin{inifile}
**.routingProtocolType = "Aodv"
\end{inifile}

There are also NED types called \nedtype{AodvRouter}, \nedtype{DymoRouter},
\nedtype{DsvRouter}, \nedtype{GpsrRouter}, which are all
\nedtype{ManetRouter}'s with the routing protocol submodule type
set appropriately.


\section{AODV}
\label{sec:adhocrouting:aodv}

AODV (Ad hoc On-Demand Distance Vector Routing) is a routing protocol
for mobile ad hoc networks and other wireless ad hoc networks.
It offers quick adaptation to dynamic link conditions, low processing and
memory overhead, low network utilization, and determines unicast
routes to destinations within the ad hoc network.

The \nedtype{Aodv} module type implements AODV, based on RFC 3561.

\nedtype{AodvRouter} is a \nedtype{ManetRouter} with the routing module type
set to \nedtype{Aodv}.


\section{DSDV}
\label{sec:adhocrouting:dsdv}

DSDV (Destination-Sequenced Distance-Vector Routing) is a table-driven
routing scheme for ad hoc mobile networks based on the Bellman-Ford algorithm.

The \nedtype{Dsdv} module type implements DSDV. It is currently a partial
implementation.

\nedtype{DsdvRouter} is a \nedtype{ManetRouter} with the routing module type
set to \nedtype{Dsdv}.


\section{DYMO}
\label{sec:adhocrouting:dymo}

The DYMO (Dynamic MANET On-demand) routing protocol is successor to the
AODV routing protocol. DYMO can work as both a pro-active and as a reactive
routing protocol, i.e. routes can be discovered just when they are needed.

The \nedtype{Dymo} module type implements DYMO, based on the IETF draft
\textit{draft-ietf-manet-dymo-24}.

\nedtype{DymoRouter} is a \nedtype{ManetRouter} with the routing module type
set to \nedtype{Dymo}.


\section{GPSR}
\label{sec:adhocrouting:gpsr}

GPSR (Greedy Perimeter Stateless Routing) is a routing protocol for
mobile wireless networks that uses the geographic positions of nodes
to make packet forwarding decisions.

The \nedtype{Gpsr} module type implements GPSR, based
on the paper ``GPSR: Greedy Perimeter Stateless Routing for Wireless
Networks'' by Brad Karp and H. T. Kung, 2000. The implementation
supports both GG and RNG planarization algorithms.

\nedtype{GpsrRouter} is a \nedtype{ManetRouter} with the routing module type
set to \nedtype{Gpsr}.


%%% Local Variables:
%%% mode: latex
%%% TeX-master: "usman"
%%% End:


\cleardoublepage

\chapter{Differentiated Services}
\label{cha:diffserv}

TODO communication between components





\cleardoublepage

\chapter{The MPLS Models}
\label{cha:mpls}

\section{Overview}
\label{sec:mpls:overview}

Multi-Protocol Label Switching (MPLS) is a ``layer 2.5'' protocol for
high-performance telecommunications networks. MPLS directs data from one network
node to the next based on numeric labels instead of network addresses, avoiding
complex lookups in a routing table and allowing traffic engineering.
The labels identify virtual links (label-switched paths or LSPs, also
called MPLS tunnels) between distant nodes rather than endpoints. The routers
that make up a label-switched network are called label-switching routers (LSRs)
inside the network (``transit nodes''), and label edge routers (LER) on the
edges of the network (``ingress'' or ``egress'' nodes).

A fundamental MPLS concept is that two LSRs must agree on the meaning of the
labels used to forward traffic between and through them.
This common understanding is achieved by using signaling protocols by which one
LSR informs another of label bindings it has made. Such signaling protocols are
also called label distribution protocols. The two main label distribution
protocols used with MPLS are LDP and RSVP-TE.

INET provides basic support for building MPLS simulations. It provides models
for the MPLS, LDP and RSVP-TE protocols and their associated data structures,
and preassembled MPLS-capable router models.

\section{Core Modules}
\label{sec:mpls:core-modules}

The core modules are:

\begin{itemize}
  \item \nedtype{Mpls} implements the MPLS protocol
  \item \nedtype{LibTable} holds the LIB (Label Information Base)
  \item \nedtype{Ldp} implements the LDP signaling protocol for MPLS
  \item \nedtype{RsvpTe} implements the RSVP-TE signaling protocol for MPLS
  \item \nedtype{Ted} contains the Traffic Engineering Database
  \item \nedtype{LinkStateRouting} is a simple link-state routing protocol
  \item \nedtype{SimpleClassifier} is a configurable ingress classifier for MPLS
\end{itemize}

\subsection{Mpls}
\label{sec:mpls:mpls}

The \nedtype{Mpls} module implements the MPLS protocol. MPLS is situated between
layer 2 and 3, and its main function is to switch packets based on their labels.
For that, it relies on the data structure called LIB (Label Information Base).
LIB is fundamentally a table with the following columns: \textit{input-interface},
\textit{input-label}, \textit{output-interface}, \textit{label-operation(s)}.

Upon receiving a labelled packet from another LSR, MPLS first extracts the
incoming interface and incoming label pair, and then looks it up in local LIB.
If a matching entry is found, it applies the prescribed label operations, and
forwards the packet to the output interface.

Label operations can be the following:

\begin{itemize}
  \item \textit{Push} adds a new MPLS label to a packet. (A packet may
     contain multiple labels, acting as a stack.) When a normal IP packet
     enters an LSP, the new label will be the first label on the packet.
  \item \textit{Pop} removes the topmost MPLS label from a packet.
     This is typically done at either the penultimate or the egress router.
  \item \textit{Swap}: Replaces the topmost label with a new label.
\end{itemize}

In INET, the local LIB is stored in a \nedtype{LibTable} module in the router.

Upon receiving an unlabelled (e.g. plain IPv4) packet, MPLS first determines the
forwarding equivalence class (FEC) for the packet using an ingress classifier,
and then inserts one or more labels in the packet's newly created MPLS header.
The packet is then passed on to the next hop router for the LSP.

The ingress classifier is also a separate module; it is selected depending
on the choice of the signaling protocol.


\subsection{LibTable}
\label{sec:mpls:libtable}

\nedtype{LibTable} stores the LIB (Label Information Base), as described
in the previous section. \nedtype{LibTable} is expected to have one instance
in the router.

LIB is normally filled and maintained by label distribution protocols (RSVP-TE,
LDP), but in INET it is possible to preload it with initial contents.

The \nedtype{LibTable} module accepts an XML config file whose structure
follows the contents of the LIB table. An example configuration:

\begin{XML}
<libtable>
    <libentry>
        <inLabel>203</inLabel>
        <inInterface>ppp1</inInterface>
        <outInterface>ppp2</outInterface>
        <outLabel>
            <op code="pop"/>
            <op code="swap" value="200"/>
            <op code="push" value="300"/>
        </outLabel>
        <color>200</color>
    </libentry>
</libtable>
\end{XML}

There can be multiple \ttt{<libentry>} elements, each describing a row in the
table. Colums are given as child elements: \ttt{<inLabel>}, \ttt{<inInterface>},
etc. The \ttt{<color>} element is optional, and it only exists to be able to
color LSPs on the GUI. It is not used by the protocols.

\subsection{Ldp}
\label{sec:mpls:ldp}

The \nedtype{Ldp} module implements the Label Distribution Protocol (LDP).
LDP is used to establish LSPs in an MPLS network when traffic engineering is not
required. It establishes LSPs that follow the existing IP routing table, and is
particularly well suited for establishing a full mesh of LSPs between all of the
routers on the network.

LDP relies on the underlying routing information provided by a routing protocol
in order to forward label packets. The router's forwarding information base, or
FIB, is responsible for determining the hop-by-hop path through the network.

In INET, the \nedtype{Ldp} module takes routing information from \nedtype{Ted}
module. The \nedtype{Ted} instance in the network is filled and maintained
by a \nedtype{LinkStateRouting} module. Unfortunately, it is currently not
possible to use other routing protocol implementations such as \nedtype{Ospf}
in conjunction with \nedtype{Ldp}.

When \nedtype{Ldp} is used as signaling protocol, it also serves as ingress
classifier for \nedtype{Mpls}.

\subsection{Ted}
\label{sec:mpls:ted}

The \nedtype{Ted} module contains the Traffic Engineering Database (TED).
In INET, \nedtype{Ted} contains a link state database, including reservations
for each link by RSVP-TE.

\subsection{LinkStateRouting}
\label{sec:mpls:linkstaterouting}

The \nedtype{LinkStateRouting} module provides a simple link state routing
protocol. It uses \nedtype{Ted} as its link state database. Unfortunately, the
\nedtype{LinkStateRouting} module cannot operate independently, it can only be
used inside an MPLS router.

 \subsection{RsvpTe}
\label{sec:mpls:rsvpte}

The \nedtype{RsvpTe} module implements RSVP-TE (Resource Reservation Protocol --
Traffic Engineering), as signaling protocol for MPLS. RSVP-TE handles bandwidth
allocation and allows traffic engineering across an MPLS network. Like LDP, RSVP
uses discovery messages and advertisements to exchange LSP path information
between all hosts. However, whereas LDP is restricted to using the configured
IGP's shortest path as the transit path through the network, RSVP can take
taking into consideration network constraint parameters such as available
bandwidth and explicit hops. RSVP uses a combination of the Constrained Shortest
Path First (CSPF) algorithm and Explicit Route Objects (EROs) to determine how
traffic is routed through the network.

When \nedtype{RsvpTe} is used as signaling protocol, \nedtype{Mpls} needs a
separate ingress classifier module, which is usually a \nedtype{SimpleClassifier}.

The \nedtype{RsvpTe} module allows LSPs to be specified statically in an XML
config file. An example \ttt{traffic.xml} file:

TODO Figure out what stuff means. What is tunnel\_id, what is lspid? (which one is the label?)
which interface of host3 is used as endpoint?

\begin{XML}
<sessions>
    <session>
        <endpoint>host3</endpoint>
        <tunnel_id>1</tunnel_id>
        <paths>
            <path>
                <lspid>100</lspid>
                <bandwidth>100000</bandwidth>
                <route>
                    <node>10.1.1.1</node>
                    <lnode>10.1.2.1</lnode>
                    <node>10.1.4.1</node>
                    <node>10.1.5.1</node>
                </route>
                <permanent>true</permanent>
                <color>100</color>
            </path>
        </paths>
    </session>
</sessions>
\end{XML}

In the route, \ttt{<node>} stands for strict hop, and \ttt{<lnode>} for loose hop.

Paths can also be set up and torn down dynamically with \nedtype{ScenarioManager}
commands (see chapter \ref{cha:scenario-scripting}).
\nedtype{RsvpTe} understands the \ttt{<add-session>} and \ttt{<del-session>}
\nedtype{ScenarioManager} commands. The contents of the \ttt{<add-session>}
element can be the same as the \ttt{<session>} element for the \ttt{traffic.xml}
above. The \ttt{<del-command>} element syntax is also similar, but only
\ttt{<endpoint>}, \ttt{<tunnel\_id>} and \ttt{<lspid>} need to be specified.

The following is an example \ttt{scenario.xml} file:

\begin{XML}
<scenario>
    <at t="2">
        <add-session module="LSR1.rsvp">
            <endpoint>10.2.1.1</endpoint>
            <tunnel_id>1</tunnel_id>
            <paths>
                ...
            </paths>
        </add-session>
    </at>
    <at t="2.4">
        <del-session module="LSR1.rsvp">
            <endpoint>10.2.1.1</endpoint>
            <tunnel_id>1</tunnel_id>
            <paths>
                <path>
                    <lspid>100</lspid>
                </path>
            </paths>
        </del-session>
    </at>
</scenario>
\end{XML}

\section{Classifier}
\label{sec:mpls:classifier}

The \nedtype{RsvpClassifier} module implements an ingress classifier for
\nedtype{Mpls} when using \nedtype{RsvpTe} for signaling. The classifier can be
configured with an XML config file.

\begin{inifile}
**.classifier.config = xmldoc("fectable.xml");
\end{inifile}

An example \ttt{fectable.xml} file:

\begin{XML}
<fectable>
    <fecentry>
        <id>1</id>
        <destination>host5</destination>
        <source>host1</source>
        <tunnel_id>1</tunnel_id>
        <lspid>100</lspid>
    </fecentry>
</fectable>
\end{XML}

TODO figure out what is id, tunnel\_id, lspid!

\section{MPLS-Enabled Router Models}
\label{sec:mpls:mpls-enabled-router-models}

INET provides the following pre-assembled MPLS routers:

\begin{itemize}
  \item \nedtype{LdpMplsRouter} is an MPLS router with the LDP signaling protocol
  \item \nedtype{RsvpMplsRouter} is an MPLS router with the RSVP-TE signaling protocol
\end{itemize}

% HINT: A good MPLS primer:
% "MPLS for Dummies", Richard A Steenbergen <ras@nlayer.net>, nLayer Communications, Inc.
% https:www.nanog.org/meetings/nanog49/presentations/Sunday/mpls-nanog49.pdf


%%% Local Variables:
%%% mode: latex
%%% TeX-master: "usman"
%%% End:

\cleardoublepage

\chapter{Point-to-Point Links}
\label{cha:ppp}


\section{Overview}

The modules of the PPP model can be found in the \nedtype{inet.linklayer.ppp}
package. The \nedtype{Ppp} simple module performs encapsulation
of network datagrams into PPP frames and decapsulation of
the incoming PPP frames.

\section{Sending PPP Frames}

TODO how to send; accepted tags; example code

\section{Receiving PPP Frames}

TODO tags PPP attaches to packets; example code

\section{Extending the PPP Module}

TODO how (override handleMessage etc.)

signals:

Notifications are sent when
transmission of a new PPP frame started (\verb!NF_PP_TX_BEGIN!), finished
(\verb!NF_PP_TX_END!) or when a PPP frame received (\verb!NF_PP_RX_END!).


\cleardoublepage

\chapter{The Ethernet Model}
\label{cha:ethernet}

\section{Sending Ethernet Frames}

TODO tags etc

\section{Receiving Ethernet Frames}

TODO tags etc

\section{Frames}

The INET defines these frames in the \ffilename{EtherFrame.msg} file.
The models supports Ethernet II, 803.2 with LLC header, and 803.3 with LLC and SNAP headers.
The corresponding classes are:
\msgtype{EthernetIIFrame}, \msgtype{EtherFrameWithLlc} and \msgtype{EtherFrameWithSNAP}. They all class
from \msgtype{EtherFrame} which only represents the basic MAC frame with source and
destination addresses. \nedtype{EtherMac} only deals with \msgtype{EtherFrame}'s, and does not
care about the specific subclass.

Ethernet frames carry data packets as encapsulated cMessage objects.
Data packets can be of any message type (cMessage or cMessage subclass).

The model encapsulates data packets in Ethernet frames using the \ttt{encapsulate()}
method of cMessage. Encapsulate() updates the length of the Ethernet frame too,
so the model doesn't have to take care of that.

The fields of the Ethernet header are passed in a \cppclass{Ieee802Ctrl}
control structure to the LLC by the network layer.


EtherJam, EtherPadding (interframe gap), EtherPauseFrame?


\section{EtherLlc}

EtherFrameWithLLC

SAP registration

% TODO delete EtherLLC, because LLC without SNAP is not used with IP (no ARP,IPv6 SAP)
% TODO modify EtherEncap to handle EtherFrameWithSNAP frames too (we can not send EtherFrameWithSNAP now)

\subsubsection{\nedtype{EtherLlc} and higher layers}

The \nedtype{EtherLlc} module can serve several applications (higher layer protocols),
and dispatch data to them. Higher layers are identified by DSAP.
See section "Application registration" for more info.

\nedtype{EtherEncap} doesn't have the functionality to dispatch to different
higher layers because in practice it'll always be used with IP.

\subsubsection{Communication between LLC and Higher Layers}

Higher layers (applications or protocols) talk to the \nedtype{EtherLlc} module.

When a higher layer wants to send a packet via Ethernet, it just
passes the data packet (a cMessage or any subclass) to \nedtype{EtherLlc}.
The message kind has to be set to IEEE802CTRL\_DATA.

In general, if \nedtype{EtherLlc} receives a packet from the higher layers,
it interprets the message kind as a command. The commands include
IEEE802CTRL\_DATA (send a frame), IEEE802CTRL\_REGISTER\_DSAP (register highher layer)
IEEE802CTRL\_DEREGISTER\_DSAP (deregister higher layer) and IEEE802CTRL\_SENDPAUSE
(send PAUSE frame) -- see EtherLLC for a more complete list.

The arguments to the command are NOT inside the data packet but
in a "control info" data structure of class \cppclass{Ieee802Ctrl}, attached to
the packet. See controlInfo() method of cMessage (OMNeT++ 3.0).

For example, to send a packet to a given MAC address and protocol
identifier, the application sets the data packet's message kind
to ETH\_DATA ("please send this data packet" command),
fills in the \nedtype{Ieee802Ctrl} structure with the destination MAC address and
the protocol identifier, adds the control info to the message, then sends
the packet to \nedtype{EtherLlc}.

When the command doesn't involve a data packet (e.g.
IEEE802CTRL\_(DE)REGISTER\_DSAP, IEEE802CTRL\_SENDPAUSE), a dummy packet
(empty cMessage) is used.

\subsubsection{Rationale}

The alternative of the above communications would be:

\begin{itemize}
  \item adding the parameters such as destination address into the data
    packet. This would be a poor solution since it would make the
    higher layers specific to the Ethernet model.
  \item encapsulating a data packet into an \textit{interface packet} which
    contains the destination address and other parameters. The
    disadvantages of this approach is the overhead associated with
    creating and destroying the interface packets.
\end{itemize}

Using a control structure is more efficient than the interface packet
approach, because the control structure can be created once inside
the higher layer and be reused for every packet.

It may also appear to be more intuitive in Tkenv because one can observe
data packets travelling between the higher layer and Ethernet
modules -- as opposed to "interface" packets.


\subsubsection{EtherLLC: SAP Registration}

The Ethernet model supports multiple applications or higher layer
protocols.

So that data arriving from the network can be dispatched to the
correct applications (higher layer protocols), applications
have to register themselves in \nedtype{EtherLlc}. The registration
is done with the IEEE802CTRL\_REGISTER\_DSAP command
(see section "Communication between LLC and higher layers")
which associates a SAP with the LLC port. Different applications
have to connect to different ports of \nedtype{EtherLlc}.

The ETHERCTRL\_REGISTER\_DSAP/IEEE802CTRL\_DEREGISTER\_DSAP commands use only the
dsap field in the \cppclass{Ieee802Ctrl} structure.

\section{EtherMac}

The operation of the MAC module can be schematized by the following state chart:

\begin{center}
\includegraphics{figures/EtherMAC_txstates}
\end{center}

Unlike \nedtype{EtherMacFullDuplex}, this MAC module processes the incoming packets when their
first bit is received. The end of the reception is calculated by the MAC and
detected by scheduling a self message.

When frames collide the transmission is aborted -- in this case the transmitting
station transmits a jam signal. Jam signals are represented
by a \msgtype{EtherJam} message. The jam message contains the tree identifier
of the frame whose transmission is aborted. When the \nedtype{EtherMac} receives a jam
signal, it knows that the corresponding transmission ended in jamming and have
been aborted. Thus when it receives as many jams as collided frames, it can
be sure that the channel is free again. (Receiving a jam message marks the
beginning of the jam signal, so actually has to wait for the duration of the jamming.)

\section{EtherMacFullDuplex}

Outgoing packets are transmitted according to the following state diagram:

\begin{center}
\includegraphics{figures/EtherMACFullDuplex_txstates}
\end{center}

\section{EthernetInterface}

\subsection*{Queueing}

When the transmission line is busy, messages received from the upper layer
needs to be queued.

In routers, MAC relies on an external queue module (see \nedtype{OutputQueue}),
and requests packets from this external queue one-by-one. The name of the
external queue must be given as the \fpar{queueModule} parameer.
There are implementations of \nedtype{OutputQueue} to model finite buffer,
QoS and/or RED.

In hosts, no such queue is used, so MAC contains an internal
queue named \fvar{txQueue} to queue up packets waiting for transmission.
Conceptually, \fvar{txQueue} is of infinite size, but for better diagnostics
one can specify a hard limit in the \fpar{txQueueLimit} parameter -- if this is
exceeded, the simulation stops with an error.

\subsection*{PAUSE handling}
\label{subsec:pause_handling}

The 802.3x standard supports PAUSE frames as a means of flow
control. The frame contains a timer value, expressed as a multiple
of 512 bit-times, that specifies how long the transmitter should
remain quiet. If the receiver becomes uncongested before the
transmitter's pause timer expires, the receiver may elect to send
another PAUSE frame to the transmitter with a timer value of zero,
allowing the transmitter to resume immediately.

\nedtype{EtherMac} will properly respond to PAUSE frames it receives
(\msgtype{EtherPauseFrame} class),
however it will never send a PAUSE frame by itself. (For one thing,
it doesn't have an input buffer that can overflow.)

\nedtype{EtherMac}, however, transmits PAUSE frames received by higher layers,
and \nedtype{EtherLlc} can be instructed by a command to send a PAUSE frame to MAC.

% FIXME PAUSE frames should only be sent on full-duplex ethernet.
%       If a switch uses half-duplex mode to connect to hosts, it can ask sending hosts
%       to slow down their sending rates:
%       - force collisions with incoming frames
%       - make it appear as if the channel is busy
% FIXME PAUSE frames should have 0x8808 in the etherType field

\subsection*{Error handling}

If the MAC is not connected to the network ("cable unplugged"), it will
start up in "disabled" mode. A disabled MAC simply discards any messages
it receives. It is currently not supported to dynamically connect/disconnect
a MAC.

CRC checks are modeled by the \fvar{bitError} flag of the packets. Erronous
packets are dropped by the MAC.


%\subsection*{Auto-Negotiation}
% Ethernet Auto-Negotiation not supported



\cleardoublepage

\chapter{The 802.11 Model}
\label{cha:80211}


\section{Sending and Receiving 802.11 Frames}

TODO tags, etc.

\section{Architecture of the 802.11 MAC Model}

TODO



\cleardoublepage

\chapter{The 802.15.4 Model}
\label{cha:802154}

\section{Overview}
\label{sec:802154:overview}

IEEE 802.15.4 is a technical standard which defines the operation of low-rate
wireless personal area networks (LR-WPANs). IEEE 802.15.4 was designed for data
rates of 250 kbit/s or lower, in order to achieve long battery life (months or
even years) and very low complexity. The standard specifies the physical layer
and media access control.

IEEE 802.15.4 is the basis for the ZigBee, ISA100.11a, WirelessHART, MiWi, SNAP,
and the Thread specifications, each of which further extends the standard by
developing the upper layers which are not defined in IEEE 802.15.4.
Alternatively, it can be used with 6LoWPAN, the technology used to deliver IPv6
over WPANs, to define the upper layers. (Thread is also 6LoWPAN-based.)

% https://en.wikipedia.org/wiki/IEEE_802.15.4
% https://pdfs.semanticscholar.org/0bb3/76e03998fb86c489e4952d5ac3fc898e7ab4.pdf
%   (An Overview of the IEEE 802.15.4a Standard)

The INET Framework contains a basic implementation of IEEE 802.15.4 protocol.


\section{Network Interfaces}
\label{sec:802154:network-interfaces}

There are two network interfaces that differ in the type of radio:

\begin{itemize}
  \item \nedtype{Ieee802154NarrowbandInterface} is for use with narrowband radios
  \item \nedtype{Ieee802154UwbIrInterface} is for use with the UWB-IR radio
\end{itemize}


To create a wireless node with a 802.15.4 interface, use a node type
that has a wireless interface, and set the interface type to the
appropriate type. For example, \nedtype{WirelessHost} is a node type
which is preconfigured to have one wireless interface, \ttt{wlan[0]}.
\ttt{wlan[0]} is of parametric type, so if you build the network from
\nedtype{WirelessHost} nodes, you can configure all of them to use
802.15.4 with the following line in the ini file:

\begin{inifile}
**.wlan[0].typename = "Ieee802154NarrowbandInterface"
\end{inifile}

\section{Physical Layer}
\label{sec:802154:physical-layer}

The IEEE 802.15.4 standard defines several alternative PHYs. There are
several narrowband radios at various frequency bands using various modulation
schemes (DSSS, O-QPSK, MPSK, GFSK BPSK, etc.), a Direct Sequence ultra-wideband
(UWB), and one using chirp spread spectrum (CSS).

INET provides the following radios:

\begin{itemize}
  \item \nedtype{Ieee802154NarrowbandScalarRadio} is currently a partially
    parameterized version of the APSK radio. Before using this radio,
    one must check its parameters and make sure that they correspond to the
    specification of the 802.15.4 narrowband PHY to be simulated.
  \item \nedtype{Ieee802154UwbIrRadio} models the 802.14.5 UWB radio.
\end{itemize}

One must choose a matching medium model, for example
\nedtype{Ieee802154UwbIrRadioMedium} for \nedtype{Ieee802154UwbIrRadio},
and \nedtype{Ieee802154NarrowbandScalarRadioMedium} for
\nedtype{Ieee802154\-NarrowbandScalarRadio}.


\section{MAC Protocol}
\label{sec:802154:mac-protocol}

The 802.15.4 MAC is based on collision avoidance via CSMA/CA. Important other
features include real-time suitability by reservation of guaranteed time slots,
and integrated support for secure communications. Devices also include power
management functions such as link quality and energy detection.

The \nedtype{Ieee802154Mac} type in INET is currently a parameterized
version of a generic CSMA/CA protocol model with ACK support.

There is also a \nedtype{Ieee802154NarrowbandMac}.

TODO why does Ieee802154NarrowbandMac exist?

Ieee802154UwbIrMac -- missing?


%%% Local Variables:
%%% mode: latex
%%% TeX-master: "usman"
%%% End:


\cleardoublepage

\include{ch-sensor-macs}
\cleardoublepage

\ifdraft TODO

\chapter{The Physical Layer}
\label{cha:physicallayer}

TODO which modules? C++ interface

\begin{itemize}
  \item ongoing transmissions
  \item recent successful receptions
  \item recent obstacle intersections and surface normal vectors
\end{itemize}

\begin{verbatim}
TODO: 
 - exploit multiple CPUs and the highly parallel GPU to increase performance
 - provide performance vs. accuracy tradeoff configuration options
   (e.g. range filter, radio mode filter, listening mode filter, MAC address filter)
 - support different level of details (see details below)
 - support different transmitters and receivers (scale from flat to layered models)
 - support different radio signal models (scale from range based to accurate emulation models)
 - support different propagation models (scale from immediate to accurate models)
 - support different attenuation models (scale from free space to trace based models)
 - support different antenna models (scale from isotropic to directional models)
 - support different power consumption models (scale from mode based to signal based models)
 - provide concurrent transmitter and receiver mode (transceiver mode)
 - provide burst mode (back to back) transmissions
 - provide synchronization/preamble detection
 - provide capture during reception (switching to another transmission)
 - provide finite time radio mode switching

TODO: scalar vs dimensional
TODO: flat vs layered
TODO: Generic, IEEE 802.11, IEEE 802.15.4
TODO: acoustic underwater example
TODO: wireless vs. wired medium

\end{verbatim}

\section{Overview}

Today's computing devices use a variety of wireless communication means, such
as Wifi, Bluetooth, NFC, UMTS, and LTE. Despite the diversity of these devices,
there are many commonalities in the way their physical layers can be simulated.
The models often have similar signal representations and signal processing steps,
and they also share the physical medium model where communication takes place.

Simulating the physical layer is generally a computation-intensive task. A
detailed simulation of signal propagation, signal fading, signal
interference, and signal decoding often results in an unacceptable
performance. Finding the right abstractions, the right level of detail, and
the right trade-offs between accuracy and performance is both difficult and
important.

The physical layer model in the INET Framework has been designed with the
following goals in mind:
\begin{itemize}
  \item customizability
  \item extensibility
  \item scalable level of detail
  \item ability to exploit parallel hardware
\end{itemize}

The following sections provide a brief overview of the physical layer model. For
more details on the available modules, their parameterization and the actual
implementations please refer to the documentation in the corresponding NED and
C++ source files.

\subsection{Customizability}

XXX mit jelent a Customizability mint *requirement*? ez a section most azt mondja,
hogy sok mindenfele parameter van a modellben, vagyis Customizability == Lots of Parameters?
ez akar a mondanivalo lenni? Meg leirja hogy van ilyen meg olyan fajta parameter is,
de nem latszik, mit akar ezzel kozolni

XXX kellene egy ilyen: By customizability, we mean that....

Real-world communication devices often provide a wide variety of configuration
options to allow them to be adaped to the physical conditions where they are required
to operate. For example, a Wifi router's administration interface often provides
parameters to configure the transmission power, bitrate, preamble type, carrier
frequency, RTS threshold, beacon interval, etc. Most of those parameters have
sensible default values assigned, so the user is not required to configure them,
but may override them as needed.

Similar to real-world devices, the INET physical layer models also provide
a wide variety of parameters to control their behavior. Many NED parameters
correspond to physical quantities such as transmission power \ttt{[W]},
reception sensitivity \ttt{[W]}, carrier frequency \ttt{[Hz]},
communication range \ttt{[m]}, propagation speed \ttt{[m/s]}, SNIR
reception threshold \ttt{[dB]}, or bitrate \ttt{[b/s]}. Occasionally,
models have more degrees of freedom than the real devices, to allow further
experimentation.

Another important and commonly used parameter kind selects among alternative
implementations of a particular interface by providing its name. Different
implementations are often separate modules, which come with their own set of
parameters to avoid the confusion of mixing their unrelated parameters. Some
modules may be split into more submodules. This further deepens the module
hierarchy, but allows better extensibility.

XXX itt azt akarjuk igazabol mondani, hogy bizonyos koddarabok kicserelhetok?
mert nem a parameter a lenyeg itt, ha jol ertem

\subsection{Extensibility}

Similar to designing other simulation models, modeling the physical layer is
not at all an unambiguous task. For example, the research literature contains a
number of different path loss models for signal propagation, there are different
bit error models for a particular protocol standard, representing the signal in
the analog domain can also be done in several different ways, and so on.

In order to support this diversity, the physical layer model is designed to be
extensible with alternative implementations at various parts of the model. This
is realized by separately defining C++ and NED interfaces between modules, and
also by providing parameters in their parent modules to easily select among the
available implementations.

New models can be added by implementing the required interfaces from scratch, or
by deriving from already existing implementations and overriding functionality.
This architecture allows the user to create new models with less effort, and to
focus on the real differences, while the rest of the physical layer remains the
same.

\subsection{Scalable Level of Detail}

There are many possible ways to model various aspects of the physical layer.
The most important difference lies in the trade-off between performance versus
accuracy. In order to support the different trade-offs the physical layer is
designed to be scalable with respect to the simulated level of detail. In other
words, it is scalable from high-performance less accurate simulations to high
fidelity slower simulations.

The physical layer model is scalable along the following axes:

\begin{itemize}
  \item simulation model -- statistical vs accurate
  \item software architecture -- flat vs layered
  \item data representation -- scalar vs multi-dimensional
  \item number of messages -- TODO 
\end{itemize}

The simulation model might vary from simple statistical models to accurate
emulation. The simplest models ignore the actual bits of the transmission. For
example, the extremely simple unit disc radio even ignores the signal power. The
most accurate models use precise signal representation for all four domains:
bit domain, symbol domain, sample domain, and analog domain representations.
They also emulate most functions of real hardware in detail: forward error
correction, interleaving, scrambling, modulation, spreading, pulse shaping, and
so on.

The software architecture might vary from flat to layered. A flat architecture
is efficient but not modular. Functionality can only be affected through simple
parameters and not by providing alternative implementations. Whereas a layered
architecture is more flexible at the cost of more complex data structures, more
data conversions, more resource management, and thus slower processing. On the
other hand, it provides more customization opportunities to replace parts with
alternative implementations and to do research easier in the area.

The data representation might vary from scalar to multidimensional values. In
the analog domain of the physical layer data quite often changes over time,
frequency, space, or any combination thereof. The most obvious example is the
analog signal power, but there are others such as signal phase or the signal to
noise ratio.

The number of messages per transmission added to the future event queue might
vary from one to the number of radios. One message might be sufficient, for
example, if the transmission is intended to a single destination, and other
receivers are either not affected, or the effect is negligible. On the other
hand, it might be necessary to process all transmissions by all receivers in
order to have the desired effect on the higher layers. For example, if a MAC
model is configured to promiscuous mode, it needs to receive all transmissions.

\subsection{Exploiting Parallel Hardware}

The physical processes simulated by the physical layer are inherently parallel.
The computation of the transmission arrival space-time coordinates, the analog
signal representation of transmissions and receptions, the interfering
receptions and noises, the signal to noise ratio, the decoded bits, the bit
errors, and the physical layer indications all provide a good parallelization
opportunity, because they dominate the physical layer performance and are
independent for each receiver. Therefore the physical layer is designed to be
able to utilize parallel hardware, multi-core CPUs, vector instructions and the
highly parallel GPU.

The idea is to have a central component in the software architecture where
parallel computation can happen. This central component is the medium model
that knows about all radios, transmissions, interferences, and receptions
anyway. It uses optimistic parallel computation in multiple background threads
while the main simulation thread continues normal execution. When a new
transmission enters the channel the already computed and affected results are
invalidated or updated, and the affected ongoing optimistic parallel
computations are canceled.

\section{The Radio Model}

The radio model describes the physical device that is capable of transmitting
and receiving signals on the medium. It contains an antenna model, a transmitter
model, a receiver model, and an energy consumer model. The antenna model is
shared between the transmitter model and the receiver model. The separation of
the transmitter model and the receiver model allows asymmetric configurations.
The energy consumer model is optional and it is only used when the simulation of
energy consumption is necessary.

The radio model has an operational mode that is called the radio mode. The radio
mode is externally controlled usually by the MAC model. In transceiver mode, the
radio can simultaneously transmit and receive a signal. Changing the radio mode
may optionally take a non-zero amount of time. The supported radio modes are the
following:

\begin{itemize}
  \item \textit{off}: communication isn't possible, energy consumption is zero
  \item \textit{sleep}: communication isn't possible, energy consumption is minimal
  \item \textit{receiver}: only reception is possible, energy consumption is low
  \item \textit{transmitter}: only transmission is possible, energy consumption is
high
  \item \textit{transceiver}: reception and transmission is simultaneously
possible, energy consumption is high
  \item \textit{switching}: communication isn't possible, energy consumption is
minimal
\end{itemize}

In addition to the radio mode, the transmitter and the receiver models have
separate states which describe what they are doing. Changes to these states are
automatically published by the radio. The signaled transmitter states are the
following:

\begin{itemize}
  \item \textit{undefined}: not in operation
  \item \textit{idle}: no transmission in progress
  \item \textit{transmitting}: a transmission is in progress
\end{itemize}

The signaled receiver states are the following:

\begin{itemize}
  \item \textit{undefined}: not in operation
  \item \textit{idle}: no reception in progress
  \item \textit{busy}: received signal is not interpretable
  \item \textit{synchronizing}: synchronization is in progress
  \item \textit{receiving}: reception is in progress
\end{itemize}

When a radio wants to transmit a signal on the medium, it sends direct messages
to all affected radios with the help of the central medium module. The messages
contain a shared data structure which describes the transmission the way it
entered the medium. The messages arrive at the moment when start of the
transmission arrive at the receiver. The receiver radios also handle the
incoming messages with the help of the central medium module. This kind of
centralization allows the medium to do shared computations in a more efficient
way and it also makes parallel computation possible.

To maintain modularity, the radio module delegates many of is functions to submodules.
The following sections describe the parts of the radio model.

\subsection{Antenna Models}

The antenna model describes the effects of the physical device which converts
electric signals into radio waves, and vice versa. This model captures the
antenna characteristics that heavily affect the quality of the communication
channel. For example, various antenna shapes, antenna size and geometry, antenna
arrays, and antenna orientation causes different directional or frequency
selectivity.

The antenna model provides a position and an orientation using a mobility model
that defaults to the mobility of the node. The main purpose of this model is to
compute the antenna gain based on the specific antenna characteristics and the
direction of the signal. The signal direction is computed by the medium from the
position and the orientation of the transmitter and the receiver. The following
list provides some examples:

\begin{itemize}
  \item \nedtype{IsotropicAntenna}: antenna gain is exactly 1 in any direction
  \item \nedtype{ConstantGainAntenna}: antenna gain is a constant determined by
a parameter
  \item \nedtype{DipoleAntenna}: antenna gain depends on the direction according
to the dipole antenna characteristics
  \item \nedtype{InterpolatingAntenna}: antenna gain is computed by linear
interpolation according to a table indexed by the direction angles
\end{itemize}

The antenna models are in the \ttt{src/physicallayer/antenna/} directory.

\subsection{Transmitter Models}

The transmitter model describes the physical process which converts packets into
electric signals. In other words, this model converts a MAC packet into a signal
that is transmitted on the medium. The conversion process and the representation
of the signal depends on the level of detail and the physical characteristics
of the implemented protocol.

In the flat model the transmitter model skips the symbol domain and the sample
domain representations, and it directly creates the analog domain representation.
The bit domain representation is reduced to the bit length of the packet and the
actual bits are ignored.

In the layered model the conversion process involves various processing steps
such as packet serialization, forward error correction encoding, scrambling,
interleaving, and modulation. This transmitter model requires much more
computation, but it produces accurate bit domain, symbol domain, and sample
domain representations.

The various protocol specific transmitter models are in the corresponding
directories.

\subsection{Receiver Models}

The receiver model describes the physical process which converts electric
signals into packets. In other words, this model converts a reception, along
with an interference computed by the medium model, into a MAC packet and a
reception indication. It also determines the following for each transmission:

\begin{itemize}
  \item \textit{is the reception possible or not}: based on the signal
characteristics such as reception power, carrier frequency, bandwidth, preamble
mode, modulation scheme
  \item \textit{if the reception is possible, is reception attempted or not}: based
on the ongoing reception and the support of signal capturing
  \item \textit{if the reception is attempted, is reception successful or not}:
based on the error model and the simulated part of the signal decoding
\end{itemize}

In the flat model the receiver model skips the sample domain, the symbol domain,
and the bit domain representations, and it directly creates the packet domain
representation by copying the packet from the transmission. It uses the error
model to decide if the reception is successful or not.

In the layered model the conversion process involves various processing steps
such as demodulation, descrambling, deinterleaving, forward error correction
decoding, and deserialization. This reception model requires much more
computation, but it produces accurate sample domain, symbol domain, and bit
domain representations.

The various protocol specific receiver models are in the corresponding
directories.

\subsection{Transmission Error Modeling}

Determining the reception errors is a crucial part of the reception process.
There are often several different statistical error models in the literature
even for a particular physical layer. In order to support this diversity the
error model is a separate replaceable component of the receiver.

The error model describes how the signal to noise ratio affects the amount of
errors at the receiver. The main purpose of this model is to determine whether
if the received packet has errors or not. It also computes various physical
layer indications for higher layers such as packet error rate, bit error rate,
and symbol error rate. For the layered reception model it needs to compute the
erroneous bits, symbols, or samples depending on the lowest simulated physical
domain where the real decoding starts. The error model is optional, if omitted
all receptions are considered successful.

The error models are in the \ttt{src/physicallayer/errormodel/} directory and
also in the corresponding protocol specific directories.

\subsection{Power Consumption Models}

A substantial part of the energy consumption of communication devices comes from
transmitting and receiving signals. The energy consumer model describes how the
radio consumes energy depending on its activity. This model is optional, if
omitted energy consumption is ignored. The following list provides some examples:

\begin{itemize}
  \item \nedtype{StateBasedEnergyConsumer}: the constant power consumption is
determined by valid combinations of the radio mode, the transmitter state and
the receiver state
\end{itemize}

The energy consumer models are in the \ttt{src/physicallayer/energyconsumer/} directory.

TODO: layered

This module further splits the transmitter and receiver models to allow bit
precise communication modeling.

TODO: layered

The following sections describe the parts of the layered radio model.

\subsubsection{Encoding and Decoding}

This module describes how the packet domain signal representation is converted
into the bit domain, and vice versa.

TODO: layered

\subsubsection{Modulation and Demodulation}

This module describes how the bit domain signal representation is converted into
the symbol domain, and vice versa.

TODO: layered

\subsubsection{Pulse Shaping and Pulse Filtering}

This module describes how the symbol domain signal representation is converted
into the sample domain, and vice versa.

TODO: layered


\subsubsection{Digital Analog and Analog Digital Conversion}

This module describes how the sample domain signal representation is converted
into the analog domain, and vice versa.

TODO: layered

\section{The Medium Model}

The medium model describes the shared physical medium where communication takes
place. It keeps track of radios, noise sources, ongoing transmissions,
background noise, and other ongoing noises. The medium computes when, where and
how transmissions and noises arrive at receivers. It also efficiently provides
the set of interfering transmissions and noises for the receivers. It doesn't
send or handle messages on its own, it rather acts as a mediator between radios.

<<<<<<< 3e2116b1e231e67a425b791a6b3bf6627e5e442f:doc/src/developers-guide/ch-physicallayer.tex
The medium model has a separate chapter devoted to it, see \ref{cha:transmission-medium}. 
=======
The medium model provides a couple of parameters to optimize for performance.
For example, the filter parameters control how the model determines the set of
affected radios when a new transmission enters the medium. The medium model
maintains the following global limits among the registered radios to support
various optimizations:

\begin{itemize}
  \item maximum speed
  \item maximum antenna gain
  \item maximum transmission power
  \item minimum interference power
  \item minimum reception power
  \item minimum interference time
  \item maximum transmission duration
  \item maximum detection range
  \item maximum interference range
  \item maximum communication range
  \item minimum and maximum mobility constraints
\end{itemize}

The medium module utilizes multiple submodules to further split its task. This
design makes it extensible and customizable. The following sections describe
the parts of the medium.

\subsection{Propagation Models}

Whenever the radio transmits a signal, it starts to propagate through space. The
transmitter might move during the transmission interval, and the receiver might
also move during both the propagation time and the reception interval. The
effect of these movements becomes more and more important as the propagation
speed and the speed of the radios gets closer to each other. This is especially
true for acoustic communication because the propagation speed of the signal is
much smaller. In general, it is difficult to accurately compute the signal
arrival, but it is usually not really necessary, some kind of approximation
suffices.

The propagation model is a separate component of the medium model to make it
extensible with alternative implementations. This model describes how signals
propagate through space over time. The main purpose of this model is to compute
the arrival space-time coordinates for transmissions. The following list
provides some examples:

\begin{itemize}
  \item \nedtype{ConstantTimePropagation}: propagation time is independent of
the traveled distance and it is determined by a constant parameter
  \item \nedtype{ConstantSpeedPropagation}: propagation time is proportional to
the traveled distance determined by a constant propagation speed parameter
\ifdraft
TODO: parallel
  \item \nedtype{ConstantSpeedGPUPropagation}: propagation time is computed in
parallel on the GPU for all receivers
\fi
\end{itemize}

The propagation models are in the \ttt{src/physicallayer/propagation/} directory.

\subsection{Path Loss Models}

As the signal propagates through space its power density decreases. Path loss
might be due to the combination of many effects, such as free-space loss,
refraction, diffraction, reflection, and absorption. There are several different
models in the literature, which differ in their parameterization and application
area.

The path loss model is a separate component of the medium model to make it
extensible with alternative implementations. This model describes the reduction
of power as the signal propagates through space. It computes the power loss
factor based on the traveled distance, the signal frequency and the propagation
speed. It may also provide the opposite, that is the traveled distance based on
the power loss factor, the signal frequency and the propagation speed. The
latter computation is useful for determining the maximum communication range
based on the transmission power and the reception sensitvity. The following list
provides some examples:

\begin{itemize}
  \item \nedtype{FreeSpacePathLoss}
  \item \nedtype{LogNormalShadowing}
  \item \nedtype{TwoRayGroundReflection}
  \item \nedtype{BreakpointPathLoss}
  \item \nedtype{NakagamiFading}
  \item \nedtype{RayleighFading}
  \item \nedtype{RicianFading}
  \item \nedtype{SUIPathLoss}
  \item \nedtype{UWBIRStochasticPathLoss}
  \item etc.
\end{itemize}

The path loss models are in the \ttt{src/physicallayer/pathloss/} directory.

\subsection{Obstacle Loss Models}

When the signal propagates through space it also passes through physical objects
present in that space. As the signal passes through, its power decreases when it
reflects from the surfaces of physical objects, and also when it is absorbed by
their material. There are various ways to model this effect, which differ in the
trade-off between accuracy and performance.

The obstacle loss model is a separate component of the medium model to make it
extensible with alternative implementations. This model describes the reduction
of power as the signal passes through physical objects. The main purpose of this
model is to compute the power loss factor based on the traveled straight path,
the signal frequency and the physical properties of the obstructing physical
objects. The obstacle model utilizes the physical environment model to query the
obstructing physical objects. The following list provides some examples:

\begin{itemize}
  \item \nedtype{TracingObstacleLoss}: the power loss is based on computing the
accurate dielectric and reflection loss along the straight path considering the
shape, the position, the orientation, and the material of obstructing physical
objects
\end{itemize}

The obstacle loss models are in the \ttt{src/physicallayer/obstacleloss/}
directory.

\ifdraft
TODO: multipath
\subsection{Multipath Models}

The signal reflects from the surfaces, it refracts as it passes through the
surfaces, it diffracts as it passes around the objects.

The multipath model describes the alternative paths that the signal travels
before it either reaches the receiver or sufficiently fades.
\fi

\subsection{Background Noise Models}

The thermal noise, the cosmic background noise, and other random fluctuations of
the electromagnetic field affect the quality of the communication channel. This
kind of noise doesn't come from a particular source and it doesn't propagate
through space.

The background noise model is a separate component of the medium model to make
it extensible with alternative implementations. This model describes how the
background noise changes over space and time. The main purpose of this model is
to compute the analog representation of the noise signal for a given space-time
interval.

\begin{itemize}
  \item \nedtype{IsotropicBackgroundNoise}: the background noise is independent
of the time and the position, and its power is determined by a constant parameter
\end{itemize}

The background noise models are in the \ttt{src/physicallayer/backgroundnoise/}
directory.

\subsection{Neighbor Cache Models}

Transceivers are considered neighbors if successful communication is possible
between them. In wired communication systems it is mostly quite obvious which
transceivers are neighbors, because they are connected by wires. In contrast,
in wireless communication systems determining which radios are neighbors isn't
obvious at all.

The neighbor cache model is a separate component of the medium model to make it
extensible with alternative implementations. This model provides an efficient
way of keeping track of the neighbor relationship between radios. The main
purpose of this model is to compute the affected set of receivers on the medium
for a given transmission. The model also has to follow the movement of radios,
but it might provide conservative approximations for queries. The following list
provides some examples:

\begin{itemize}
<<<<<<< c77084e2adfb092997e906f45d0021c53f02b14d
  \item \nedtype{NeighborListNeighborCache}: maintains a separate periodically
updated  neighbor list for each radio
=======
  \item \nedtype{ListNeighborCache}: maintains a separate periodically updated
neighbor list for each radio
>>>>>>> phy /1
  \item \nedtype{GridNeighborCache}: organizes radios in a 3 dimensional grid
with constant cell size and updates periodically
  \item \nedtype{QuadTreeNeighborCache}: organizes radios in a 2 dimensional
quad tree (ignoring the Z axis) with constant node size and updates periodically
\end{itemize}

The neighbor cache models are in the \ttt{src/physicallayer/neighborcache/}
directory.
>>>>>>> phy /1:doc/src/manual/ch-physicallayer.tex

\section{Signal Representation}

The data structures that represent the transmitted and the received signals
might contain many different data depending on the simulated level of detail. In
addition, the reception data structure might contain various physical layer
indications, which are computed during the reception process. The following list
provides some examples:

\begin{itemize}
  \item \textit{packet domain}: actual packet, packet error rate, packet error bit,
etc.
  \item \textit{bit domain}: various bit lengths, bitrates, actual bits, forward
error correction code, interleaving scheme, scrambling scheme, bit error rate,
number of bit errors, actual erroneous bits, etc.
  \item \textit{symbol domain}: number of symbols, symbol rate, actual symbols,
modulation scheme, symbol error rate, number of symbol errors, actual erroneous
symbols, etc.
  \item \textit{sample domain}: number of samples, sampling rate, actual samples,
etc.
  \item \textit{analog domain}: space-time coordinates, antenna orientations,
communication range, interference range, detection range, carrier frequency,
subcarrier frequencies, bandwidths, scalar or dimensional power, receive signal
strength indication, signal to noise and interference ratio, etc.
\end{itemize}

In simple case the packet domain specifies the MAC packet only, and the bit
domain specifies the bit length and the bitrate. The symbol domain specifies the
used modulation, and the sample domain is simply ignored. The most important
part is the analog domain representation, because it is indispensable to be able
to compute some kind of signal to noise and interference ratio. The following
figure shows four different kinds of analog domain representations, but other
representations are also possible.

\begin{figure}[h!]
\centering
\includegraphics[width=\textwidth]{figures/phyanalog}
\caption{Various analog signal representations}
\end{figure}

The first representation is called range-based, and it is used by the unit disc
radio. The advantage of this data structure is that it is compact, predictable,
and provides high performance. The disadvantage is that it is very inaccurate in
terms of modeling reality. Nevertheless, this representation might be sufficient
for developing a new routing protocol if accurate simulation of packet loss is
not important.

The second data structure represents a narrowband signal with a scalar signal
power, a carrier frequency, and a bandwidth. The advantage of this
representation is that it allows to compute a real signal to noise ratio, which
in turn can be used by the error model to compute bit and packet error rates.
This representation is most of the time sufficient for the simulation of IEEE
802.11 networks.

The third data structure describes a signal power that changes over time. In
this case the signal power is represented with a one-dimensional time dependent
value that precisely follows the transmitted pulses. This representation is used
by the IEEE 802.15.4a UWB radio.

The last representation uses a multi-dimensional value to describe the signal
power that changes over both time and frequency. The IEEE 802.11b model might
use this representation to simulate crosstalk, where one channel interferes with
another. In order to make it work the frequency spectrum of the signal has to
follow the real spectrum more precisely at both ends of the band.

The flat signal representation uses a single object to simulatenously describe
all domains of the transmission or the reception. In contrast, the layered
signal representation uses one object to describe every domain seperately. The
advantage of the latter is that it is extensible with alternative implementations
for each domain. The disadvantage is that it needs more allocation and resource
management.

\section{Signal Processing}

The following figure shows the process of how a MAC packet gets from the
transmitter radio through the medium to the receiver radio. The figure focues on
how data flows between the processing components of the physical layer. The blue
boxes represent the data structures, and the red boxes represent the processing
components.

\begin{figure}[h!]
\centering
\includegraphics[width=\textwidth]{figures/phydataflow}
\caption{Signal processing data flow}
\end{figure}

The transmission process starts in the transmitter radio when it receives a MAC
packet from the higher layer. The radio must be in transmitter or transceiver
mode before receiving a MAC packet, otherwise it throws an exception. At first
the transmitter model creates a data structure that describes the transmitted
signal based on the received MAC packet and the attached transmission request.
The resulting data structure is immutable, it is not going to be changed in any
later processing step.

Thereafter the propagation model computes the arrival space-time coordinates for
all receivers. In the next step the medium model determines the set of affected
receivers. Which radio constitutes affected depends on a number of factors such
as the maximum communication range of the transmitter, the radio mode of the
receiver, the listening mode of the receiver, or potentially the MAC address of
the receiver. Using the result the medium model sends a separate message with
the shared transmission data structure to all affected receivers. There's no
need to send a message to all radios on the channel, because the computation
of interfering signals is independent of this step.

Thereafter the attenuation model computes the reception for the receiver using
the original transmission and the arrival data structure. It applies the path
loss model, the obstacle loss model and the multipath model to the transmission.
The resulting data structure is also immutable, it is not going to be changed in
any later processing step.

Thereafter the medium model computes the interference for the reception by
collecting all interfering receptions and noises. Another signal is considered
interfering if it owerlaps both in time and frequency domains with respect to
the minimum interference parameters. The background noise model also computes a
noise signal that is added to the interference.

The reception process starts in the receiver radio when it receives a message
from the transmitter radio. The radio must be in receiver or transceiver mode
before the message arrives, otherwise it ignores the message. At first the
receiver model determines is whether the reception is actually attempted or not.
This decision depends on the reception power, whether there's another ongoing
reception process, and capturing is enabled or not.

Thereafter the receiver model computes the signal to noise and interference
ratio from the reception and the interference. Using the result, the bitrate,
and the modulation scheme the error model computes the necessary error rates.
Alternatively the error model might compute the erroneous bits, or symbols by
altering the corresponding data of the original transmission.

Thereafter the receiver determines the received MAC packet by either simply
reusing the original, or actually decoding from the lowest represented domain
in the reception. Finally, it attaches the physical layer indication to the MAC
packet, and sends it up to the higher layer.

The following sections describe the data structures that are created during
signal processing.

\subsubsection{Transmission Request}

This data structure contains parameters that control how the transmitter
produces the transmission. For example, it might override the default
transmission power, ot the default bitrate of the transmitter. It is attached as
a control info object to the MAC packet sent down from the MAC module to the
radio.

\subsubsection{Transmission}

This data structure describes the transmission of a signal. It specifies the
start/end time, start/end antenna position, start/end antenna orientation of the
transmitter. In other words, it describes when, where and how the signal
started/ended to interact with the medium. The transmitter model creates one
transmission instance per MAC packet.

\subsubsection{Arrival}

This data structure decscirbes the space and time coordinates of a transmission
arriving at a particular receiver. It specifies the start/end time, start/end
antenna position, start/end antenna orientation of the receiver. The propagation
model creates one arrival instance per transmission per receiver.

\subsubsection{Listening}

This data structure describes the way the receiver radio is listening on the
medium. The physical layer ignores certain transmissions either during computing
the interference or even the complete reception of such transmissions. For
example, a narrowband listening specifies a carrier frequency and a bandwidth.

\subsubsection{Reception}

This data structure describes the reception of a signal by a particular receiver.
It specifies at least the start/end time, start/end antenna position, start/end
antenna orientation of the receiver. The attenuation model creates one reception
instance per transmission per receiver.

\subsubsection{Noise}

This data structure describes a meaningless signal or a meaningless composition
of multiple signals. All models contain at least the start/end time, and
start/end position.

\subsubsection{Interference}

This data structure describes the interfering signals and noises that affect a
particular reception. It also specifies the total noise that is the composition
of all interference.

\subsubsection{SNIR}

This data structure describes the signal to noise and interference ratio of a
particular reception. It also specifies the minimum signal to noise and
interference ratio.

\subsubsection{Reception Decision}

This data structure describes whether if the reception of a signal is possible
or not, is attempted or not, and is successful or not.

\subsubsection{Reception Indication}

This data structure describes the physical layer indications such as RSSI, SNIR,
PER, BER, SER. These physical properties are optional and may be omitted if the
receiver is configured to do so or if it doesn't support providing the data. The
reception indication is attached as a control info object to the MAC packet sent
up from the radio to the MAC module.

\section{Visualization}

In order to help understanding the communication in the network the physical
layer supports visualizing its state. The following list shows what can be
displayed:

\begin{itemize}
  \item ongoing transmissions
  \item recent successful receptions
  \item recent obstacle intersections and surface normal vectors
\end{itemize}

The ongoing transmissions can be displayed with 3 dimensional spheres or with 2
dimensional rings laying in the XY plane. As the signal propagates through space
the figure grows with it to show where the beginning of the signal is. The inner
circle of the ring figure shows as the end of the signal propagates through
space.

The recent successful receptions are displayed as straight lines between the
original positions of the transmission and the reception. The recent obstacle
intersections are also displayed as straight lines from the start of the
intersection to the end of it.

<<<<<<< 3e2116b1e231e67a425b791a6b3bf6627e5e442f:doc/src/developers-guide/ch-physicallayer.tex
\section{TODO other stuff}
=======
<<<<<<< c77084e2adfb092997e906f45d0021c53f02b14d
\ifdraft
TODO: 
=======
\iffalse
TODO:
>>>>>>> phy /1

 - exploit multiple CPUs and the highly parallel GPU to increase performance
 - provide performance vs. accuracy tradeoff configuration options
   (e.g. range filter, radio mode filter, listening mode filter, MAC address filter)
 - support different level of details (see details below)
 - support different transmitters and receivers (scale from flat to layered models)
 - support different radio signal models (scale from range based to accurate emulation models)
 - support different propagation models (scale from immediate to accurate models)
 - support different attenuation models (scale from free space to trace based models)
 - support different antenna models (scale from isotropic to directional models)
 - support different power consumption models (scale from mode based to signal based models)
 - provide concurrent transmitter and receiver mode (transceiver mode)
 - provide burst mode (back to back) transmissions
 - provide synchronization/preamble detection
 - provide capture during reception (switching to another transmission)
 - provide finite time radio mode switching
>>>>>>> phy /1:doc/src/manual/ch-physicallayer.tex

TODO: scalar vs dimensional

TODO: flat vs layered

TODO: Generic, IEEE 802.11, IEEE 802.15.4

TODO: acoustic underwater example

TODO: wireless vs. wired medium

\section{Use Cases}

\fi


\cleardoublepage

\chapter{The Transmission Medium}
\label{cha:transmission-medium}

\section{Overview}
\label{sec:medium:overview}

For wireless communication, an additional module is required to model the
shared physical medium where the communication takes place. This module
keeps track of transceivers, noise sources, ongoing transmissions,
background noise, and other ongoing noises.

It relies on several models:

\begin{enumerate}
  \item signal propagation model
  \item path loss model
  \item obstacle loss model
  \item background noise model
  \item signal analog model
\end{enumerate}

With the help of the above models, the medium module computes
when, where, and how signals arrive at receivers, including
the set of interfering signals and noises. In addition,
the medium module also contains various mechanisms and ways
to improve the scalability of wireless network simulations.

\section{RadioMedium}
\label{sec:medium:radiomedium}

The standard transmission medium model in INET is \nedtype{RadioMedium}.
\nedtype{RadioMedium} is as an OMNeT++ compound module with
several replaceable submodules. It contains submodules for
each of the above models (signal propagation, path loss, etc.),
and various caches for efficiency.

Note that \nedtype{RadioMedium} is an active compound module, that is,
it has an associated C++ class that encapsulates the computations.

\nedtype{RadioMedium} contains its components as submodules
with parametric types:

\begin{ned}
propagation: <propagationType> like IPropagation;
analogModel: <analogModelType> like IAnalogModel;
backgroundNoise: <backgroundNoiseType> like IRadioBackgroundNoise
    if backgroundNoiseType != "";
pathLoss: <pathLossType> like IPathLoss;
obstacleLoss: <obstacleLossType> like IObstacleLoss
    if obstacleLossType != "";
mediumLimitCache: <mediumLimitCacheType> like IMediumLimitCache;
communicationCache: <communicationCacheType> like ICommunicationCache;
neighborCache: <neighborCacheType> like INeighborCache
    if neighborCacheType != "";
\end{ned}

There are many preconfigured versions of \nedtype{RadioMedium}:

\begin{itemize}
  \item For use with \nedtype{UnitDiskRadio}: \nedtype{UnitDiskRadioMedium}
  \item For APSK radios: \nedtype{ApskScalarRadioMedium}, \nedtype{ApskDimensionalRadioMedium},
    \nedtype{ApskLayeredScalarRadioMedium}, \nedtype{ApskLayeredDimensionalRadioMedium},
  \item For IEEE 802.11: \nedtype{Ieee80211ScalarRadioMedium}, \nedtype{Ieee80211DimensionalRadioMedium},
    \nedtype{Ieee80211LayeredScalarRadioMedium}, \nedtype{Ieee80211LayeredDimensionalRadioMedium},
  \item For IEEE 802.15.4: \nedtype{Ieee802154UwbIrRadioMedium}, \nedtype{Ieee802154NarrowbandScalar\-RadioMedium}
\end{itemize}

The following sections describe the parts of the medium model.

\section{Propagation Models}
\label{sec:medium:propagation-models}

When a transmitter starts to transmit a signal, the beginning of the signal
propagates through the transmission medium. When the transmitter ends the
transmission, the signal's end propagates similarly. The propagation model
describes how a signal moves through space over time. Its main purpose is
to compute the arrival space-time coordinates at receivers. There are two
built-in models in INET, implemented as simple modules:

\begin{itemize}
        \item \nedtype{ConstantTimePropagation} is a simplistic model where the propagation time is independent of the traveled distance. The propagation time is simply determined by a module parameter.
        \item \nedtype{ConstantSpeedPropagation} is a more realistic model where the propagation time is proportional to the traveled distance. The propagation time is independent of the transmitter and receiver movement during both signal transmission and propagation. The propagation speed is determined by a module parameter.
\end{itemize}

The default propagation model is configured as follows:

\inisnippet{PropagationModelConfigurationExample}{Propagation model configuration example}

A more accurate model could take into consideration the transmitter and
receiver movement. This effect becomes especially important for acoustic
communication, because the propagation speed of the signal is much more
comparable to the speed of the transceivers.

\section{Path Loss Models}
\label{sec:medium:path-loss-models}

As a signal propagates through space its power density decreases. This is
called path loss and it is the combination of many effects such as
free-space loss, refraction, diffraction, reflection, and absorption. There
are several different path loss models in the literature, which differ in
their parameterization and application area.

In INET, a path loss model is an OMNeT++ simple module implementing a
specific path loss algorithm. Its main purpose is to compute the power loss
for a given signal, but it is also capable of estimating the range for a
given loss. The latter is useful, for example, to allow visualizing
communication range. INET contains a number of built-in path loss
algorithms, each comes with its own set of parameters:

\begin{itemize}
        \item \nedtype{FreeSpacePathLoss} models line of sight path loss for air or vacuum.
        \item \nedtype{BreakpointPathLoss} refines it using dual slope model with two separate path loss exponents.
        \item \nedtype{LogNormalShadowing} models path loss for a wide range of environments (e.g. urban areas, and buildings)
        \item \nedtype{TwoRayGroundReflection} models interference between line of sight and single ground reflection.
        \item \nedtype{TwoRayInterference} refines the above for inter-vechicle communication.
        \item \nedtype{RicianFading} is a stochastical model for the anomaly caused by partial cancellation of a signal by itself.
        \item \nedtype{RayleighFading} is a stochastical model for heavily built-up urban environments when there is no dominant propagation along the line of sight.
        \item \nedtype{NakagamiFading} further refines the above two models for cellular systems.
\end{itemize}

The following example replaces the default free-space path loss model with
log normal shadowing:

\inisnippet{PathLossConfigurationExample}{Path loss configuration example}

\section{Obstacle Loss Models}
\label{sec:medium:obstacle-loss-models}

When the signal propagates through space it also passes through physical
objects present in that space. As the signal penetrates physical objects,
its power decreases when it reflects from surfaces, and also when it is
absorbed by their material. There are various ways to model this effect,
which differ in the trade-off between accuracy and performance.

In INET, an obstacle loss model is an OMNeT++ simple module. Its main
purpose is to compute the power loss based on the traveled path and the
signal frequency. The obstacle loss models most often use the physical
environment model to determine the set of penetrated physical objects.
INET contains a few built-in obstacle loss models:

\begin{itemize}
        \item \nedtype{IdealObstacleLoss} model determines total or no power loss at all by checking if there is any obstructing physical object along the straight propagation path.
        \item \nedtype{DielectricObstacleLoss} computes the power loss based on the accurate dielectric and reflection loss along the straight path considering the shape, position, orientation, and material of obstructing physical objects.
\end{itemize}

By default, the medium module doesn't contain any obstacle loss model, but
configuring one is very simple:

\inisnippet{ObstacleLossModelConfigurationExample}{Obstacle loss model configuration example}

Statistical obstacle loss models are also possible but currently not provided.

\section{Background Noise Models}
\label{sec:medium:background-noise-models}

Thermal noise, cosmic background noise, and other random fluctuations of
the electromagnetic field affect the quality of the communication channel.
This kind of noise doesn't come from a particular source, so it doesn't
make sense to model its propagation through space. The background noise
model describes instead how it changes over space and time.

In INET, a background noise model is an OMNeT++ simple module. Its main
purpose is to compute the analog representation of the background noise for
a given space-time interval. For example,
\nedtype{IsotropicScalarBackgroundNoise} computes a background noise that is
independent of space-time coordinates, and its scalar power is determined
by a module parameter.

The simplest background noise model can be configured as follows:

\inisnippet{BackgroundNoiseModelConfigurationExample}{Background noise model configuration example}

\section{Analog Models}
\label{sec:medium:analog-models}

The analog signal is a complex physical phenomenon which can be modeled in
many different ways. Choosing the right analog domain signal representation
is the most important factor in the trade-off between accuracy and
performance. The analog model of the transmission medium determines how
signals are represented while being transmitted, propagated, and received.

In INET, an analog model is an OMNeT++ simple module. Its main purpose is
to compute the received signal from the transmitted signal. The analog
model combines the effect of the antenna, path loss, and obstacle loss
models. Transceivers must be configured transmit and receive signals
according to the representation used by the analog model.

The most commonly used analog model, which uses a scalar signal power
representation over a frequency and time interval, can be configured as
follows:

\inisnippet{AnalogModelConfigurationExample}{Analog model configuration example}

\section{Neighbor Cache}
\label{sec:medium:neighbor-cache}

Transceivers are considered neighbors if successful communication is
possible between them. For wired communication it is easy to determine
which transceivers are neighbors, because they are connected by wires. In
contrast, in wireless communication determining which transceivers are
neighbors isn't obvious at all.

TODO: it's about *notification*, i.e. giving the radio a chance
to react. Signals are counted as interference, regardless of being neighbor or not!

TODO: if no cache, all receivers will be notified

TODO: query always happens with a *radius*!  ``which radios are in your  X-meter proximity''

In INET, a neighbor cache is an OMNeT++ simple module which provides
an efficient way of keeping track of the neighbor relationship between
transceivers. Its main purpose is to compute the set of affected receivers
for a given transmission. All built-in models in INET provide a
conservative approximation only, because they update their state
periodically:

\begin{itemize}
  \item \nedtype{NeighborListNeighborCache} takes a range as parameter,
    and for each transceiver it maintains the list of receivers within
    range (\textit{neighbor list}).
  \item \nedtype{GridNeighborCache} organizes transceivers in a 3D grid with
    constant cell size.
  \item \nedtype{QuadTreeNeighborCache} organizes transceivers in a 2D quad tree
    (ignoring the Z axis) with constant node size.
\end{itemize}

The following example sets \nedtype{QuadTreeNeighborCache} as neighbor cache:

\inisnippet{NeighborCacheModelConfigurationExample}{Neighbor cache model configuration example}

How should one decide which neighbor cache to choose for a given simulation?
As the sole purpose of the neighbor cache is to speed up the simulation,
one should choose the one that leads to the best performance for that particular
network. Which one performs best is best determined by experimentation, as it
depends on many factors: number of nodes, their spatial distribution, their
speed and movement pattern, their communication pattern, and so on.
Note that not only the choice of neighbor cache but also its parameterization
can affect performance.


\section{Medium Limit Cache}
\label{sec:medium:medium-limit-cache}

The medium limit cache (and its default implementation \nedtype{MediumLimitCache})
keeps track of certain thresholds and minimum/maximum values of quantities
related to layer 1 modeling. Some of these limits can be gathered from other
modules in the network, but still, all of them can be explicitly specified by the user.
The quantities include:

\begin{itemize}
    \item maximum speed (can be gathered from mobility models)
    \item maximum transmission power
    \item minimum interference power and reception power
    \item maximum antenna gain (can be computed from antenna models)
    \item minimum time interval to consider two overlapping signals interfering
    \item maximum duration of a transmission
    \item maximum communication range and interference range
      (can be computed from transmitter and receiver models)
\end{itemize}

These limits allow the transmission medium model to make assumptions about the
locations of nodes (i.e. the maximum distance they can move during some
interval), about the possibility of interference, and about the possibility
of a signal being receivable.


\section{Communication Cache}
\label{sec:medium:communication-cache}

The communication cache is used to cache various intermediate computation
results related to the communication on the medium. The main motivation to have
multiple implementations is that different implementations may be the most
efficient in different simulations. Also, a conservative (simple but robust)
implementation may be used for validating new (more efficient but also more
complex) implementations.

Implementations include:

\begin{itemize}
  \item \nedtype{ReferenceCommunicationCache}
  \item \nedtype{MapCommunicationCache}
  \item \nedtype{VectorCommunicationCache}
\end{itemize}


\section{Improving Scalability}
\label{sec:medium:improving-scalability}

The simulation of wireless networks is inherently less scalable than
that of wired networks. In wired networks, a transmission only affects
the host's neighbors on the link, which is usually 1 in modern networks
that are dominated by point-to-point links. The wireless medium, however,
is a broadcast medium. Any transmission is ``heard'' by all nodes
within interference range, not only the intended recipients.
The signal may be receivable by them (and must be indeeded received
before the destination address field in it can be examined),
or may interfere with the reception of other transmissions.
Whichever the case, the transmission must be evaluated or processed
by a much larger number of nodes than in the wired case.
This makes the computational complexity at least $O(n^2)$ ($n$ being
the number of nodes.) Other effects may further increase the exponent.

The medium module provides a set of parameters that can be used
to alleviate the scalability issue. These \textit{filter} parameters
that can be used to reduce the amount of processing at nodes that are
not the indended recipients of the frame, increasing simulation performance.

There are several filters that can be enabled/disabled individually:

\begin{itemize}
  \item \textit{Range filter}. When this filter is active, the medium module
    does not send signals to a radio if it is outside interference range
    (or communication range, this option can also be selected.)
  \item \textit{Radio mode filter}. When this filter is active,
    the medium module does not send signals to a radio if it is neither
    in \textit{receiver} nor in \textit{transceiver} mode.
  \item \textit{Listening filter}. When this filter is active, the medium module
    does not send signals to a radio if it listens on the channel in
    incompatible mode (e.g. different carrier frequency and bandwidth,
    or different modulation)
  \item \textit{MAC address filter}. When this filter is active, the radio medium
    does not send signals to a radio if it the destination MAC address
    does not match
\end{itemize}

The corresponding module parameters are called \ttt{rangeFilter},
\ttt{radioModeFilter}, \ttt{listeningFilter} and \ttt{macAddressFilter}.
By default, all filters are turned off.

TODO when is it safe to use them?

TODO example

\section{Pitfalls}
\label{sec:medium:pitfalls}

Why a packet is not received correctly by the radio (PHY)?
- radio mode
- listening mode
- range filter (or out of range)
- attenuation (signal too weak)
- interference too strong
- capture not supported (radio already receiving another frame, and does notswitch)
- sensitivity is too low (threshold is too high) [W]
- SNIR threshold
- error model (random)
- PHY layer checksum



\section{Optimizations}
\label{sec:medium:optimizations}

turn on filters

experiment with caches (neighbor cache, comm cache, limits cache -- this one
may numerically alter results) 

use a more abstract radio model that's still suitable


%%% Local Variables:
%%% mode: latex
%%% TeX-master: "usman"
%%% End:

\cleardoublepage

\chapter{The Physical Environment}
\label{cha:environment}

TODO C++ interface




\cleardoublepage

\chapter{Node Mobility}
\label{cha:mobility}

\section{Mobility in INET}

\subsection{MobilityBase class}

The abstract \cppclass{MobilityBase} class is the base of the mobility
modules defined in the INET framework. This class implements things like
constraint area (or cubic volume), initial position, and border policy.

When the module is initialized it sets the initial position of the node
by calling the \ffunc{initializePosition()} method. The default implementation
handles the \fpar{initFromDisplayString}, \fpar{initialX}, \fpar{initialY}, \fpar{initialZ}
parameters.

The module is responsible for periodically updating the position.
For this purpose it should send timer messages to itself. These messages
are processed in the \ffunc{handleSelfMessage} method. In derived
classes, \ffunc{handleSelfMessage} should compute the new position
and update the display string and publish the new position by calling
the \ffunc{positionUpdated} method.

When the node reaches the boundary of the constraint area, the mobility
component has to prevent the node to exit. It can call the
\ffunc{handleIfOutside} method, which offers policies like
reflect, torus, random placement, and error.



\subsection{MovingMobilityBase}

The abstract \cppclass{MovingMobilityBase} class can be used to model
mobilities when the node moves on a continous trajectory and
updates its position periodically. Subclasses only need to implement
the \ffunc{move} method that is responsible to update the current
position and speed of the node.

The abstract \ffunc{move} method is called autmotically in every
\fpar{updateInterval} steps. The method is also called when a client
requested the current position or speed or when the \ffunc{move} method
requested an update at a future moment by setting the \fvar{nextChange}
field. This can be used when the state of the motion changes at a
specific time that is not a multiple of \fpar{updateInterval}.
The method can set the \fpar{stationary} field to \ttt{true} to
indicate that the node reached its final position and no more position
update is needed.

% TODO draw a plot of t-position function marking the points when
%      move() is called, stationary set to true, etc.

\subsection{LineSegmentsMobilityBase}

The path of a mobile node often consist of linear movements of constant
speed. The node moves with some speed for some time, then with another
speed for another duration and so on. If a mobility model fits this
description, it might be suitable to derive the implementing C++ class
from \cppclass{LineSegmentsMobilityBase}.

The module first choose a target position and a target time by calling
the \ffunc{setTargetPosition} method. If the target position differs
from the current position, it starts to move toward the target and
updates the position in the configured \fpar{updateInterval} intervals.
When the target position reached, it chooses a new target.

% TODO draw a plot like above, but containing linear segments, mark
%      the points when setTargetPosition called.

% FIXME LineSegmentsMobilityBase should not schedule the self message at updateInterval
%       when lastPosition==targetPosition (the node is waiting at the current position,
%       e.g. every second step in RandomWPMobility)
% TODO Consider an updateInterval computed from an updateDistance and speed, because position change
%      may be irrevelant during a preconfigured updateInterval.





\cleardoublepage

\chapter{The Power Model}
\label{cha:power}

TODO C++ interface

\section{Energy Consumer Models}

TODO C++ interface

\section{Energy Generator Models}

TODO C++ interface

\section{Energy Storage Models}

TODO C++ interface


\cleardoublepage

\chapter{Network Emulation}
\label{cha:emulation}

\section{Motivation}
\label{sec:emulation:motivation}

There are several projects that may benefit from the network emulation
capabilities of INET, that is, from the ability to mix simulated components
with real networks.

Some example scenarios:

\begin{itemize}
  \item Run a simulated component, such as an app or a routing protocol,
    on nodes of an actual ad-hoc network. This setup would allow testing
    the component's behavior under real-life conditions.
  \item Test the interoperability of a simulated protocol with its real-world
    counterparts. Several setups are possible: simulated node in a real network;
    a simulated subnet in real network; real-world node in simulated network; etc.
  \item As a means of implementing hybrid simulation. The real network
    (or a single host OS) may contain several network emulator devices
    or simulations running in emulation mode. Such a setup provides a relatively
    easy way for connecting heterogenous simulators/emulators with each
    other, sparing the need for HLA or a custom interoperability solution.
\end{itemize}


\section{Overview}
\label{sec:emulation:overview}

To act as a network emulator, the simulation must run in real time,
and must be able to communicate with the real world.

% [Background] To understand how it works, first look at how
% a normal simulation is run. There is a scheduler which always
% returns the earliest event from the FES, and the simulation processes
% these events in as fast succession as possible.
%
% For real-time simulation, this scheduler is replaced with one
% augmented with wait calls (e.g. usleep()) that synchronize the
% simulation time to the system clock.
%
% For emulation, the real-time scheduler is augmented with code
% that captures packets from real network devices, and inserts
% them into the simulation.
%
% INET contains an emulation scheduler, which uses the pcap library
% to capture packets, and raw sockets to send packets to a real network device.
%

This is achieved with two components in INET:

\begin{itemize}
  \item \nedtype{ExtInterface} is an INET network interface that represents
    a real interface (an interface of the host OS) in the simulation.
    Packets sent to an \nedtype{ExtInterface} will be sent out on the
    host OS interface, and packets received by the host OS interface
    (or rather, the appropriate subset of them) will appear in the
    simulation as if received on an \nedtype{ExtInterface}. The code
    uses the pcap library for capturing packets, and raw sockets for sending.
 \item \cppclass{RealTimeScheduler}, a socket-aware real-time scheduler class.
\end{itemize}

\begin{note}
It is probably needless to say, but the simulation must be fast enough
to be able to keep up with real time. That is, its relative speed compared
to real time (the simsec/sec value) must be >>1.  (Under Qtenv, this
can usually only be achieved in Express mode.)
\end{note}

The simulation is run under Qtenv,

\section{Preparation}
\label{sec:emulation:preparation}

There are a few things that need to be arranged before you can successfully
run simulations in network emulation mode.

First, network emulation is a separate \textit{project feature} that needs to
be enabled before it can be used. (Project features can be reviewed and changed
in the \textit{Project | Project Features...} dialog in the IDE.)

The network emulation code makes use of the pcap library, and therefore
it must be available on your system. On Ubuntu, for example, pcap can be
installed with the following command:

\begin{verbatim}
$ sudo apt install libpcap-dev
\end{verbatim}

Also, when running a simulation, make sure you have the necessary permissions.
Sending uses raw sockets (type \ttt{SOCK\_RAW}), which, on many systems,
is only allowed for processes that have root (administrator) privileges.


\section{Configuring}
\label{sec:emulation:configuring}

INET nodes such as \nedtype{StandardHost} and \nedtype{Router}
can be configured to have \nedtype{ExtInterface}'s.
The simulation may contain several nodes with external interfaces,
and one node may also have several external interfaces.

A network node can be configured to have an external interface
in the following way:

\begin{inifile}
**.host1.numExtInterfaces = 1
\end{inifile}

Also, the simulation must be configured to run under control the of the
appropriate real-time scheduler class:

\begin{inifile}
scheduler-class = "inet::RealTimeScheduler"
\end{inifile}

\nedtype{ExtInterface} has two important parameters which need to be
configured. The \fpar{device} parameter should be set to the name of the real
interface on the host OS, and \fpar{filterString} should contain a packet
filter expression that selects which packets captured on the real interface
should  be relayed into the simulation via this \nedtype{ExtInterface}.
(\fpar{filterString} is simply passed to the pcap library, so it should
follow the \textit{tcpdump} filter expressions syntax that pcap understands.)

An example configuration:

\begin{inifile}
**.numExtInterfaces = 1
**.ext[0].ext.filterString = "(sctp or icmp) and ip dst host 10.1.1.1"
**.ext[0].ext.device = "eth0" # or "en0" on macOS, or something
**.ext[0].ext.mtu = 1500B
\end{inifile}

The filter string \ttt{"(sctp or icmp) and ip dst host 10.1.1.1"} means
that the protocol must be SCTP or ICMP, and the destination host must be
10.1.1.1.

\begin{note}
Why is filtering of incoming packets done at packet capture (in pcap),
and not in \nedtype{ExtInterface}? The reason is performance: it costs
much fewer CPU cycles to discard unnecessary packets right where
they come in, and not send them up into the simulation for the
same decision. And, given that the simulation needs to keep up with
real time, saving CPU cycles is important.
\end{note}

Let us examine the paths outgoing and incoming packets take, and the
necessary configuration requirements to make them work. We assume IPv4
as network layer protocol, but the picture does not change much with
other protocols. We assume the external interface is named \ttt{ext[0]}.

\subsection*{Outgoing path}

The network layer of the simulated node routes datagrams to its
\ttt{ext[0]} external interface.

For that to happen, the routing table needs to contain an entry
where the interface is set to \ttt{ext[0]}. Such entries are
not created automatically, one needs to add them to the routing
table explicitly, e.g. by using an \nedtype{Ipv4NetworkConfigurator}
and an appropriate XML file.

Another point is that if the packet comes from a local app (and from
another simulated node), it needs to have a source IP address assigned.
There are two ways for that to happen. If the sending app specified
a source IP address, that will be used. Otherwise, the IP address
of the \ttt{ext[0]} interface will be used, but for that, the interface
needs to have an IP address at all.

Once in \ttt{ext[0]}, the datagram is serialized.
Serialization is a built-in feature of INET packets. (Packets, or rather,
packet chunks have multiple alternative representations, i.e. C++ object
and serialized form, and conversion between them is transparent.)

The result of serialization is a byte string, which is written into
a raw socket with a \ttt{sendto} system call.

% TODO this was true in 4.0 prereleases, hopefully fixed in final 4.0:
% The host OS will again route the packet, and send it out on the
% output interface determined by the packet's destination IP address.
% (Note that this may not be the same as the interface specified
% in \ttt{ext[0]}'s \fpar{device} parameter. Unfortunately,
% in the current implementation one raw socket is shared by
% all \nedtype{ExtInterface} instances which may have different
% \fpar{device} settings, so the socket cannot be bound to the
% real interface associated with originator \nedtype{ExtInterface}.)

The packet will then travel normally in the real network to the
destination address.

\subsection*{Incoming path}

First of all, packets intended to be received by the simulation
need to find their way to the correct  interface of the host that
runs the simulation. For that, IP addresses of simulated hosts
must be routable in the real network, and routed to the captured
interface of the host OS. (On Linux, for example, this can be achieved
by adding static routes with the \ttt{ip route add <prefix> via
<host>} command.)

As packets are received by the interface of the host OS, they
are examined by the pcap library to find out whether they match
the filter expression. If the filter matches, pcap hands the
packet over to the simulation, and after deserialization
it pops out of \ttt{ext[0]} and sent up to the network
layer. After that, it is routed to the simulated destination host
in the normal way.

The pcap filter expression must be crafted so that it matches
the packets destined to simulated hosts, and does not match
any other packet.

Moreover, if the simulation contains several external interfaces
that map to the same real interface, care must be taken so that
filter expressions are disjunct. Otherwise, a packet may be
matched by more than one filter, and then it will be inserted
into the simulation in multiple copies (once for each matching
\nedtype{ExtInterface}.) This is usually not what is wanted.

%%% Local Variables:
%%% mode: latex
%%% TeX-master: "usman"
%%% End:


\cleardoublepage

\chapter{Network Autoconfiguration}
\label{cha:network-autoconfiguration}

\section{Overview}
\label{sec:autoconfig:overview}

This chapter describes static autoconfiguration of networks.

\section{Configuring IPv4 Networks}
\label{sec:autoconfig:configuring-ipv4-networks}

An IPv4 network is composed of several nodes like hosts, routers,
switches, hubs, Ethernet buses, or wireless access points.
The nodes having a IPv4 network layer (hosts and routers) should be
configured at the beginning of the simulation. The configuration
assigns IP addresses to the nodes, and fills their routing tables.
If multicast forwarding is simulated, then the multicast routing
tables also must be filled in.

% TODO define nodes, IP nodes, routers, multicast routers

The configuration can be manual (each address and route is fully specified
by the user), or automatic (addresses and routes are generated by
a configurator module at startup).

Before version 1.99.4 INET offered \nedtype{Ipv4FlatNetworkConfigurator}
for automatic and routing files for manual configuration.
Both had serious limitations, so a new configurator has been added
in version 1.99.4: \nedtype{Ipv4NetworkConfigurator}. This configurator
supports both fully manual and fully automatic configuration. It
can also be used with partially specified manual configurations,
the configurator fills in the gaps automatically.

The next section describes the usage of \nedtype{Ipv4NetworkConfigurator}.
The legacy solutions \nedtype{Ipv4FlatNetworkConfigurator} and
routing files are described in subsequent sections.

\subsection{Ipv4NetworkConfigurator}
\label{sec:autoconfig:ipv4networkconfigurator}

The \nedtype{Ipv4NetworkConfigurator} assigns IP addresses and sets up
static routing for an IPv4 network.

It assigns per-interface IP addresses, strives to take subnets into account,
and can also optimize the generated routing tables by merging routing entries.

Hierarchical routing can be set up by using only a fraction of configuration
entries compared to the number of nodes. The configurator also does
routing table optimization that significantly decreases the size of routing
tables in large networks.

The configuration is performed in stage 2 of the initialization. At this
point interface modules (e.g. PPP) has already registered their interface
in the interface table. If an interface is named \ttt{ppp[0]}, then the
corresponding interface entry is named \ttt{ppp0}. This name can be used
in the config file to refer to the interface.

The configurator goes through the following steps:

\begin{enumerate}
  \item  Builds a graph representing the network topology. The graph
     will have a vertex for every module that has a \ttt{@node} property (this
     includes hosts, routers, and L2 devices like switches, access points,
     Ethernet hubs, etc.) It also assigns weights to vertices and edges that
     will be used by the shortest path algorithm when setting up routes.
     Weights will be infinite for IP nodes that have IP forwarding disabled
     (to prevent routes from transiting them), and zero for all other nodes
     (routers and and L2 devices). Edge weights are chosen to be inversely
     proportional to the bitrate of the link, so that the configurator
     prefers connections with higher bandwidth. For internal purposes,
     the configurator also builds a table of all "links" (the link data
     structure consists of the set of network interfaces that are
     on the same point-to-point link or LAN)
  \item  Assigns IP addresses to all interfaces of all nodes. The
     assignment process takes into consideration the addresses and netmasks
     already present on the interfaces (possibly set in earlier initialize
     stages), and the configuration provided in the XML format (described
     below). The configuration can specify "templates" for the address
     and netmask, with parts that are fixed and parts that can be chosen
     by the configurator (e.g. "10.0.x.x"). In the most general case,
     the configurator is allowed to choose any address and netmask for all
     interfaces (which results in automatic address assignment). In the most
     constrained case, the configurator is forced to use the requested addresses
     and netmasks for all interfaces (which translates to manual address assignment).
     There are many possible configuration options between these two extremums. The
     configurator assigns addresses in a way that maximizes the number of
     nodes per subnet. Once it figures out the nodes that belong to a single
     subnet it, will optimize for allocating the longest possible netmask.
     The configurator might fail to assign netmasks and addresses according
     to the given configuration parameters; if that happens, the assignment
     process stops and an error is signalled.
  \item  Adds the manual routes that are specified in the configuration.
  \item  Adds static routes to all routing tables in the network. The
     configurator uses Dijkstra's weighted shortest path algorithm to find
     the desired routes between all possible node pairs. The resulting
     routing tables will have one entry for all destination interfaces in the
     network. The configurator can be safely instructed to add default routes
     where applicable, significantly reducing the size of the host routing
     tables. It can also add subnet routes instead of interface routes further
     reducing the size of routing tables. Turning on this option requires
     careful design to avoid having IP addresses from the same subnet on
     different links. CAVEAT: Using manual routes and static route generation
     together may have unwanted side effects, because route generation ignores
     manual routes.
  \item  Then it optimizes the routing tables for size. This optimization allows
     configuring larger networks with smaller memory footprint and makes the
     routing table lookup faster. The resulting routing table might be
     different in that it will route packets that the original routing table
     did not. Nevertheless the following invariant holds: any packet routed
     by the original routing table (has matching route) will still be routed
     the same way by the optimized routing table.
  \item  Finally it dumps the requested results of the configuration. It can
     dump network topology, assigned IP addresses, routing tables and its
     own configuration format.
\end{enumerate}

The module can dump the result of the configuration in the XML format
which it can read. This is useful to save the result of a time consuming
configuration (large network with optimized routes), and use it as
the config file of subsequent runs.

\subsubsection*{Network topology graph}

The network topology graph is constructed from the nodes
of the network. The node is a module having a \ttt{@node} property
(this includes hosts, routers, and L2 devices like switches,
 access points, Ethernet hubs, etc.). An IP node is a node
that contains an \nedtype{InterfaceTable} and a \nedtype{Ipv4RoutingTable}.
A router is an IP node that has multiple network interfaces,
and IP forwarding is enabled in its routing table module.
In multicast routers the \fpar{forwardMulticast} parameter
is also set to \ttt{true}.

A link is a set of interfaces that can send datagrams to each other
without intervening routers. Each interface belongs to exactly
one link. For example two interface connected
by a point-to-point connection forms a link. Ethernet interfaces
connected via buses, hubs or switches.
The configurator identifies links by discovering
the connections between the IP nodes, buses, hubs, and switches.

Wireless links are identified by the \fpar{ssid} or \fpar{accessPointAddress}
parameter of the 802.11 management module. Wireless interfaces
whose node does not contain a management module are supposed
to be on the same wireless link. Wireless links can also be
configured in the configuration file of \nedtype{Ipv4NetworkConfigurator}:

\begin{XML}
<config>
  <wireless hosts="area1.*" interfaces="wlan*">
</config>
\end{XML}

puts wlan interfaces of the specified hosts into the same wireless link.

If a link contains only one router, it is marked as the gateway
of the link. Each datagram whose destination is outside the link
must go through the gateway.

\subsubsection*{Address assignment}

Addresses can be set up manually by giving the address and netmask for
each IP node. If some part of the address or netmask is unspecified,
then the configurator can fill them automatically. Unspecified fields
are given as an ``x'' character in the dotted notation of the address.
For example, if the address is specified as 192.168.1.1 and the
netmask is 255.255.255.0, then the node address will be 192.168.1.1
and its subnet is 192.168.1.0. If it is given as 192.168.x.x and
255.255.x.x, then the configurator chooses a subnet address in the range
of 192.168.0.0 - 192.168.255.252, and an IP address within the chosen
subnet. (The maximum subnet mask is 255.255.255.252 allows 2 nodes in the subnet.)

The following configuration generates network addresses below the 10.0.0.0
address for each link, and assign unique IP addresses to each host:

\begin{XML}
<config>
  <interface hosts="*" address="10.x.x.x" netmask="255.x.x.x"/>
</config>
\end{XML}

The configurator tries to put nodes on the same link into the same subnet,
so its enough to configure the address of only one node on each link.

The following example configures a hierarchical network in a way that keeps
routing tables small.
\begin{XML}
<config>
  <interface hosts="area11.lan1.*" address="10.11.1.x" netmask="255.255.255.x"/>
  <interface hosts="area11.lan2.*" address="10.11.2.x" netmask="255.255.255.x"/>
  <interface hosts="area12.lan1.*" address="10.12.1.x" netmask="255.255.255.x"/>
  <interface hosts="area12.lan2.*" address="10.12.2.x" netmask="255.255.255.x"/>
  <interface hosts="area*.router*" address="10.x.x.x" netmask="x.x.x.x"/>
  <interface hosts="*" address="10.x.x.x" netmask="255.x.x.0"/>
</config>
\end{XML}

The XML configuration must contain exactly one \verb!<config>! element. Under the
root element there can be multiple of the following elements:

The interface element provides configuration parameters for one or more
interfaces in the network. The selector attributes limit the scope where
the interface element has effects. The parameter attributes limit the
range of assignable addresses and netmasks.
The \verb!<interface>! element may contain the following attributes:
\begin{itemize}
    \item \ttt{@hosts}
      Optional selector attribute that specifies a list of host name patterns.
      Only interfaces in the specified hosts are affected. The pattern might
      be a full path starting from the network, or a module name anywhere in
      the hierarchy, and other patterns similar to ini file keys. The default
      value is "*" that matches all hosts.
      e.g. "subnet.client*" or "host* router[0..3]" or "area*.*.host[0]"
    \item \ttt{@names}
      Optional selector attribute that specifies a list of interface name
      patterns. Only interfaces with the specified names are affected. The
      default value is "*" that matches all interfaces.
      e.g. "eth* ppp0" or "*"
    \item \ttt{@towards}
      Optional selector attribute that specifies a list of host name patterns.
      Only interfaces connected towards the specified hosts are affected. The
      specified name will be matched against the names of hosts that are on
      the same LAN with the one that is being configured. This works even if
      there's a switch between the configured host and the one specified here.
      For wired networks it might be easier to specify this parameter instead
      of specifying the interface names. The default value is "*".
      e.g. "ap" or "server" or "client*"
    \item \ttt{@among}
      Optional selector attribute that specifies a list of host name patterns.
      Only interfaces in the specified hosts connected towards the specified
      hosts are affected.
      The 'among="X Y Z"' is same as 'hosts="X Y Z" towards="X Y Z"'.
    \item \ttt{@address}
      Optional parameter attribute that limits the range of assignable
      addresses. Wildcards are allowed with using 'x' as part of the address
      in place of a byte. Unspecified parts will be filled automatically by
      the configurator. The default value "" means that the address will not
      be configured. Unconfigured interfaces still have allocated addresses
      in their subnets allowing them to become configured later very easily.
      e.g. "192.168.1.1" or "10.0.x.x"
    \item \ttt{@netmask}
      Optional parameter attribute that limits the range of assignable
      netmasks. Wildcards are allowed with using 'x' as part of the netmask
      in place of a byte. Unspecified parts will be filled automatically be
      the configurator. The default value "" means that any netmask can be
      configured.
      e.g. "255.255.255.0" or "255.255.x.x" or "255.255.x.0"
    \item \ttt{@mtu}                number
      Optional parameter attribute to set the MTU parameter in the interface.
      When unspecified the interface parameter is left unchanged.
    \item \ttt{@metric}                number
      Optional parameter attribute to set the Metric parameter in the interface.
      When unspecified the interface parameter is left unchanged.
\end{itemize}

Wireless interfaces can similarly be configured by adding
\verb!<wireless>! elements to the configuration. Each \verb!<wireless>!
element with a different id defines a separate subnet.

\begin{itemize}
    \item \ttt{@id} (optional)
      identifies wireless network, unique value used if missed
    \item \ttt{@hosts}
      Optional selector attribute that specifies a list of host name patterns.
      Only interfaces in the specified hosts are affected. The default value
      is "*" that matches all hosts.
    \item \ttt{@interfaces}
      Optional selector attribute that specifies a list of interface name
      patterns. Only interfaces with the specified names are affected. The
      default value is "*" that matches all interfaces.
\end{itemize}


\subsubsection{Multicast groups}
\label{sec:autoconfig:multicast-groups}

Multicast groups can be configured by adding \verb!<multicast-group>!
elements to the configuration file. Interfaces belongs to a multicast
group will join to the group automatically.

For example,

\begin{XML}
<config>
  <multicast-group hosts="router*" interfaces="eth*" address="224.0.0.5"/>
</config>
\end{XML}

adds all Ethernet interfaces of nodes whose name starts with ``router''
to the 224.0.0.5 multicast group.

The \verb!<multicast-group>! element has the following attributes:
\begin{itemize}
    \item \ttt{@hosts}
      Optional selector attribute that specifies a list of host name patterns.
      Only interfaces in the specified hosts are affected. The default value
      is "*" that matches all hosts.
    \item \ttt{@interfaces}
      Optional selector attribute that specifies a list of interface name
      patterns. Only interfaces with the specified names are affected. The
      default value is "*" that matches all interfaces.
    \item \ttt{@towards}
      Optional selector attribute that specifies a list of host name patterns.
      Only interfaces connected towards the specified hosts are affected.
      The default value is "*".
    \item \ttt{@among}
      Optional selector attribute that specifies a list of host name patterns.
      Only interfaces in the specified hosts connected towards the specified
      hosts are affected.
      The 'among="X Y Z"' is same as 'hosts="X Y Z" towards="X Y Z"'.
    \item \ttt{@address}
      Mandatory parameter attribute that specifies a list of multicast group
      addresses to be assigned. Values must be selected from the valid range
      of multicast addresses.
      e.g. "224.0.0.1 224.0.1.33"
\end{itemize}


\subsubsection*{Manual route configuration}

The \nedtype{Ipv4NetworkConfigurator} module allows the user
to fully specify the routing tables of IP nodes at the beginning
of the simulation.

The \verb!<route>! elements of the configuration add a route to the
routing tables of selected nodes. The element has the following attributes:
\begin{itemize}
    \item \ttt{@hosts}
      Optional selector attribute that specifies a list of host name patterns.
      Only routing tables in the specified hosts are affected. The default
      value "" means all hosts will be affected.
      e.g. "host* router[0..3]"
    \item \ttt{@destination}
      Optional parameter attribute that specifies the destination address in
      the route (L3AddressResolver syntax). The default value is "*".
      e.g. "192.168.1.1" or "subnet.client[3]" or "subnet.server(ipv4)" or "*"
    \item \ttt{@netmask}
      Optional parameter attribute that specifies the netmask in the route.
      The default value is "*".
      e.g. "255.255.255.0" or "/29" or "*"
    \item \ttt{@gateway}
      Optional parameter attribute that specifies the gateway (next-hop)
      address in the route (L3AddressResolver syntax). When unspecified
      the interface parameter must be specified. The default value is "*".
      e.g. "192.168.1.254" or "subnet.router" or "*"
    \item \ttt{@interface}
      Optional parameter attribute that specifies the output interface name
      in the route. When unspecified the gateway parameter must be specified.
      This parameter has no default value.
      e.g. "eth0"
    \item \ttt{@metric}
      Optional parameter attribute that specifies the metric in the route.
      The default value is 0.
\end{itemize}

Multicast routing tables can similarly be configured by adding
\verb!<multicast-route>! elements to the configuration.
\begin{itemize}
    \item \ttt{@hosts}
      Optional selector attribute that specifies a list of host name patterns.
      Only routing tables in the specified hosts are affected.
      e.g. "host* router[0..3]"
    \item \ttt{@source}
      Optional parameter attribute that specifies the address of the source
      network. The default value is "*" that matches all sources.
    \item \ttt{@netmask}
      Optional parameter attribute that specifies the netmask of the source
      network. The default value is "*" that matches all sources.
    \item \ttt{@groups}
      Optional List of IPv4 multicast addresses specifying the groups this entry
      applies to. The default value is "*" that matches all multicast groups.
      e.g. "225.0.0.1 225.0.1.2".
    \item \ttt{@metric}
      Optional parameter attribute that specifies the metric in the route.
    \item \ttt{@parent}
      Optional parameter attribute that specifies the name of the interface
      the multicast datagrams are expected to arrive. When a datagram arrives
      on the parent interface, it will be forwarded towards the child interfaces;
      otherwise it will be dropped. The default value is the interface on the
      shortest path towards the source of the datagram.
    \item \ttt{@children}
      Mandatory parameter attribute that specifies a list of interface name
      patterns:
      \begin{itemize}
        \item a name pattern (e.g. "ppp*") matches the name of the interface
        \item a 'towards' pattern (starting with ">", e.g. ">router*") matches the interface
         by naming one of the neighbour nodes on its link.
      \end{itemize}
      Incoming multicast datagrams are forwarded to each child interface except the
      one they arrived in.
\end{itemize}

The following example adds an entry to the multicast routing table of \ttt{router1},
that intsructs the routing algorithm to forward multicast datagrams whose source
is in the 10.0.1.0 network and whose destinatation address is 225.0.0.1 to
send on the \ttt{eth1} and \ttt{eth2} interfaces assuming it arrived on the
\ttt{eth0} interface:

\begin{XML}
<multicast-route hosts="router1" source="10.0.1.0" netmask="255.255.255.0"
                 groups="225.0.0.1" metric="10"
                 parent="eth0" children="eth1 eth2"/>
\end{XML}

\subsubsection*{Automatic route configuration}

If the \fpar{addStaticRoutes} parameter is true, then
the configurator add static routes to all routing tables.

The configurator uses Dijkstra's weighted shortest path algorithm to find
the desired routes between all possible node pairs. The resulting
routing tables will have one entry for all destination interfaces in the
network.

%     Weights will be infinite for IP nodes that have IP forwarding disabled
%     (to prevent routes from transiting them), and zero for all other nodes
%     (routers and and L2 devices). Edge weights are chosen to be inversely
%     proportional to the bitrate of the link, so that the configurator
%     prefers connections with higher bandwidth. For internal purposes,

The configurator can be safely instructed to add default routes
where applicable, significantly reducing the size of the host routing
tables. It can also add subnet routes instead of interface routes further
reducing the size of routing tables. Turning on this option requires
careful design to avoid having IP addresses from the same subnet on
different links.


\begin{caution}
Using manual routes and static route generation
together may have unwanted side effects, because route generation ignores
manual routes. Therefore if the configuration file contains
manual routes, then the \fpar{addStaticRoutes} parameter should be set
to \ttt{false}.
\end{caution}

\subsubsection*{Route optimization}

If the \fpar{optimizeRoutes} parameter is \ttt{true} then the
configurator tries to optimize the routing table for size.
This optimization allows configuring larger networks with smaller
memory footprint and makes the routing table lookup faster.

The optimization is performed by merging routes whose gateway and
outgoing interface is the same by finding a common prefix that
matches only those routes. The resulting routing table might be
different in that it will route packets that the original routing table
did not. Nevertheless the following invariant holds: any packet routed
by the original routing table (has matching route) will still be routed
the same way by the optimized routing table.

\subsubsection*{Parameters}

This list summarize the parameters of the \nedtype{IPv4NetorkConfigurator}:

\begin{itemize}
  \item \fpar{config}: XML configuration parameters for IP address assignment
    and adding manual routes.
  \item \fpar{assignAddresses}: assign IP addresses to all interfaces in the network
  \item \fpar{assignDisjunctSubnetAddresses}: avoid using the same address prefix and
    netmask on different links when assigning IP addresses to interfaces
  \item \fpar{addStaticRoutes}: add static routes to the routing tables of all nodes
    to route to all destination interfaces (only where applicable; turn off when
    config file contains manual routes)
  \item \fpar{addDefaultRoutes}: add default routes if all routes from a source node
     go through the same gateway (used only if addStaticRoutes is true)
  \item \fpar{addSubnetRoutes}: add subnet routes instead of destination interface routes
    (only where applicable; used only if addStaticRoutes is true)
  \item \fpar{optimizeRoutes}: optimize routing tables by merging routes, the
    resulting routing table might route more packets than the original
    (used only if addStaticRoutes is true)
  \item \fpar{dumpTopology}: if true, then the module prints extracted network topology
  \item \fpar{dumpAddresses}: if true, then the module prints assigned IP addresses
    for all interfaces
  \item \fpar{dumpRoutes}: if true, then the module prints configured and optimized
    routing tables for all nodes to the module output
  \item \fpar{dumpConfig}: name of the file, write configuration into the given
    config file that can be fed back to speed up subsequent runs (network configurations)
\end{itemize}

\subsection{Ipv4FlatNetworkConfigurator (Legacy)}
\label{sec:autoconfig:ipv4flatnetworkconfigurator}

The \nedtype{Ipv4FlatNetworkConfigurator} module configures
IP addresses and routes of IP nodes of a network.
All assigned addresses share a common subnet prefix,
the network topology will be ignored. Shortest path
routes are also generated from any node to any other
node of the network. The Gateway (next hop) field of the routes
is not filled in by these configurator, so it relies
on proxy ARP if the network spans several LANs.
It does not perform routing table optimization (i.e.
merging similar routes into a single, more general route.)

\begin{warning}
\nedtype{Ipv4FlatNetworkConfigurator} is considered
legacy, do not use it for new projects.
\end{warning}

The \nedtype{Ipv4FlatNetworkConfigurator} module configures
the network when it is initialized. The configuration
is performed in stage 2, after interface tables are
filled in. Do not use a \nedtype{Ipv4FlatNetworkConfigurator}
module together with static routing files, because they
can iterfere with the configurator.

The \nedtype{Ipv4FlatNetworkConfigurator} searches each IP nodes of the network.
(IP nodes are those modules that have the @node NED property and
has a \nedtype{Ipv4RoutingTable} submodule named ``routingTable'').
The configurator then assigns IP addresses to the IP nodes, controlled
by the following module parameters:
\begin{itemize}
  \item \fpar{netmask} common netmask of the addresses (default is 255.255.0.0)
  \item \fpar{networkAddress} higher bits are the network part of the addresses,
        lower bits should be 0. (default is 192.168.0.0)
\end{itemize}

With the default parameters the assigned addresses are in the range
192.168.0.1 - 192.168.255.254, so there can be maximum 65534 nodes in the
network. The same IP address will be assigned to each interface
of the node, except the loopback interface which always has address 127.0.0.1
(with 255.0.0.0 mask).

After assigning the IP addresses, the configurator fills in the routing tables.
There are two kind of routes:
\begin{itemize}
  \item default routes: for nodes that has only one non-loopback interface
        a route is added that matches with any destination address
        (the entry has 0.0.0.0 \ttt{host} and \ttt{netmask} fields).
        These are remote routes, but the gateway address is left unspecified.
        The delivery of the datagrams rely on the proxy ARP feature of the
        routers.
  \item direct routes following the shortest paths: for nodes that has more
        than one non-loopback interface a separate route is added to each
        IP node of the network. The outgoing interface is chosen by the
        shortest path to the target node. These routes are
        added as direct routes, even if there is no direct link with the
        destination. In this case proxy ARP is needed to deliver the datagrams.
\end{itemize}

\begin{note}
This configurator does not try to optimize the routing tables.
If the network contains $n$ nodes, the size of all routing tables
will be proportional to $n^2$, and the time of the lookup of the
best matching route will be proportional to $n$.
\end{note}

% FIXME weird FlatNetworkConfigurator behaviour.
%       Assigned IP addresses does not mirror the hierachy of networks (e.g. each node in an Ethernet LAN handled as a one-element subnet).
%       No gateway address is set in the routes, delivery relies on proxy ARPing.
%       Direct routes created to each node, even if there is no direct link to it.
%       Different interfaces of a node should have different IP address.
%       Broadcast capable interfaces should have a real netmast (not 255.255.255.255) to support subnet directed IP broadcasts.

\subsection{Routing Files (Legacy)}
\label{sec:autoconfig:routing-files}

Routing files are files with \ttt{.irt} or \ttt{.mrt} extension,
and their names are passed in the \fpar{routingFile} parameter
to \nedtype{Ipv4RoutingTable} modules.

Routing files may contain network interface configuration and static
routes. Both are optional. Network interface entries in the file
configure existing interfaces; static routes are added to the route table.

\begin{warning}
\nedtype{Routing files} are considered legacy, use do not use them for new
projects. Their contents can be expressed in \nedtype{Ipv4NetworkConfigurator}
config files.
\end{warning}

Interfaces themselves are represented in the simulation by modules
(such as the PPP module). Modules automatically register themselves
with appropriate defaults in the IPv4RoutingTable, and entries in the
routing file refine (overwrite) these settings.
Interfaces are identified by names (e.g. ppp0, ppp1, eth0) which
are normally derived from the module's name: a module called
\ttt{"ppp[2]"} in the NED file registers itself as interface ppp2.

An example routing file (copied here from one of the example simulations):

\begin{verbatim}
ifconfig:

# ethernet card 0 to router
name: eth0   inet_addr: 172.0.0.3   MTU: 1500   Metric: 1  BROADCAST MULTICAST
Groups: 225.0.0.1:225.0.1.2:225.0.2.1

# Point to Point link 1 to Host 1
name: ppp0   inet_addr: 172.0.0.4   MTU: 576   Metric: 1

ifconfigend.

route:
172.0.0.2   *           255.255.255.255  H  0   ppp0
172.0.0.4   *           255.255.255.255  H  0   ppp0
default:    10.0.0.13   0.0.0.0          G  0   eth0

225.0.0.1   *           255.255.255.255  H  0   ppp0
225.0.1.2   *           255.255.255.255  H  0   ppp0
225.0.2.1   *           255.255.255.255  H  0   ppp0

225.0.0.0   10.0.0.13   255.0.0.0        G  0   eth0

routeend.
\end{verbatim}

The \ttt{ifconfig...ifconfigend.} part configures interfaces,
and \ttt{route..routeend.} part contains static routes.
The format of these sections roughly corresponds to the output
of the \ttt{ifconfig} and \ttt{netstat -rn} Unix commands.

An interface entry begins with a \ttt{name:} field, and lasts until
the next \ttt{name:} (or until \ttt{ifconfigend.}). It may
be broken into several lines.

Accepted interface fields are:

\begin{itemize}
  \item \ttt{name:} - arbitrary interface name (e.g. eth0, ppp0)
  \item \ttt{inet\_addr:} - IP address
  \item \ttt{Mask:} - netmask
  \item \ttt{Groups:} Multicast groups. 224.0.0.1 is added automatically,
     and 224.0.0.2 also if the node is a router (IPForward==true).
  \item \ttt{MTU:} - MTU on the link (e.g. Ethernet: 1500)
  \item \ttt{Metric:} - integer route metric
  \item flags: \ttt{BROADCAST}, \ttt{MULTICAST}, \ttt{POINTTOPOINT}
\end{itemize}

The following fields are parsed but ignored: \ttt{Bcast}, \ttt{encap},
\ttt{HWaddr}.

Interface modules set a good default for MTU, Metric (as $2*10^9$/bitrate) and
flags, but leave \fvar{inet\_addr} and \fvar{Mask} empty. \fvar{inet\_addr} and
\fvar{mask} should be set either from the routing file or by a dynamic network
configuration module.

The route fields are:

\begin{verbatim}
Destination  Gateway  Netmask  Flags  Metric Interface
\end{verbatim}

\fvar{Destination}, \fvar{Gateway} and \fvar{Netmask} have the usual meaning.
The \fvar{Destination} field should either be an IP address or ``default''
(to designate the default route). For \fvar{Gateway}, \ttt{*} is also
accepted with the meaning \ttt{0.0.0.0}.

\fvar{Flags} denotes route type:

\begin{itemize}
  \item \textit{H} ``host'': direct route (directly attached to the router), and
  \item \textit{G} ``gateway'': remote route (reached through another router)
\end{itemize}

\fvar{Interface} is the interface name, e.g. \ttt{eth0}.

\begin{important}
The meaning of the routes where the destination is a multicast address
has been changed in version 1.99.4. Earlier these entries was used
both to select the outgoing interfaces of multicast datagrams
sent by the higher layer (if multicast interface was otherwise unspecified)
and to select the outgoing interfaces of datagrams that are received from
the network and forwarded by the node.

From version 1.99.4 multicast routing applies reverse path forwarding.
This requires a separate routing table, that can not be populated from
the old routing table entries. Therefore simulations that use multicast
forwarding can not use the old configuration files, they should be
migrated to use an \nedtype{Ipv4NetworkConfigurator} instead.

Some change is needed in models that use link-local multicast too.
Earlier if the IP module received a datagram from the higher layer
and multiple routes was given for the multicast group,
then IP sent a copy of the datagram on each interface of that routes.
From version 1.99.4, only the first matching interface is used (considering
longest match). If the application wants to send the multicast datagram
on each interface, then it must explicitly loop and specify the multicast
interface.
\end{important}

% FIXME 'H' and 'G' flags should be independent. Now they excludes each other, the parser sets route.type to the last one.
%       H = host/network
%       G = indirect/direct

% TODO warn that multicast configuration has changed

\section{Configuring Layer 2}
\label{sec:autoconfig:configuring-layer-2}

The \nedtype{L2Configurator} module allows configuring network scenarios at layer 2.
The STP/RTP-related parameters such as link cost, port priority
and the ``is-edge'' flag can be configured with XML files.

This module is similar to \nedtype{Ipv4NetworkConfigurator}. It supports
the selector attributes \ttt{@hosts}, \ttt{@names}, \ttt{@towards}, \ttt{@among},
and they behave similarly to its \nedtype{Ipv4NetworkConfigurator} equivalent.
The \ttt{@ports} selector is also supported, for configuring per-port parameters.

The following example configures port 5 (if it exists) on all switches,
and sets cost=19 and priority=32768:

\begin{XML}
<config>
  <interface hosts='**' ports='5' cost='19' priority='32768'/>
</config>
\end{XML}

For more information about the usage of the selector attributes see
\nedtype{Ipv4NetworkConfigurator}.



%%% Local Variables:
%%% mode: latex
%%% TeX-master: "usman"
%%% End:


\cleardoublepage

\include{ch-scenario-scripting}
\cleardoublepage

\include{ch-lifecycle}
\cleardoublepage

\chapter{Collecting Results}
\label{cha:collecting-results}

TODO

\section{Recording Statistics}
\label{sec:results:recording-statistics}

@statistic

\section{Recording PCAP Traces}
\label{sec:results:recording-pcap-traces}

PcapRecorder

\section{Recording Routing Tables}
\label{sec:results:recording-routing-tables}

RoutingTableRecorder

\section{Packet Recorder}
\label{sec:results:packet-recorder}

\section{Eventlog Recording}
\label{sec:results:eventlog-recording}

mention

%%% Local Variables:
%%% mode: latex
%%% TeX-master: "usman"
%%% End:

\cleardoublepage

\chapter{Visualization}
\label{cha:visualization}

\section{Overview}
\label{sec:visualization:overview}

The INET Framework is able to visualize a wide range of events and conditions
in the network: packet drops, data link connectivity, wireless signal path loss,
transport connections, routing table routes, and many more. Visualization is
implemented as a collection of configurable INET modules that can be added
to simulations at will.

\section{Visualizing Network Communication}
\label{sec:visualization:network-communication}

\subsection{Visualizing Packet Drops}
\label{sec:visualization:packet-drops}

Several network problems manifest themselves as excessive packet drops, for
example poor connectivity, congestion, or misconfiguration. Visualizing packet
drops helps identifying such problems in simulations, thereby reducing time
spent on debugging and analysis. Poor connectivity in a wireless network can
cause senders to drop unacknowledged packets after the retry limit is exceeded.
Congestion can cause queues to overflow in a bottleneck router, again resulting
in packet drops.

Packet drops can be visualized by including a \nedtype{PacketDropVisualizer}
module in the simulation. The \nedtype{PacketDropVisualizer} module indicates
packet drops by displaying an animation effect at the node where the packet drop
occurs. In the animation, a packet icon gets thrown out from the node icon, and
fades away.

The visualization of packet drops can be enabled with the visualizer's
\fpar{displayPacketDrops} parameter. By default, packet drops at all nodes are
visualized. This selection can be narrowed with the \fpar{nodeFilter},
\fpar{interfaceFilter} and \fpar{packetFilter} parameters.

One can click on the packet drop icon to display information about the packet
drop in the inspector panel.

Packets are dropped for the following reasons:

\begin{itemize}
  \item queue overflow
  \item retry limit exceeded
  \item unroutable packet
  \item network address resolution failed
  \item interface down
\end{itemize}


\subsection{Visualizing Transport Path Activity}
\label{sec:visualization:transport-path-activity}

With INET simulations, it is often useful to be able to visualize network
traffic. INET provides several visualizers for this task, operating at various
levels of the network stack. One of such visualizers is \nedtype{TransportRouteVisualizer}
that provides graphical feedback about transport layer traffic.

\nedtype{TransportRouteVisualizer} visualizes traffic that passes through the 
transport layers of two endpoints. Adding an \nedtype{IntegratedVisualizer} is
also an option, because it also contains a \nedtype{TransportRouteVisualizer}. Transport
path activity visualization is disabled by default, it can be enabled by setting
the visualizer's \fpar{displayRoutes} parameter to true.

\nedtype{TransportRouteVisualizer} observes packets that pass through the transport layer,
i.e. carry data from/to higher layers.

The activity between two nodes is represented visually by a polyline arrow which
points from the source node to the destination node. \nedtype{TransportRouteVisualizer}
follows packets throughout their path so that the polyline goes through all
nodes which are the part of the path of packets. The arrow appears after the
first packet has been received, then gradually fades out unless it is reinforced
by further packets. Color, fading time and other graphical properties can be
changed with parameters of the visualizer.

By default, all packets and nodes are considered for the visualization. This
selection can be narrowed with the visualizer's packetFilter and nodeFilter
parameters.

\subsection{Visualizing Network Path Activity}
\label{sec:visualization:network-path-activity}

Network layer traffic can be visualized by including a \nedtype{NetworkRouteVisualizer}
module in the simulation. Adding an \nedtype{IntegratedVisualizer} module is also an
option, because it also contains a \nedtype{NetworkRouteVisualizer} module. Network path
activity visualization is disabled by default, it can be enabled by setting the
visualizer's \fpar{displayRoutes} parameter to true.

\nedtype{NetworkRouteVisualizer} currently observes packets that pass through the network
layer (i.e. carry data from/to higher layers), but not those that are internal
to the operation of the network layer protocol. That is, packets such as ARP,
although potentially useful, will not trigger the visualization.

The activity between two nodes is represented visually by a polyline arrow which
points from the source node to the destination node. \nedtype{NetworkRouteVisualizer}
follows packet throughout its path so the polyline goes through all nodes that
are part of the packet's path. The arrow appears after the first packet has been
received, then gradually fades out unless it is reinforced by further packets.
Color, fading time and other graphical properties can be changed with parameters
of the visualizer.

By default, all packets and nodes are considered for the visualization. This
selection can be narrowed with the visualizer's packetFilter and nodeFilter
parameters.


\subsection{Visualizing Data Link Activity}
\label{sec:visualization:data-link-activity}

Data link activity (layer 2 traffic) can be visualized by adding a \nedtype{DataLinkVisualizer} 
module to the simulation. Adding an \nedtype{IntegratedVisualizer} module is also an option,
because it includes a \nedtype{DataLinkVisualizer} module. Data link visualization is
disabled by default, it can be enabled by setting the visualizer's displayLinks
parameter to true.

\nedtype{DataLinkVisualizer} currently observes packets that pass through the data link
layer (i.e. carry data from/to higher layers), but not those that are internal
to the operation of the data link layer protocol. That is, frames such as ACK,
RTS/CTS, Beacon or Authentication/Association frames of IEEE 802.11, although
potentially useful, will not trigger the visualization. Visualizing such frames
may be implemented in future INET revisions.

The activity between two nodes is represented visually by an arrow that points
from the sender node to the receiver node. The arrow appears after the first
packet has been received, then gradually fades out unless it is refreshed by
further packets. The style, color, fading time and other graphical properties
can be changed with parameters of the visualizer.

By default, all packets, interfaces and nodes are considered for the
visualization. This selection can be narrowed to certain packets and/or nodes
with the visualizer's \fpar{packetFilter}, \fpar{interfaceFilter}, and
\fpar{nodeFilter} parameters.


\subsection{Visualizing Physical Link Activity}
\label{sec:visualization:physical-link-activity}

Physical link activity can be visualized by including a \nedtype{PhysicalLinkVisualizer}
module in the simulation. Adding an \nedtype{IntegratedVisualizer} module is also an
option, because it also contains a \nedtype{PhysicalLinkVisualizer} module. Physical link
activity visualization is disabled by default, it can be enabled by setting the
visualizer's \fpar{displayLinks} parameter to true.

\nedtype{PhysicalLinkVisualizer} observes frames that pass through the physical layer,
i.e. are received correctly.

The activity between two nodes is represented visually by a dotted arrow which
points from the sender node to the receiver node. The arrow appears after the
first frame has been received, then gradually fades out unless it is refreshed
by further frames. Color, fading time and other graphical properties can be
changed with parameters of the visualizer.

By default, all packets, interfaces and nodes are considered for the
visualization. This selection can be narrowed with the visualizer's
\fpar{packetFilter}, \fpar{interfaceFilter}, and \fpar{nodeFilter} parameters.

\ifdraft
\subsection{Visualizing the Transmission Medium}

TODO revise!

In order to help understanding the communication in the network the physical
layer supports visualizing its state. The following list shows what can be
displayed:

\begin{itemize}
  \item ongoing transmissions
  \item recent successful receptions
  \item recent obstacle intersections and surface normal vectors
\end{itemize}

The ongoing transmissions can be displayed with 3 dimensional spheres or with 2
dimensional rings laying in the XY plane. As the signal propagates through space
the figure grows with it to show where the beginning of the signal is. The inner
circle of the ring figure shows as the end of the signal propagates through
space. 

The recent successful receptions are displayed as straight lines between the
original positions of the transmission and the reception. The recent obstacle
intersections are also displayed as straight lines from the start of the
intersection to the end of it.
\fi

\subsection{Visualizing Routing Tables}
\label{sec:visualization:routing-tables}

In a complex network topology, it is difficult to see how a packet would be
routed because the relevant data is scattered among network nodes and hidden in
their routing tables. INET contains support for visualization of routing tables,
and can display routing information graphically in a concise way. Using
visualization, it is often possible to understand routing in a simulation
without looking into individual routing tables. The visualization currently
supports IPv4.

The \nedtype{RoutingTableVisualizer} module (included in the network as part of
\nedtype{IntegratedVisualizer}) is responsible for visualizing routing table entries.

The visualizer basically annotates network links with labeled arrows that
connect source nodes to next hop nodes. The module visualizes those routing
table entries that participate in the routing of a given set of destination
addresses, by default the addresses of all interfaces of all nodes in the
network. That is, it selects the best (longest prefix) matching routes for all
destination addresses from each routing table, and shows them as arrows that
point to the next hop. Note that one arrow might need to represent several
routing entries, for example when distinct prefixes are routed towards the same
next hop.

Routing table entries are represented visually by solid arrows. An arrow going
from a source node represents a routing table entry in the source node's routing
table. The endpoint node of the arrow is the next hop in the visualized routing
table entry. By default, the routing entry is displayed on the arrows in
following format:

\begin{verbatim}
destination/mask -> gateway (interface)
\end{verbatim}

The format can be changed by setting the visualizer's \fpar{labelFormat} parameter.

Filtering is also possible. The \fpar{nodeFilter} parameter controls which nodes'
routing tables should be visualized (by default, all nodes), and the
\fpar{destinationFilter} parameter selects the set of destination nodes to consider
(again, by default all nodes.)

The visualizer reacts to changes. For example, when a routing protocol changes a
routing entry, or an IP address gets assigned to an interface by DHCP, the
visualizer automatically updates the visualizations according to the specified
filters. This is very useful e.g. for the simulation of mobile ad-hoc networks.

\subsection{Displaying IP Addresses and Other Interface Information}
\label{sec:visualization:displaying-ip-addresses-and-other-interface-information}

In the simulation of complex networks, it is often useful to be able to display
node IP addresses, interface names, etc. above the node icons or on the links.
For example, when automatic address assignment is used in a hierarchical network
(e.g. using \nedtype{Ipv4NetworkConfigurator}), visual inspection can help to
verify that the result matches the expectations. While it is true that addresses and other
interface data can also be accessed in the GUI by diving into the interface
tables of each node, that is tedious, and unsuitable for getting an overview.

The \nedtype{InterfaceTableVisualizer} module (included in the network as part of
\nedtype{IntegratedVisualizer}) displays data about network nodes' interfaces.
(Interfaces are contained in interface tables, hence the name.) By default, the
visualization is turned off. When it is enabled using the
\fpar{displayInterfaceTables} parameter, the default is that interface names, IP
addresses and netmask length are displayed, above the nodes (for wireless
interfaces) and on the links (for wired interfaces). By clicking on an interface
label, details are displayed in the inspector panel.

The visualizer has several configuration parameters. The \fpar{format} parameter
specifies what information is displayed about interfaces. It takes a format
string, which can contain the following directives:

\begin{itemize}
  \item \%N: interface name
  \item \%4: IPv4 address
  \item \%6: IPv6 address
  \item \%n: network address. This is either the IPv4 or the IPv6 address
  \item \%l: netmask length
  \item \%M: MAC address
  \item \%\textbackslash: conditional newline for wired interfaces. The '\textbackslash'
  needs to be escaped with another '\textbackslash', i.e. '\%\textbackslash\textbackslash'
  \item \%i and \%s: the info() and str() functions for the interfaceEntry class, respectively
\end{itemize}

The default format string is
\texttt{"\%N \%\textbackslash\textbackslash\%n/\%l"}, i.e. interface name, IP address and
netmask length.

The set of visualized interfaces can be selected with the configurator's
\fpar{nodeFilter} and \fpar{interfaceFilter} parameters. By default, all
interfaces of all nodes are visualized, except for loopback addresses (the default for the
\fpar{interfaceFilter} parameter is \texttt{"not lo\textbackslash*"}.)

It is possible to display the labels for wired interfaces above the node icons,
instead of on the links. This can be done by setting the
\fpar{displayWiredInterfacesAtConnections} parameter to false.

There are also several parameters for styling, such as color and font selection.


\subsection{Visualizing IEEE 802.11 Network Membership}
\label{sec:visualization:ieee-80211-network-membership}

When simulating wifi networks that overlap in space, it is difficult to see
which node is a member of which network. The membership may even change over
time. It would be useful to be able to display e.g. the SSID above node icons.

IEEE 802.11 network membership can be visualized by including a
\nedtype{Ieee80211Visualizer} module in the simulation. Adding an \nedtype{IntegratedVisualizer} is
also an option, because it also contains a \nedtype{Ieee80211Visualizer}. Displaying
network membership is disabled by default, it can be enabled by setting the
visualizer's \fpar{displayAssociations} parameter to true.

The \nedtype{Ieee80211Visualizer} displays an icon and the SSID above network nodes which
are part of a wifi network. The icons are color-coded according to the SSID. The
icon, colors, and other visual properties can be configured via parameters of
the visualizer.

The visualizer's \fpar{nodeFilter} parameter selects which nodes' memberships are
visualized. The \fpar{interfaceFilter} parameter selects which interfaces are
considered in the visualization. By default, all interfaces of all nodes are
considered.


\subsection{Visualizing Transport Connections}
\label{sec:visualization:transport-connections}

In a large network with a complex topology, there might be many transport layer
applications and many nodes communicating. In such a case, it might be difficult
to see which nodes communicate with which, or if there is any communication at
all. Transport connection visualization makes it easy to get information about
the active transport connections in the network at a glance. Visualization makes
it easy to identify connections by their two endpoints, and to tell different
connections apart. It also gives a quick overview about the number of
connections in individual nodes and the whole network.

The \nedtype{TransportConnectionVisualizer} module (also part of \nedtype{IntegratedVisualizer})
displays color-coded icons above the two endpoints of an active, established
transport layer level connection. The icons will appear when the connection is
established, and disappear when it is closed. Naturally, there can be multiple
connections open at a node, thus there can be multiple icons. Icons have the
same color at both ends of the connection. In addition to colors, letter codes
(A, B, AA, …) may also be displayed to help in identifying connections. Note
that this visualizer does not display the paths the packets take. If you are
interested in that, take a look at \nedtype{TransportRouteVisualizer}, covered in
section \ref{sec:visualization:transport-path-activity}.

The visualization is turned off by default, it can be turned on by setting the
\fpar{displayTransportConnections} parameter of the visualizer to true.

It is possible to filter the connections being visualized. By default, all
connections are included. Filtering by hosts and port numbers can be achieved by
setting the \fpar{sourcePortFilter}, \fpar{destinationPortFilter},
\fpar{sourceNodeFilter} and \fpar{destinationNodeFilter} parameters.

The icon, colors and other visual properties can be configured by setting the
visualizer's parameters.


\section{Visualizing The Infrastructure}
\label{sec:visualization:the-infrastructure}

\subsection{Visualizing the Physical Environment}
\label{sec:visualization:the-physical-environment}

The physical environment has a profound effect on the communication of wireless
devices. For example, physical objects like walls inside buildings constraint
mobility. They also obstruct radio signals often resulting in packet loss. It's
difficult to make sense of the simulation without actually seeing where physical
objects are.

The visualization of physical objects present in the physical environment is
essential.

The \nedtype{PhysicalEnvironmentVisualizer} (also part of \nedtype{IntegratedVisualizer}) is
responsible for displaying the physical objects. The objects themselves are
provided by the PhysicalEnvironment module; their geometry, physical and visual
properties are defined in the XML configuration of the PhysicalEnvironment
module.

The two-dimensional projection of physical objects is determined by the
\nedtype{SceneCanvasVisualizer} module. (This is because the projection is also needed by
other visualizers, for example \nedtype{MobilityVisualizer}.) The default view is top view
(z axis), but you can also configure side view (x and y axes), or isometric or
ortographic projection.

The visualizer also supports OpenGL-based 3D rendering using the OpenSceneGraph
(OSG) library. If the OMNeT++ installation has been compiled with OSG
support, you can switch to 3D view using the Qtenv toolbar.

\subsection{Visualizing Node Mobility}
\label{sec:visualization:node-mobility}

In INET simulations, the movement of mobile nodes is often as important as the
communication among them. However, as mobile nodes roam, it is often difficult
to visually follow their movement. INET provides a visualizer that not only
makes visually tracking mobile nodes easier, but also indicates other properties
like speed and direction.

Node mobility of nodes can be visualized by \nedtype{MobilityVisualizer} module
(included in the network as part of \nedtype{IntegratedVisualizer}). By default,
mobility visualization is enabled, it can be disabled by setting
\fpar{displayMovements} parameter to false.

By default, all mobilities are considered for the visualization. This selection
can be narrowed with the visualizer's \fpar{moduleFilter} parameter.

The visualizer has several important features:

\begin{itemize}
  \item Movement Trail: It displays a line along the recent path of movements.
        The trail gradually fades out as time passes. Color, trail length and
        other graphical properties can be changed with parameters of the
        visualizer.
  \item Velocity Vector: Velocity is represented visually by an arrow. Its
        starting point is the node, and its direction coincides with the
        movement's direction. The arrow's length is proportional to the node's
       speed.
  \item Orientation Arc: Node orientation is represented by an arc whose size
       is specified by the \fpar{orientationArcSize} parameter. This value is the
       relative size of the arc compared to a full circle. The arc's default
       value is 0.25, i.e. a quarter of a circle.
\end{itemize}

These features are disabled by default; they can be enabled by setting the
visualizer's \fpar{displayMovementTrails}, \fpar{displayVelocities} and
\fpar{displayOrientations} parameters to true.


%%% Local Variables:
%%% mode: latex
%%% TeX-master: "usman"
%%% End:


\cleardoublepage

\include{ch-authors-guide}
\cleardoublepage

\include{ch-history}
\cleardoublepage

\bibliographystyle{alpha}
\bibliography{inet-users-guide}


%% no need for the following since 'tocbibind' package
%% \phantomsection
%% \addcontentsline{toc}{chapter}{\indexname}
\printindex

\end{document}

%%% Local Variables:
%%% mode: latex
%%% TeX-master: t
%%% End:
